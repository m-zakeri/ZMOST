% !TeX TS-program = XeLaTeX

% Commands for running this thesis:
% $ xelatex main
% $ bibtex8 -W -c cp1256fa maintext
% $  xindy -L persian -C utf8 -M texindy maintext
% $ xelatex maintext
% $ xelatex maintext
% End of Commands

%%%%%%%%%%%%%%
%% Welcome to Latex Template for IUST CE Thesis
%% The v0.8 quotes
%% The nearest star is 4.25 light-years from Earth which is why most wishes take at least 9 years to come true :)
%%%%%%%%%%%%%%% 

%%%
%%% خوش آمدید!
% قالب پایان‌نامه‌های کارشناسی، ارشد و دکتری، دانشگاه علم و صنعت ایران، دانشکده مهندسی کامپیوتر، آماده شده با استفاده از کلاس IUST-Thesis.
% نــگارش 0.9 توسط مرتضی ذاکری نصرآبادی - اردیبهشت 1399
% نــگارش 0.8 توسط مرتضی ذاکری نصرآبادی - مهر 1398
% نــگارش 0.7 توسط مرتضی ذاکری نصرآبادی  - تیر 1397
% - نــگارش 0.7 براساس نگارش 0.6 دکتر محمود امین طوسی آماده شده است.
% - این نسخه بر اساس نگارش 0.7 ساخته شده است. 

%استفاده از قالب با رعایت حقوق نشر، آزاد است.
%%% 
% نگارش 0.6 توسط محمود امین‌طوسی، دانشگاه تربیت معلم سبزوار، http://profsite.sttu.ac.ir/mamintoosi/
% گروه پارسی‌لاتک  http://www.parsilatex.com
% نگارش 0.6، بر اساس نسخه‌ 0.4 از کلاس Tabriz_Thesis آقای وحید دامن‌افشان آماده شده است؛ رجوع شود به: http://damanafshan.tk
%%%%%%%%%%%%%%%%%%%%%%%%%%%%% 


%%% تغییرات 
% نسخه 0.8:
% - افزودن بسته \usepackage{verbatim} برای مدیریت توضیحات متن
% - بروزرسانی گرافیک و فونت صفحه عنوان فارسی
% - متناسب سازی برای رساله دکتری

% نسخه 0.7: 
% - بروز رسانی بسته‌ها، اصلاح مشکلات مربوط به عدم تطابق با texlive2018 و اصلاحات جزئی دیگر. پرسش و پاسخ و ارسال دیدگاه‌ها:  
% http://m-zakeri.github.io
% -ریز تغییرات به شرح زیر است:
% -- اضافه شدن بسته‌(ها)ی مدیریت واژگان و کوته‌نوشت‌ها به صورت خودکار 
% -- اضافه شدن بسته‌ها و تنظیمات مربوط به درج کدمنبع برنامه‌ها در متن پایان نامه.
% -- اضافه شدن بسته‌ها و تنظیمات مروبط به درج الگوریتم‌ در استانداردهای نوشتاری نوین در متن پایان‌نامه.
% -- اضافه شدن بسته‌ها و تنظیمات مربوط به درج جدول در استانداردهای نوشتاری نوین در متن پایان‌نامه.
% -- اضافه شدن بسته‌ها و تنظیمات درج bookmark در خروجی.
% -- اضافه شدن بسته‌ها و تنظیمات درج صحیح پاورقی‌ها
% -- اضافه شدن بسته‌ها و تنظیمات مربوط به ایجاد نسخه‌های پیش‌نویس و نیز متن پس‌زمینه
% -- اضافه شدن بسته‌ها و تنظیمات مربوط به درج نقل قول‌های مستقیم و غیر مستقیم و epigraph
% -- اضافه شدن سایر بسته‌های مربوط به مدیریت فونت و استایل پایان‌نامه و ساختاربندی مجزای تصاویر و کدها (ارائه پارتیشن‌بندی استاندارد).

 
% نسخه 0.6:
% -- اصلاح مشکل بسته subfig 
%%%%%%%%%%%%%%%%%%%%%%%%%%%%


% -- اگر قصد نوشتن پروژه کارشناسی را دارید، در خط زیر به جای msc، کلمه bsc و اگر قصد نوشتن پروژه دکترا را دارید، کلمه phd را قرار دهید. کلیه تنظیمات لازم، به طور خودکار، اعمال می‌شود.
% -- اگر مایلید پایان‌نامه  دو رو باشد به جای oneside در خط زیر از twoside استفاده کنید.
\documentclass[twoside, openany, phd]{IUST-Thesis} % draft

% -- فایل settings.tex را مطالعه کنید. دستورات مربوط به فراخوانی بسته زی‌پرشین و دیگر بسته‌ها، در این فایل قرار دارد. بهتر است که با نحوه استفاده از آنها آشنا شوید.
\input{settings}
\input{settings-glossary}
% !TeX root=z-main.tex
% !TeX TS-program = XeLaTeX
% !TEX spellcheck = fa

%%
% My Thesis Acronyms
%%

%% Chapter1
\newacronym{MOST}{MOST}{Modern Open Scientific / Standard Templates}
\newacronym{SUT}{SUT}{software under test}
\newacronym{NLP}{NLP}{natural language processing}
\newacronym{SDLC}{SDLC}{software development life cycle}
\newacronym{ASE}{ASE}{automated software engineering}
\newacronym{EaASE}{EaASE}{empirical and automated software engineering}

%% Chapter2
\newacronym{AST}{AST}{abstract syntax tree}

%% Chapter3
%\newacronym{AFL}{AFL}{\lr{American Fuzzy Lop}}
\newacronym{TFIDF}{TF-IDF}{term frequenc-inverse document frequency}

%% Chapter4
\newacronym{ASCII}{ASCII}{american standard code for information interchange}

%% Chapter5
\newacronym{SOU}{SOU}{single object update}


%% Chapter6
\newacronym{GenerativeAdversarialNetwork}{GAN}{generative adversarial network}
\newacronym{CAN}{CAN}{creative adversarial networks}

% !TeX root=z-main.tex
% !TeX TS-program = XeLaTeX
% !TEX spellcheck = fa
%glassory

%%
% CS Glossary (Universal glassory).
% To collect a complete reference, please do not remove any word and instead comment them.
%%

%%
% جستجوی واژه‌های مصوّب فرهنگستان:
% https://wiki.apll.ir/word/index.php/Paradigm
%%

%%% Chapter1

%% Problem statement glassory
\newword{divide-and-conquer}{divide and conquer}{تقسیم و غلبه}{تقسیم و غلبه}
\newword{software-testing}{software testing}{آزمون ‌نرم‌افزار}{آزمون‌های نرم‌افزار}
\newword{test-data-generation}{test data generation}{تولید داده آزمون}{تولید داده‌های آزمون}
\newword{debugging}{debugging}{اشکال‌زدایی}{اشکال‌زدایی}
\newword{fault-localization}{fault localization}{مکان‌یابی خطا}{مکان‌یابی خطاها}
\newword{repair}{repair}{ترمیم}{ترمیم‌ها}
\newword{side-effect}{side effect}{تأثیر جانبی}{تأثیرات جانبی}
\newword{alias-name}{alias name}{اسم مستعار}{اسامی مستعار}
\newword{testability}{testability}{آزمون‌پذیری}{قابلیت‌ آزمون}

\newword{clean-code}{clean code}{‌کدِ پاک}{کدهای پاک}
\newword{refactoring}{refactoring}{بازآرایی}{بازآرایی‌ها}

\newword{source-code}{source code}{کدِ منبع}{کدهای منبع}
\newword{design}{design}{طراحی}{طراحی‌ها}
\newword{requirement}{requirement}{نیازمندی}{نیازمندی‌ها}

\newword{unit-test}{unit test}{آزمون واحد}{آزمون‌ واحد‌ها}
\newword{module-test}{module test}{آزمون پیمانه}{آزمون پیمانه‌ها}
\newword{integration-test}{integration test}{آزمون یکپارچگی}{آزمون‌های یکپارچگی}
\newword{system-test}{system test}{آزمون سیستم}{آزمون سیستم‌ها}
\newword{acceptance-test}{acceptance test}{آزمون پذیرش}{آزمون پذیرش}
\newword{developer-testing}{developer testing}{آزمون توسعه‌دهنده}{آزمون توسعه‌دهنده}
\newword{regression-testing}{regression testing}{آزمون رگرسیون}{آزمون رگرسیون}

\newword{quality-attribute}{quality attribute}{صفت کیفیت}{صفات کیفیت}
\newword{internal-quality-attribute}{internal quality attribute}{صفت کیفیت داخلی}{صفات کیفیت داخلی}
\newword{external-quality-attribute}{external quality attribute}{صفت کیفیت خارجی}{صفات کیفیت خارجی}

\newword{search-based}{search-based}{مبتنی بر جست‌و‌جو}{مبتنی بر جست‌وجو}
\newword{evolutionary}{evolutionary}{تکاملی}{تکاملی}
\newword{meta-heuristic}{meta-heuristic}{فرااکتشافی}{فرااکتشافی}
\newword{objective-function}{objective-function}{تابع هدف}{توابع هدف}

\newword{coverage}{coverage}{پوشش}{پوشش‌ها}
\newword{coverage-criteria}{coverage criteria}{معیارهای پوشش}{معیارهای پوشش}
\newword{symbolic-execution}{symbolic execution}{اجرای نمادین}{اجراهای نمادین}
\newword{dynamic-symbolic-execution}{dynamic symbolic execution}{اجرای نمادینِ پویا}{اجراهای نمادینِ پویا}
\newword{concolic-execution}{concolic execution}{اجرای واقعی-نمادین}{اجراهای واقعی-نمادین}
\newword{tester}{tester}{آزمون‌گر}{آزمون‌گرها}

\newword{single-responsibility-principle}{single responsibility pronciple}{اصل تک مسئولیتی}{اصل تک مسئولیتی}
\newword{dependency-injection}{dependency-injection}{تزریق وابستگی}{تزریق وابستگی}
\newword{design-pattern}{design pattern}{الگوی طراحی}{الگوهای طراحی}
\newword{factory}{factory}{کارخانه}{کارخانه}
\newword{polymorphism}{polymorphism}{چندریختی}{چندریختی}

\newword{long-method}{long method}{تابع طولانی}{توابع طولانی}
\newword{large-class}{large class}{کلاس بزرگ}{کلاس‌های بزرگ}
\newword{extract-method}{extract method}{استخراج تابع}{استخراج توابع}
\newword{extract-class}{extract class}{استخراج کلاس}{استخراج کلاس‌ها}
\newword{move-method}{move method}{انتقال تابع}{انتقال تابع}
\newword{move-class}{move class}{انتقال کلاس}{انتقال کلاس‌ها}

\newword{dependency-graph}{dependency graph}{گراف وابستگی}{گراف وابستگی‌ها}
\newword{component}{component}{مؤلفه}{مؤلفه‌ها}
\newword{module}{module}{پیمانه}{پیمانه‌ها}
\newword{modularity}{modularity}{پیمانگی}{پیمانگی}
\newword{cohesion}{cohesion}{چسبندگی}{چسبندگی}
\newword{coupling}{coupling}{اتصال}{اتصال}

\newword{reliability}{reliability}{قابلیت اطمینان}{قابلیت اطمینان}
\newword{understandability}{understandability}{قابلیت درک}{قابلیت درک}
\newword{changeability}{changeability}{قابلیت تغییر}{قابلیت تغییر}
\newword{reusability}{reusability}{قابلیت استفاده مجدد}{قابلیت استفاده مجدد}
\newword{maintainability}{maintainability}{قابلیت نگه‌داشت}{قابلیت نگه‌داشت}

\newword{sequence}{sequence}{توالی}{توالی‌ها}
\newword{optimum}{optimum}{بهینه}{بهینه}
\newword{feasible}{feasible}{امکان‌پذیر}{امکان‌پذیر}

\newword{software-smell}{software smell}{بوی نرم‌افزار}{بوهای نرم‌افزار}
\newword{code-smell}{code smell}{بوی کد}{بوی کد}
\newword{design-smell}{design smell}{بوی طراحی}{بوی طراحی}
\newword{architecture-smell}{architecture smell}{بوی معماری}{بوهای معماری}
\newword{test-smell}{test smell}{بوی آزمون}{بوهای آزمون}
\newword{requirement-smell}{requirement smell}{بوی نیازمندی}{بوهای نیازمندی}
\newword{smelly-code}{smelly code}{کد بودار}{کدهای بودار}
\newword{smelly-word}{smelly word}{واژه بودار}{واژه‌های بودار}

\newword{polysemy}{polysemy}{چندمعنایی}{چندمعنایی}
\newword{subjective-language}{subjective-language}{زبان موضوعی}{زبان موضوعی}
\newword{non-verifiable-term}{non-verifiable term}{اصطلاح غیرقابل درستی‌یابی}{اصطلاحات غیرقابل درستی‌یابی}
\newword{ambiguous}{ambiguous}{مبهم}{مبهم}
\newword{superlative}{superlative}{صفت عالی}{صفات عالی}
\newword{comparative}{comparative}{صفت تفضیلی}{صفات تفضیلی}
\newword{vague-pronouns}{vague pronouns}{ضمیر مبهم}{ضمیر مبهم}
\newword{uncertain-verbs}{uncertain-verbs}{فعل نامعیّن}{افعال نامعیّن}

\newword{criteria}{criteria}{معیار}{معیارها}
\newword{metric}{metric}{متریک}{متریک‌ها}
\newword{measure}{measure}{سنجه}{سنجه‌ها}

%% Related work glassory
\newword{cyclomatic-complexity}{cyclomatic complexity}{پیچیدگی حلقوی}{پیچیدگی حلقوی}
\newword{test-adequacy-criteria}{test adequacy criteria}{معیار کفایت آزمون}{معیارهای کفایت آزمون}

\newword{controllability}{controllability}{کنترل‌پذیری}{کنترل‌پذیری}
\newword{observability}{observability}{مشاهده‌پذیری}{مشاهده‌پذیری}
\newword{testability-transformation}{testability transformation}{تبدیل آزمون‌پذیری}{تبدیل آزمون‌پذیری}
\newword{user-friendly}{user friendly}{کاربر پسند}{کاربر پسند}
\newword{verification}{verification}{درستی‌یابی}{درستی‌یابی}
\newword{validation}{validation}{اعتبارسنجی}{اعتبارسنجی}

\newword{reverse-engineering}{reverse engineering}{مهندسی معکوس}{مهندسی معکوس}

\newword{coveragability}{coveragability}{پوشش‌پذیری}{پوشش‌پذیری}
\newword{code-coverage}{code coverage}{پوشش کد}{پوشش کد}
\newword{statement-coverage}{statement coverage}{پوشش دستور}{پوشش دستور}
\newword{branch-coverage}{branch coverage}{پوشش شاخه}{پوشش شاخه}
\newword{test-suite}{test suite}{مجموعه آزمون}{مجموعه‌های آزمون}

\newword{warm-start}{warm start}{شروع گرم}{شروع گرم}
\newword{cold-start}{cold start}{شروع سرد}{شروع سرد}

\newword{multi-objective}{multi objective}{چند هدفه}{چند هدفه}
\newword{many-objective}{many objective}{چندین هدفه}{چندین هدفه}

\newword{remodularization}{remodularization}{پیمانه‌سازی مجدد}{پیمانه‌سازی‌های مجدد}
\newword{restructuring}{restructuring}{تجدید ساختار}{تجدید ساختار}
\newword{developer-experience}{developer experience}{تجربه توسعه‌دهنده}{تجربه توسعه‌دهندگان}
%%
\newword{requirement-clarity}{requirement clarity}{وضوح نیازمندی}{وضوح نیازمندی}
\newword{clean-requirement}{claen requirement}{نیازمندی پاک}{نیازمندی‌های پاک}
\newword{word-embedding}{word embedding}{تعبیه‌واژه}{تعبیه‌واژه}

\newword{requirement-refactoring}{requirement refactoring}{بازآرایی نیازمندی}{بازآرایی نیازمندی‌ها}
\newword{semi-automated}{semi-automated}{نیمه‌خودکار}{نیمه‌خودکار}
\newword{fully-automated}{fully automated}{‌خودکار}{‌خودکار}

\newword{object-oriented}{object-oriented}{شیء‌گرا}{شی‌ءگرایی}
\newword{aspect-oriented}{aspect-oriented}{جنبه‌گرا}{جنبه‌گرا}
\newword{service-oriented}{service-oriented}{سرویس‌گرا}{سرویس‌گرا}
\newword{structural}{structured}{ساخت‌یافته}{ساخت‌یافته}

\newword{static-analysis}{static-analysis}{تحلیل ایستا}{تحلیل‌های ایستا}
\newword{dynamic-analysis}{dynamic-analysis}{تحلیل پویا}{تحلیل‌های پویا}

\newword{safety-critical-system}{safety critical system}{سیستم‌ حساس به ایمنی}{سیستم‌های حساس به ایمنی}


%\gls{refactoring}

%\newword{FuzzTesting}{fuzz testing}{آزمون فازی}{آزمون فازی}
%\newword{Monitor}{monitor}{پایش}{پایش}

%\newword{Vulnerability}{vulnerability}{آسیب‌پذیری}{آسیب‌پذیری‌ها}
%\newword{Malformed}{malformed}{بدشکل}{بدشکل}
\newword{test-case}{test case}{آزمایه}{آزمایه‌‌ها}
\newword{test-data}{test data}{داده آزمون}{داده‌های آزمون}
\newword{test-double}{test double}{همزاد آزمون}{ همزدا آزمون}

\newword{parser}{parser}{تجزیه‌گر}{تجزیه‌گرها}

%\newword{MagicByte}{magic bytes}{بایت جادویی}{بایت‌های جادویی}
%\newword{Marker}{markers}{نشان‌گر}{نشان‌گرها}
\newword{parse}{parse}{تجزیه}{تجزیه}
\newword{render}{render}{پرداخت}{پرداخت}

%\newword{MemoryCorruption}{memory corruption}{فساد حافظه}{فساد حافظه}
%\newword{Metadata}{metadata}{فراداده}{فراداده}

%\newword{Ransomware}{ransomware}{باج‌افزار}{باج‌افزارها}
%\newword{Attacker}{attacker}{مهاجم}{مهاجمان}

%\newword{ExceptionHandling}{exception handling}{کنترل استثنا}{کنترل استثنا}
%\newword{InternetOfThings}{internet of things}{اینترنت اشیاء}{اینترنت اشیاء}


%%%%%%%%%%%%%%%%%%%%%%%%%%%%%%
%% Chapter2
\newword{fault}{fault}{خطا}{خطاها}
\newword{error}{error}{اشکال}{اشکال‌ها}
\newword{failure}{failure}{خرابی}{خرابی‌ها}
\newword{specification}{specification}{مشخصه}{مشخصه‌ها}
\newword{dependability}{dependability}{اتکاپذیری}{اتکاپذیری}
\newword{undecidable}{undecidable}{تصمیم‌ناپذیر}{تصمیم‌ناپذیر}

\newword{InputSpacePartitioning}{input space partitioning}{افراز فضای ورودی}{افراز فضای ورودی}
\newword{GraphCoverage}{graph coverage}{پوشش گراف}{پوشش گراف}
\newword{LogicCoverage}{logic coverage}{پوشش منطق}{پوشش منطق}
\newword{SyntaxBasedCoverage}{syntax-based coverage}{پوشش ساختار نحوی}{پوشش ساختار نحوی}

\newword{RecognizerProgram}{recognizer program}{برنامه شناسنده}{برنامه شناسنده}
\newword{GeneratorProgram}{generator program}{برنامه مولد}{برنامه مولد}

\newword{line-coverage}{line coverage}{پوشش خط}{پوشش خط}
\newword{partial-line-coverage}{partial line coverage}{پوشش خط جزئی}{پوشش خط جزئی}
\newword{basic-block}{basic block}{بلوک پایه}{بلوک پایه}
\newword{basic-block-coverage}{basic block coverage}{پوشش بلوک پایه}{پوشش بلوک پایه}
\newword{path-coverage}{path coverage}{پوشش مسیر}{پوشش مسیر}
\newword{domain-coverage}{domain coverage}{پوشش دامنه}{پوشش دامنه}
\newword{Instrumenting}{instrumenting}{ابزارگذاری}{ابزارگذاری}

\newword{software-sizing}{software sizing}{اندازه نرم‌‌افزار}{اندازه نرم‌افزار}
\newword{parse-tree}{parse tree}{درخت تجزیه}{درخت تجزیه}
\newword{symbol-table}{symbol table}{جدول نماد}{جدول نمادها}
\newword{class-diagram}{class diagram}{نمودار کلاس}{نمودارهای کلاس}

\newword{objective}{objective}{عینی}{عینی}

\newword{anti-pattern}{anti-pattern}{ضد الگو}{ضد الگو}
\newword{technical-debt}{technical debt}{بدهی فنی}{بدهی‌های فنی}
\newword{code-debt}{test debt}{بدهی کد}{بدهی‌های کد}
\newword{design-debt}{design debt}{بدهی طراحی}{بدهی‌های طراحی}
\newword{test-debt}{test debt}{بدهی آزمون}{بدهی‌های آزمون}

\newword{roll-back}{roll-back}{بازگرد}{بازگردها}
\newword{anomally-detection}{anomally detection}{تشخیص ناهنجاری}{تشخیص ناهنجاری}

\newword{imbalanced}{imbalanced}{نامتوازن}{نامتوازن}


\newword{software-analytics}{software-analytics}{واکاوش نرم‌افزار}{واکاوش نرم‌افزار}
\newword{data-driven}{data-driven}{داده‌رانه}{داده‌رانه}
\newword{data-science}{data-science}{علم داده‌ها}{علم ‌داده‌ها}

\newword{representation}{representation}{بازنمایی}{بازنمایی}
\newword{concept}{concept}{مفهوم}{مفاهیم}
\newword{supervised}{supervised}{بانظارت}{بانظارت}
\newword{supervised-learning}{supervised learning}{یادگیری بانظارت}{یادگیری بانظارت}
\newword{unsupervised-learning}{unsupervised learning}{یادگیری بدون‌ نظارت}{یادگیری بودن ‌نظارت}
\newword{semi-supervised-learning}{semi-supervised learning}{یادگیری نیمه‌ نظارتی}{یادگیری بودن ‌نظارتی}
\newword{reinforcement-learning}{reinforcement learning}{یادگیری تقویتی}{یادگیری تقویتی}

\newword{multi-agent-learning}{multi-agent learning}{یادگیری چند عاملی}{یادگیری چند عاملی}

\newword{label}{label}{برچسب}{برچسب}
\newword{task}{task}{وظیفه}{وظایف}
\newword{regression}{regression}{رگرسیون}{رگرسیون‌ها}
\newword{classification}{classification}{طبقه‌بندی}{‌طبقه‌بندی‌ها}
\newword{clustering}{clustering}{خوشه‌بندی}{خوشه‌بندی‌ها}


\newword{lemmatization}{lemmatization}{بن واژه‌سازی}{بن ‌واژه‌سازی}
\newword{stemming}{stemming}{ریشه‌یابی}{ریشه‌یابی}


%% Not used in proposal
\newword{ShortCircuit}{short circuit}{مدار کوتاه}{مدار کوتاه}

\newword{BlackBox}{black box}{جعبه سیاه}{جعبه سیاه}
\newword{WhiteBox}{white box}{جعبه سفید}{جعبه سفید}
\newword{GrayBox}{gray box}{جعبه خاکستری}{جعبه خاکستری}

\newword{StressTesting}{stress testing}{آزمون فشار}{آزمون فشار}
\newword{Fuzzer}{fuzzer}{فازر}{فازر}
\newword{MutationBased}{mutation based}{مبتنی بر جابه‌جایی}{مبتنی بر جابه‌جایی}

\newword{GenerationBased}{generation based}{مبتنی بر تولید}{مبتنی بر تولید}
\newword{SampleInput}{sample input}{ورودی نمونه}{ورودی نمونه}
\newword{InputData}{input data}{داده ورودی}{داده ورودی}
\newword{Valid}{Valid}{معتبر}{معتبر}
\newword{InitialSeed}{initial seed}{دانه اولیه}{دانه‌های اولیه}
\newword{GenerativeModel}{generative model}{مدل مولد}{مدل‌های مولد}
\newword{Module}{module}{پیمانه}{پیمانه‌ها}
\newword{FileFormatFuzzer}{file format fuzzer}{فازر قالب فایل}{فازر قالب فایل}

\newword{PenetrationTesting}{penetration testing}{آزمـون نفوذ}{آزمون‌های نفوذ}
\newword{RobustnessTesting}{robustness testing}{آزمون قدرت‌مندی}{آزمون‌های قدرت‌مندی}
\newword{SecurityTesting}{security testing}{آزمون امنیت}{آزمون‌های امنیت}

\newword{Exploit}{exploit}{بهره‌برداری}{بهره‌برداری}
\newword{Buffer}{buffer}{میانگیر}{میانگیرها}

\newword{DeepLearning}{deep learning}{یادگیری ژرف}{یادگیری ژرف}
\newword{NeuralNetwork}{neural network}{شبکه عصبی}{شبکه‌های عصبی}
\newword{DeepNeuralNetwork}{deep neural network}{شبکه عصبی ژرف}{شبکه عصبی ژرف}
\newword{FeedforwardNeuralNetwork}{feed-forward neural network}{شبکه عصبی روبه‌جلو}{شبکه‌های عصبی روبه‌جلو}

\newword{HiddenLayer}{hidden layer}{لایه پنهان}{لایه‌های پنهان}
\newword{Neuron}{neuron}{عصب}{عصب‌ها}
\newword{Unit}{unit}{واحد}{واحدها}
\newword{Bias}{bias}{بایاس}{بایاس}
\newword{ActivationFunction}{activation function}{تابع انگیزش}{توابع انگیزش}

\newword{TrainingSet}{training set}{مجموعه آموزش}{مجموعه‌های آموزش}
\newword{TestSet}{test set}{مجموعه آزمون}{مجموعه‌های آزمون}
\newword{ValidationSet}{validation set}{مجموعه ارزیابی}{مجموعه‌های ارزیابی}

\newword{ForwardPass}{forward pass}{گذرِجلو}{گذرِجلو}
\newword{ErrorFunction}{error function}{تابع خطا}{تابع خطا}
\newword{CostFunction}{cost function}{تابع هزینه}{توابع هزینه}
\newword{MeanAbsoluteError}{mean absolute error}{خطای مطلق میانگین}{خطای مطلق میانگین}
\newword{MeanSquaredError}{mean squared error}{خطای میانگین مربعات}{خطای میانگین مربعات}
\newword{BinaryCrossEntropyError}{binary cross-entropy}{خطای آنتروپی متقاطع دودویی}{خطای آنتروپی متقاطع دودویی}
\newword{CrossEntropyError}{cross-entropy error}{خطای آنتروپی متقاطع}{خطای آنتروپی متقاطع}
\newword{Hyperparameter}{Hyper-parameter}{ابَر پارامتر}{ابَر پارامتر}
\newword{Backpropagation}{back-propagation}{پس‌انتشار}{پس‌انتشار}
\newword{LearningRate}{learning rate}{نرخ یادگیری}{نرخ یادگیری}

\newword{Epoch}{epoch}{دوره}{دوره}
\newword{Regularization}{regularization}{منظم‌سازی}{منظم‌سازی}
\newword{Overfitting}{over-fitting}{بیش‌برازش}{بیش‌برازش}
\newword{Noise}{Noise}{اختلال}{اختلال‌ها}

\newword{TimeStep}{time step}{گام زمانی}{گام‌های زمانی}
\newword{Softmax}{softmax}{بیشینه هموار}{بیشینه هموار}
\newword{Step}{step}{گام}{گام‌ها}
\newword{Vanishing}{vanishing}{ناپدید شدن}{ناپدید شدن}
\newword{Explosion}{explosion}{انفجار}{انفجار}
\newword{LanguageModel}{Language model}{مدل زبانی}{مدل‌های زبانی}
\newword{Context}{context}{زمینه}{زمینه‌ها}
\newword{History}{history}{تاریخچه}{تاریخچه‌ها}
\newword{Smoothing}{smoothing}{هموارسازی}{هموارسازی}

\newword{Perplexity}{perplexity}{سرگشتگی}{سرگشتگی}
\newword{Vocabulary}{vocabulary}{واژگان}{واژگان}
\newword{BranchingFactor}{branching factor}{ضریب انشعاب}{ضریب انشعاب}

%\gls{component}
%\glspl{HiddenLayer}

%%%%%%%%%%%%%%%%%%%%%%%%%%%%%%
%%%%% modify alphabet capital letter until here
%% Chapter3 
\newword{explainability}{explainability}{شرح‌پذیری}{شرح‌پذیری}
\newword{model-inspection}{model inspection}{بازرسی مدل}{بازرسی مدل}
\newword{graph-embedding}{graph embedding}{تعبیه‌ گراف}{تعبیه گراف}
\newword{code-embedding}{code embedding}{تعبیه‌ کد}{تعبیه کد}
\newword{clustering-coefficient}{clustering-coefficient}{ضریب خوشه‌بندی}{ضرایب خوشه‌بندی}
\newword{context-vector}{context vector}{بردار زمینه}{بردار زمینه}

\newword{intrinsic-evaluation}{intrinsic evaluation}{ارزیابی درونی}{ارزیابی درونی}
\newword{extrinsic-evaluation}{extrinsic evaluation}{ارزیابی بیرونی}{ارزیابی بیرونی}
\newword{model-error}{model-error}{خطای مدل}{خطاهای مدل}

\newword{accuracy}{accuracy}{دقت}{دقت}
\newword{precision}{precision}{صحت}{صحت}
\newword{recall}{recall}{یادآوری}{یادآوری}

\newword{corpus}{corpus}{پیکره}{پیکره‌ها}


\newword{LearnAndFuzz}{Learn{\&}Fuzz}{یادگیری و فاز}{یادگیری و فاز}
\newword{EncoderDecoder}{Encoder-Decoder}{کدگذار-کدگشا}{کدگذار-کدگشا}
\newword{EndToEnd}{End to End}{انتهابه‌انتها}{انتهابه‌انتها}
\newword{Prefix}{Prefix}{پیشوند}{پیشوند}
\newword{Greedy}{Greedy}{حریصانه}{حریصانه}
\newword{Wellformed}{Well-formed}{خوش‌شکل}{خوش‌شکل}
\newword{Consistent}{Consistent}{سازگار}{سازگارها}
\newword{Token}{Token}{توکن}{توکن‌ها}
\newword{Portability}{Portability}{قابلیت حمل}{قابلیت حمل}
\newword{Component}{Component}{مؤلفه}{مؤلفه‌ها}

%\newword{AugmentedAFL}{Augmented-AFL}{\lr{AFL}افزوده}{\lr{AFL}افزوده}
%\newword{Augmented}{Augmented}{افزوده}{افزوده}

%\newword{VulnerablePattern}{Vulnerable Pattern}{الگوی آسيب‌پذير}{الگوهای آسیب‌پذیر}

%%%%%%%%%%%%%%%%%%%%%%%%%%%%%%
%% Chapter 4
\newword{stop-word}{stop word}{ایست واژه}{ایست واژه‌ها‌}
\newword{requirements-refactoring}{requirements refactoring}{بازآرایی نیازمندی‌ها}{بازآرایی‌های نیازمندی‌ها‌}
\newword{interactive}{interactive}{تعاملی}{تعاملی}
\newword{checklist}{checklist}{فهرست هم‌سنجی}{فهرست‌های هم‌سنجی}
\newword{artifact-consistency-graph}{artifact consistency graph}{گراف سازگاری مصنوعات}{گراف سازگاری مصنوعات}
\newword{tripartite-graph}{tripartite graph}{گراف سه‌بخشی}{گراف سه‌بخشی}


\newword{Unidirectional}{Unidirectional}{یک‌سویه}{یک‌سویه}
\newword{Bidirectional}{Bidirectional}{دوسویه}{دوسویه}
\newword{Forward}{Forward}{روبه‌جلو}{روبه‌جلو}
\newword{Backward}{Backward}{روبه‌عقب}{روبه‌عقب}
\newword{MergeFunction}{Merge Function}{تابع ادغام}{تابع ادغام}

\newword{MultinomialDistribution}{Multinomial Distribution}{توزیع چندجمله‌ای}{توزیع چندجمله‌ای}

\newword{Diversity}{Diversity}{تنوع}{تنوع}
\newword{BinaryToken}{Binary Token}{توکن دودویی}{توکن دودویی}


%% Chapter 5
\newword{Host}{Host}{میزبان}{میزبان}
\newword{Shuffle}{Shuffle}{درهم‌آمیزی}{درهم‌آمیزی}
\newword{BaselineCodeCoverage}{Baseline Code Coverage}{پوشش کد مبنا}{پوشش کد‌های مبنا}

%%%%%%%%%%%%%%%%%%%%%%%%%%%%%%
%% Chapter 6
\newword{Generalization}{Generalization}{تعمیم‌پذیر}{تعمیم‌پذیر}
\newword{Adaptive}{Adaptive}{تطبیق‌پذیر}{تطبیق‌پذیر}
\newword{Background}{Background}{پس‌زمینه}{پس‌زمینه‌ها}
\newword{BackgroundProcess}{Background Process}{پردازه پس‌زمینه}{پردازه‌های پس‌زمینه}
\newword{AntiFuzzing}{Anti-fuzzing}{ضد فازینگ}{ضد فازینگ}
\newword{SeedMinimization}{Seed Minimization}{کمینه‌سازی دانه}{کمینه‌سازی دانه}
\newword{Filter}{Filter}{پاک‌ساز}{پاک‌ساز}
\newword{Normalization}{Normalization}{نرمال‌سازی}{نرمال‌سازی}
\newword{Oracle}{Oracle}{سروش}{سروش}


%%%%%%%%%%%%%%%%%%%%%%%%%%%%%%
%% Apendix1
\newword{DataObject}{Data Object}{شیء داده}{اشیای داده‌}
\newword{Object}{Object}{شیء}{اشیاء}
\newword{Body}{Body}{بدنه}{بدنه‌ها}
\newword{CrossReferenceTable}{Cross-reference Table}{جدول ارجاع متقابل}{جدول ارجاعات متقابل}
\newword{Dictionary}{Dictionary}{فرهنگِ لغت}{فرهنگ‌های لغت}
\newword{BinaryStream}{Binary Stream}{جریان دودویی}{جریان دودویی}

\newword{Incremental}{Incremental}{افزایشی}{افزایشی}

%%%%%%%%%%%%%%%%%%%%%%%%%%%%%%
%% Apendix2
\newword{Configuration}{Configuration}{پیکربندی}{پیکربندی}
 
\newword{Gloss}{glossary}{واژه‌نامه}{واژه‌نامه‌ها}
\newword{Acronym}{acronym}{اختصار}{اختصارات}
\newword{Description}{description}{توصیف}{توصیف‌ها}
\newword{Draft}{draft}{پیش‌نویس}{پیش‌نویس‌ها}
\newword{Absorption}{absorption}{جذب}{جذب‌ها}
\newword{RandomVariable}{random variable}{متغیر تصادفی}{متغیرهای تصادفی}
\newword{Action}{action}{کنش}{کنش‌ها}
\newword{Optimization}{optimization}{بهینه‌سازی}{بهینه‌سازی}


\begin{document}
 %% Set spacing
\titlespacing\section{1pt}{2pt plus 2pt minus 2pt}{0pt plus 2pt minus 2pt}
\titlespacing\subsection{1pt}{2pt plus 2pt minus 2pt}{0pt plus 2pt minus 2pt}
\titlespacing\subsubsection{1pt}{2pt plus 2pt minus 2pt}{0pt plus 2pt minus 2pt}
%\setlength{\textfloatsep}{\baselineskip}
\setlength{\textfloatsep}{10pt}
%
%\setstretch{1}
\setlength{\abovedisplayskip}{1pt}
\setlength{\belowdisplayskip}{1pt}
%% End of set spacing


\pagenumbering{harfi} % شماره صفحه حرفی %
% -- مشخصات پایان‌نامه را در فایل‌های titlefa و titleen وارد نمایید.
% !TeX root=maintext.tex
% !TeX TS-program = XeLaTeX
% !TEX spellcheck = fa
% Farsi Title
% Persian absteract and other info
% در این فایل، عنوان پایان‌نامه، مشخصات خود، متن تقدیمی‌، ستایش، سپاس‌گزاری و چکیده پایان‌نامه را به فارسی، وارد کنید.
% توجه داشته باشید که جدول حاوی مشخصات پروژه/پایان‌نامه/رساله و همچنین، مشخصات داخل آن، به طور خودکار، درج می‌شود.
%%%%%%%%%%%%%%%%%%%%%%%%%%%%%%%%%%%%
% دانشگاه خود را وارد کنید
\university{علم و صنعت ایران}
% دانشکده، آموزشکده و یا پژوهشکده  خود را وارد کنید
\faculty{دانشکده مهندسی کامپیوتر}
% گروه آموزشی خود را وارد کنید
\department{گروه نرم‌افزار}
% گروه آموزشی خود را وارد کنید
\subject{مهندسی کامپیوتر}
% گرایش خود را وارد کنید

\field{نرم‌افزار}
% عنوان پایان‌نامه را وارد کنید
\title{
    عنوان پروژه / پایان‌نامه / رساله
 }
% نام استاد(ان) راهنما را وارد کنید
\firstsupervisor{دکتر ...}
%\secondsupervisor{استاد راهنمای دوم}
% نام استاد(دان) مشاور را وارد کنید. چنانچه استاد مشاور ندارید، دستور پایین را غیرفعال کنید.
\firstadvisor{دکتر ...}
%\secondadvisor{استاد مشاور دوم}
% نام دانشجو را وارد کنید
\name{نام}
% نام خانوادگی دانشجو را وارد کنید
\surname{نام خانوادگی}
% شماره دانشجویی دانشجو را وارد کنید
\studentID{97923000}
% تاریخ پایان‌نامه را وارد کنید
\thesisdate{اسفند 1399}
% به صورت پیش‌فرض برای پایان‌نامه‌های کارشناسی تا دکترا به ترتیب از عبارات «پروژه»، «پایان‌نامه» و »رساله» استفاده می‌شود؛ اگر  نمی‌پسندید هر عنوانی را که مایلید در دستور زیر قرار داده و آنرا از حالت توضیح خارج کنید.
%\projectLabel{پایان‌نامه}

% به صورت پیش‌فرض برای عناوین مقاطع تحصیلی کارشناسی تا دکترا به ترتیب از عبارات «کارشناسی»، «کارشناسی ارشد» و »دکتری» استفاده می‌شود؛ اگر  نمی‌پسندید هر عنوانی را که مایلید در دستور زیر قرار داده و آنرا از حالت توضیح خارج کنید.
\degree{دکتری}

\firstPage

\besmPage
\davaranPage

% در این قسمت اسامی اساتید راهنما، مشاور و داور باید به صورت دستی وارد شوند.
\begin{center}
\begin{tabularx}{0.95\linewidth}{|c|c|c|c|c|c|}
\hline
\makecell[c]{ردیف} & 
\makecell[c]{سمت} & 
\makecell[c]{نام و نام‌خانوادگی} & 
\makecell[c]{مرتبه دانشگاهی } & 
\makecell[c]{دانـشگاه / مؤسـسه } & 
\makecell[c]{~~~~~~~امضــاء~~~~~~~~~} \\ \hline
	1 & 
	\makecell[c]{استاد \\ راهنمای اول} & 
	\makecell[c]{دکتر \\ ...} &
 	\makecell[c]{دانشیار} &
	\makecell[c]{دانشگاه \\ علم و صنعت ایران} & 
    \makecell[c]{~ ~ ~ ~ ~ ~ ~ ~ ~} \\ \hline
	2 & 
	\makecell[c]{استاد \\ راهنمای دوم} &
	\makecell[c]{---} &
	\makecell[c]{---}&
	\makecell[c]{---} &	
     \makecell[c]{~ ~ ~ ~ ~ ~ ~ ~ ~} \\ \hline
	3 & 
	\makecell[c]{استاد \\ مشاور اول } &
	\makecell[c]{دکتر \\ ...} &
   \makecell[c]{استادیار} &
	\makecell[c]{دانشگاه \\ علم و صنعت ایران} &
	 \makecell[c]{~ ~ ~ ~ ~ ~ ~ ~ ~} \\ \hline
	4 & 
	\makecell[c]{استاد \\ مشاور دوم } &
	\makecell[c]{---} &
	\makecell[c]{---} &
	\makecell[c]{---} &	
     \makecell[c]{~ ~ ~ ~ ~ ~ ~ ~ ~} \\ \hline
	5 & 
	\makecell[c]{استاد \\ مدعو داخلی} & 
	\makecell[c]{دکتر \\ ...} & 
    \makecell[c]{استادیار} &
	\makecell[c]{دانشگاه \\ علم و صنعت ایران} & 
	 \makecell[c]{~ ~ ~ ~ ~ ~ ~ ~ ~} \\ \hline
	6 &	 
	\makecell[c]{استاد \\ مدعو داخلی} & 
	\makecell[c]{---} & 
	\makecell[c]{---} &
	\makecell[c]{دانشگاه \\ علم و صنعت ایران} & 
	 \makecell[c]{~ ~ ~ ~ ~ ~ ~ ~ ~} \\ \hline
	7 &	
	\makecell[c]{استاد \\ مدعو خارجی} &
	\makecell[c]{دکتر \\ ...} & 
   \makecell[c]{استادیار} &
	\makecell[c]{دانشگاه \\ ...} & 
     \makecell[c]{~ ~ ~ ~ ~ ~ ~ ~ ~} \\ \hline
	8 &	
	\makecell[c]{استاد \\ مدعو خارجی} &
	\makecell[c]{---} & 
   \makecell[c]{---} &
	\makecell[c]{---} & 
      \makecell[c]{~ ~ ~ ~ ~ ~ ~ ~ ~} \\ \hline
\end{tabularx}
\end{center}

\esalatPage
\mojavezPage

% -- چنانچه مایل به چاپ صفحات «تقدیم»، «نیایش» و «سپاس‌گزاری» در خروجی نیستید، خط‌های زیر را با گذاشتن ٪ در ابتدای آنها غیرفعال کنید.
% -- پایان‌نامه خود را حتماً تقدیم کنید! مرتضی :)

\newpage
\thispagestyle{empty}
%\watermark{مرتضی}
\centerline{\Large \titlefont  تقـــدیم }
\begin{center}
	محل قرار گرفتن متن قـدرانی و تقدیم در نــسخه نهایی پایان‌نامه. 
	
\end{center}


% -- متن سپاس‌گزاری
\begin{acknowledgementpage}
    از زحمات استاد گران‌قدر و فرزانه جناب آقای / سرکار خانم 
\myadvisor~
    که راهنمایی اینجانب را در دوره 
\mydegree~
   عهده‌دار بودند، بی‌نهایت سپاس‌گزار هستم. همچنین ...
	%نسخه پیشرو یک نسخه پیش‌نویس و غیرنهایی از پایان‌نامه، صرفاً جهت اهداف داوری بـــوده و غیـر قابل اســـتـنــاد است. برای تسهیل در امر داوری طرح پیشنهادی و کد پروژه در ادامه درج شده است. همچنین در قسمت مراجع، برای هر مرجع شماره صفحاتی که به آن ارجاع داده‌ شده نیز پیوند شده است. این بخش‌ها در نسخه نهایی حذف می‌گردد. 

% با استفاده از دستور زیر، امضای شما، به طور خودکار، درج می‌شود.
\signature 
\end{acknowledgementpage}

%%%%%%%%%%%%%%%%%%%%%%%%%%%%%%%%%%%%
% کلمات کلیدی پایان‌نامه را وارد کنید
\keywords{
    تعداد واژه‌ها يا عبارات كليدی حداكثر می‌تواند پنج كلمه يا عبارت باشد.
}
%چکیده پایان‌نامه را وارد کنید، برای ایجاد پاراگراف جدید از \\ استفاده کنید. اگر خط خالی دشته باشید، خطا خواهید گرفت.
%% 
%% 
\hypertarget{abstractfa}
{\fa-abstract{
        قالب مطابق ضوابط تعریف شده توسط دانشگاه علم و صنعت برای پروژه‌‌های کارشناسی، پایان‌نامه‌های کارشناسی ارشد و رساله‌های دکتری ایجاد شده است. 
        چکیده حداکثر یک صفحه باشد. در متن چکیده، از ارجاع به منابع و اشاره به جداول و نمودارها اجتناب شده و اصطلاح یا كلمه‌ای زیرنویس نشود. در صورت نیاز به معرفی حوزه تحقیق و مبانی نظری آن، حداکثر در پاراگراف اول چکیده ارائه شود. فقط به ارائه‌ روش تحقیق و نتایج نهایی و محوری آن بسنده شده و از ارائه‌ موضوعات و نتایج كلی اجتناب شود. 
        كلمات یا عباراتی كه در این بخش توضیح داده می‌شوند، باید كاملاً محوری و مرتبط با موضوع تحقیق و دست‌آوردهای نظری و عملی آن باشند. 
        در سه قسمت تنظیم شود. نخست شرح مسئله انجام شود. سپس، انگیزه‌ها و اهداف پژوهش بیان شود. در ادامه، روش یا طرح پیشنهادی، روش تحلیل یا ارزیابی نتایج و نتایج بدست آمده در مقایسه با روش‌های و کارهای موجود بیان شود.
%%        
\\}
}\label{abstractfa2} 

\abstractPage
\newpage\clearpage
 % مشخصات فارسی پایان نامه %
%\input{listacronyms}

\hypertarget{tableofcontent}{\tableofcontents}\label{tableofcontent2} % فهرست مطالب %
\clearpage

\listoffigures 
\clearpage

\listoftables
\clearpage

%\hypertarget{tableofalgorithms}{\listofalgorithms}
%\phantomsection
%\addcontentsline{toc}{chapter}{\listalgorithmname}
%\listofalgorithms
%\clearpage

\printacronyms
\clearpage

\pagestyle{fancy}
% اگر فصل اول  خود را در فایلی به جز chapter1 همراه با این کلاس نوشته‌اید، باید چند خط اول chapter1 را در فایل خود کپی کنید. %

% !TeX root=_main_.tex
% chapter1
% دستور زیر باید در اولین فصل شما باشد. آن را حذف نکنید!
\pagenumbering{arabic}

\chapter{مقدمه}\label{chapter1}
\thispagestyle{empty}


%\begin{flushright}
%\begin{displayquote}	
\epigraph{
«من می‌گویم، امنیت، بالاترین اولویت ماست؛ زیرا برای همه چیزهای هیجان‌انگیزی که شما قادر به انجام دادن آن با کامپیوترها هستید – سازمان‌دهی زندگی‌تان، در ارتباط ماندن با دیگران، خلاق بودن – اگر ما مسائل امنیتی را حل نکنیم، مردم از همه این‌ها عقب خواهند ماند.»
}
{$ \maltese $ {\large بیل گیتس}}
%\end{displayquote}
%\end{flushright}





\section{پیش‌زمینه}
در حـوزه مهندسی نرم‌افزار خودکار (\gls{ASE})،
%\footnote{\lr{automate software engineering}}
\index{مهندسی نرم‌افزار خودکار}
یکی از زمینه‌های مورد مطالعه و پژوهش، خودکارسازی فرایند آزمون نرم‌افزار، به‌عنوان یکی از مراحل مهم توسعه و ساخت یک سیستم نرم‌افزاری است. به‌طور کلی هدف از خودکارسازی، کاهش هزینه و زمان و افزایش دقت در اجرای یک فرایند است.
\gls{FuzzTesting}\index{آزمون!نرم‌افزار}\index{آزمون!فازی}
یکی از فنون آزمون خودکار نرم‌افزار است. آزمون فازی در یافتن
\gls{Fault}ها و\gls{Vulnerability} ها در نرم‌افزارهای دنیای واقعی مانند مرورگرهای وب، ویرایش‌گرهای متن، پخش‌کننده‌های چندرسـانه‌ای و غیره، بسیار مـؤثر واقع شده است
\cite{Takanen:2008:FSS:1404500, Sutton:2007:FBF:1324770}.
در این فن ورودی‌هایی
\gls{Malformed}
\index{بدشکل}
توسط یک برنامه دیگر، یعنی با روش خودکار، تولید شده و به نرم‌افزار تحت آزمون (\gls{SUT})
\index{نرم‌افزار تحت آزمون}
تزریق می‌شود.
\gls{SUT}
در عین حال، به امید یافتن خطا بر اثر پردازش ورودی تزریق شده، \gls{Monitor} می‌شود. ورودی تولید شده که به برنامه داده می‌شود، نقش
\textbf{\gls{TestData}}
\index{داده آزمون} 
را داشته و عامل اصلی نمایان‌سازی \gls{Fault}(های) احتمالی موجود در برنامه با بردن آن به یک حالت \gls{Failure}
\index{خرابی}
 است. به‌همین علّت مهم‌ترین مرحله در فرایند آزمون فازی \index{آزمون!فازی} را می‌توان تولید خودکار داده‌های آزمون دانست، به‌نحوی که بیشترین \gls{Fault}ها، ایراد‌ها و آسیب‌پذیری‌ها شناسایی گردند.





   
\section{شرح مسئله}\label{problem_statement}
راه‌کارهای مطرح در فن \gls{FuzzTesting}
\cite{Miller:1990:ESR:96267.96279,Miller1995,Forrester:2000:ESR:1267102.1267108,Miller:2006:ESR:1145735.1145743}،
برای شناسایی \gls{Fault}ها \index{خطا} و \gls{Vulnerability}‌ها \index{آسیب‌پذیری} نیازمند تولید تعداد زیادی \gls{TestData} هستند. در نرم‌افزارهایی با ساختار ورودی ساده، تولید داده آزمون نیز ساده است. برای مثال می‌توان با روش تصادفی این کار را انجام داد. اما در نرم‌افزارهایی با ساختار ورودی پیچیده، مانند فایل با قالب مشخص تولید داده آزمون متنوع که بتواند مسیرهای اجرایی بیشتری را پوشش دهد، کار آسانی نیست. تعداد و عمق مسیرهای اجرایی در یک برنامه با ساختار ورودی پیچیده به مراتب بیشتر از یک برنامه با ساختار ورودی ساده است. بررسی‌ها نشان می‌دهد بسیاری از داده‌های آزمون تولید شده برای چنین نرم‌افزارهایی، مسیرهای یکسان و سطحی (کم عمق) را می‌پیمایند
\cite{Rawat2017VUzzerAE}
و در مجموع، آزمون‌های فازی معمول پوشش کد ضعیفی دارند
\cite{Kargen:2015:TPA:2786805.2786844}.
درصد بالایی از داده‌های آزمون ساخته شده به‌صورت تصادفی، از لحاظ ساختاری کاملاً نامعتبر هستند و در همان مراحل اولیه بررسی صحت فایل، به‌وسیله \gls{Parser} ورودی برنامه هدف، رد می‌شوند
\cite{10.1007/978-3-319-45744-4_29, Rawat2017VUzzerAE}.
در چنین شرایطی، قادر به نفوذ به عمق برنامه، کشف و آزمایش مسیرهای جدید نخواهیم بود. در واقع این نوع ورودی‌ها به نوعی تکراری و هدر رفته محسوب می‌گردند.


برای حل مسائل بالا، داده آزمون را با استفاده از قالب یا گرامر ورودی تولید می‌کنند، روش‌هایی مثل 
\cite{Godefroid:2012:SWF:2090147.2094081}
. قالب یا گرامر اما به صورت دستی و از روی مستندات تهیه می‌شود که با توجه به پیچیده بودن ساختار آن، عملی زمان‌بر، پرهزینه و مستعد خطا است
\cite{Godefroid:2017:LML:3155562.3155573}.
همچنین مستندات ساختار ورودی همواره دردسترس آزمون‌گر نیست. با این اوصاف روش مذکور تا به امروز، یکی از مؤثرترین روش‌های آزمون و یافتن خطا در برنامه‌هایی مانند مرورگرهای وب بوده، که ساختار ورودی آن فایل‌هایی با قالب‌های متنوع و پیچیده هستند
\cite{Godefroid:2017:LML:3155562.3155573, Kettunen2014}.
به همین جهت، ارایه روشی برای خودکارسازی تولید داده آزمون بر مبنای قالب ورودی ارزشمند و حائز اهمیت است. پیش از ارایه یک روش جدید در ادامه ابتدا مسئله را دقیق‌تر تبیین کرده و راه‌حل‌های قبلی و نارسایی‌های هریک از آنها را مطالعه می‌کنیم.


\subsection{شهود اولیه}\label{intuition}
برای روشن شدن مسئله و شناسایی مشکلات موجود در تولید داده آزمون، مسئله را به زیر مسائل کوچک‌تر شکسته و از زوایای گوناگون تشریح می‌کنیم. ساختار پیچیده ورودی، ساختار پیچیده کد و تمایز داده و \gls{Metadata} سه زیر مسئله‌ای هستند که ما آنها را شناسایی کرده و در این بخش، مطرح می‌کنیم. در ادامه این پایان‌نامه تمرکز خود را بر روی حل این مسائل منعطف خواهیم کرد.


\subsubsection{ساختار پیچیده ورودی}
نخستین مورد حائز اهمیت ساختار ورودی برنامه است. در برنامه‌هایی با ورودی خط فرمان (\gls{CLI}) ساختارها به نسبت ساده هستند. اما برنامه‌هایی با ورودی فایل، ساختار ورودی بسیار پیچیده‌تری دارند. در واقع بسته به کاربرد، آنها یک یا چندین قالب فایل تعریف شده را پشتیبانی می‌کنند. همچنین برای یک قالب فایل شناخته‌شده ممکن است چندین نرم‌افزار وجود داشته‌ باشد. یعنی در حالت کلی یک ارتباط چند‌به‌چند بین قالب فایل ورودی و نرم‌افزار
وجود دارد. هنگامی که یک نرم‌افزار برای آزمون انتخاب می‌شود هریک از قالب‌های فایلی که پشتیبانی ‌می‌کند بخشی از کد نرم‌افزار را اجرا خواهند کرد.

فایل \gls{PDF}
را می‌توان نمونه‌ای از یک ورودی پیچیده برای نرم‌افزارهای \gls{PDF}خوان، مثل اغلب مرورگرهای وب،  تلقی کرد. مجموعه اسناد توصیف کننده \glspl{Specification}ی کامل قالب \gls{PDF} بیش از 1300 صفحه است 
\cite{Godefroid:2017:LML:3155562.3155573}.
 جزئیات ساختار این قالب را در پیوست \ref{appendix:1} بیان کرده‌ایم. در ساختارهای پیچیده هر بایت و در مواردی هر بیت نقش ویژه‌ای ایفا می‌کند که تولید تصادفی آنها تنوعی در پوشش کد برنامه ایجاد نمی‌کند؛ زیرا اغلب در دام کدهای \gls{ExceptionHandling} می‌افتند. بنابراین داشتن یک درک حداقلی از ساختار در هنگام تولید داده جدید بسیار کمک کننده خواهد بود. چگونگی کسب این درک به‌صورت خودکار مسئله‌ای است که بایستی حل شود.



\subsubsection{ساختار پیچیده کد}
پیچیده بودن ساختار ورودی، منجربه پیچیده شدن کداجرایی و در نتیجه ممانعت از پوشش کد بالا در آزمون فازی خواهد شد. برای درک بهتر این مسئله برنامه 
\ref{codesnip1}
به زبان \lr{C} را درنظر می‌گیریم. به‌خاطر سرعت بالای اجرا، بیشتر تجزیه‌گرهای قالب‌های پیچیده به این زبان نوشته می‌شوند. این برنامه یک فایل را از ورودی خوانده و براساس بایت‌های مشخصی در آدرس نسبی آن، مسیرهای معینی را اجرا می‌کند. چندین نکته قابل توجه در قطعه کد مذکور وجود دارد
\cite{Rawat2017VUzzerAE}
:

\begin{enumerate}
	\item{
		\textbf{\glspl{MagicByte}:}
		بایت دوم و بایت اول ابتدا برای اعتبارسنجی ورودی با مقادیر ثابتی مقایسه می‌شوند. اگر نتیجه این مقایسه صحیح نباشد؛ ورودی درجا رد می‌شود. در سطر 13 این مثال ابتدا آدرسی نسبی 1 با مقدار 
		\lr{\textit{0xEF}}
		و سپس آدرس نسبی 0 با مقدار
		\lr{\textit{0xFD}}
		مقایسه می‌گردد. بایت‌های جادویی در قالب‌های فایل بسیاری وجود دارند. از جمله قالب فایل
		\lr{jpeg}
		که در ابزار
		\lr{djpeg}\LTRfootnote{\href{https://linux.die.net/man/1/djpeg}{https://linux.die.net/man/1/djpeg}}
		با همین روش، اعتبارسنجی می‌شود.
	}
	\item{
		\textbf{شرط‌های تودرتو:}
		در اجرای برنامه، هر مسیر اجرایی مهم است. هرچند رسیدن به برخی مسیرها ممکن است دشوارتر باشد یا حتی امکان‌پذیر نباشد
		\cite{ammann2016introduction}
		. در این مثال برای رسیدن به خط 18 کد بایستی همه شرایط موجود در خط 17 برقرار باشد که به‌نوبه خود نیاز هست تا شرط موجود در خط 15 نیز برقرار شد و به همین ترتیب. لذا داده آزمون تولیدی باید تا حد زیادی معتبر باشد تا بتواند به عمق مدنظر دسترسی پیدا کند.
	}
	\item{
		\textbf{\gls{Marker}}\textbf{ها:}
		برای رسیدن به کد خطادار در سطر 19 بایستی شرط سطر 18 ارضـا شود. این مقایسه با یک توالی از نشانه‌ها انجام می‌شود که آدرس نسبی شروع آن لزوماً ثابت نیست؛ البته در این مثال ثابت نشان داده شده است. در قالب‌های فایل‌هایی مانند
		\lr{png}،
		\lr{jpeg}
		و 
		\lr{gif}
		این قبیل نشان‌گرها دیده می‌شود.			
	}
	\item{
		\textbf{آدرس‌های نسبی متغیر:}
		برای رسیدن به مسیر اجرایی سطر 18 یک مقایسه برمبنای آدرس‌های نسبی در سطر 17 انجام می‌شود. آدرس نسبی به‌کار رفته در این مقایسه‌ها، از ورودی خوانده شده یا داخل برنامه محاسبه شده‌اند و بنابراین ممکن است که در هر بار اجرا متفاوت باشند. این امر برخلاف مورد بایت‌های جادویی است که آدرس نسبی ثابتی دارند.
	}
	
\end{enumerate}



%\begin{figure}%[ht]
	%\def\lstlistingname{\rl{تکه کد}}
	%\begin{lstlisting}[language=C]
\begin{LTR}
	\singlespacing
	\begin{lstlisting}[language=C, caption={\rl{یک قطعه‌کد به عنوان نمونه‌ای از نرم‌افزار تحت آزمون در این پایان‌نامه، با ساختار تودرتو که چالش‌های پیچیدگی برنامه تحت آزمون و پوشش کد در آزمون فازی قالب فایل را نشان می‌دهد \cite{Rawat2017VUzzerAE}(با تغییر).}}, label={codesnip1},lineskip=.05cm][ht]
	#include <stdio.h>
	void main(int argc, char *argv[]){
		unsigned char buffer[1024]; //Fixed size buffer
		int fd, size, i, j;
		/* Some initialization here */
		if((fd = open(argv[1], O_RDONLY)) == -1)
			exit(0);
		fstat(fd, &s);
		size = s.st_size;
		if(size > 1024)
			return -1;
		read(fd, buffer, size);
		if(buffer[1] == 0xEF && buffer[0] == 0xFD) //Complex logic expression
			printf("Magic bytes matched!\n");
		else
			EXIT_ERORR("Invalid input file\n");
		if(buffer[i] == '%' && buffer[j] == '$' ){
			if(strcmp(&buffer[15], "MAZE", 4) == 0) //Nested condition
				/* Codes contain bug here */
			else{
				/* *** Render file here (lines of code) *** */
				close(fd); 
				return 0;
				}
		else{
			EXIT_ERROR("Invalid bytes");
			close(fd);
			return 0;
			}
		close(fd);
	}\end{lstlisting}
	\doublespacing
\end{LTR}
%\end{figure}

%\begin{small}\noindent
	%یک قطعه‌کد با ساختار تودرتو که چالش‌های پیچیدگی برنامه و پوشش کد در آزمون فازی قالب فایل را نشان می‌دهد.%
%\end{small}



\subsubsection{ تمایز داده و فراداده}
	برنامه مبتنی بر ورودی فایل، به‌طور معمول دو گام مجزا را برای پردازش یک فایل طی می‌کند: گام اول \gls{Parse} فایل و گام دوم \gls{Render} آن. در مرحله \gls{Parse}، فایل در حافظه بارگذاری، مقادیر فیلدهای آن خوانده شده و تبدیل به داده‌ساختارهای داخل حافظه اصلی (مثل بافر، ساختمان یا رکورد، آرایه و غیره) می‌شود. در این مرحله چنان‌چه \gls{Error} نحوی در ساختار فایل باشد (فایل از مشخصه‌های قالب خود پیروی نکند)، باید توسط تجزیه‌گر تشخیص داده شود وگرنه منجربه اشکال \gls{MemoryCorruption} و خرابی برنامه می‌شود. در مرحله پرداخت، برنامه روی اطلاعات خوانده شده از فایل پردازش لازم را انجام می‌دهد و خروجی تولید می‌کند (مثلاً نمایش یک تصویر روی صفحه نمایش یا اجرای یک ویدئـو و غیره) \cite{Rathaus:2007:OSF:1536880}. خطاهای این مرحله معمولاً جدی‌تر بوده و تشخیص آن نیز مشکل‌تر است، زیرا در عمق بیشتری از کد اجرایی رخ می‌دهند. جایی که داده‌های آزمون کمتری به آن دست پیدا می‌کنند.
	
	
	با توجه به توضیح بالا، می‌توان یک فایل را حاوی دو دسته از مقادیر دانست: اول، مقادیری که مشخص کننده ساختار آن فایل هستند؛ برای مثال نام فیلد‌ها. این مقادیر را \gls{Metadata} یا دادگان (داده برای داده) می‌نامند. دوم، مقادیری که مشخص کننده اطلاعات هر فیلد هستند یا همان داده‌های فایل. مسئله نهفته در اینجا آن است که رویکرد آشکارسازی خطا برای هر کدام از این قسمت‌ها متفاوت خواهد بود؛ چراکه طبیعت خطاهای هر قسمت با یکدیگر متفاوت بوده و همان‌طور که گفته‌شد در مراحل مختلفی هم روی می‌دهند. برای آشکار کردن خطاهای تجزیه‌گر، لازم است تا فایل‌هایی تولید کنیم که بخش فراداده آن بدشکل شده‌اند در حالی که برای آشکار کردن خطاهای بخش پرداخت، بایستی فایل‌هایی تولید کنیم که از لحاظ نحوی معتبر بوده و بخش داده‌ آن بدشکل شده باشند. شرط لازم هر دو نوع بدشکل‌سازی داشتن سازوکاری برای تشخیص داده و فراداده از یکدیگر، در هنگام تولید داده‌های آزمون است.  
	
	همان‌طور که گفتیم، رسیدن به کدهای مرحله پرداخت یک فایل (منظور تزریق داده آزمونی است که منجربه اجرای آن شود) سخت‌تر است. در برنامه \ref{codesnip1}، فرض شده است که به عنوان مثال پرداخت فایل در خط 21 انجام می‌شود؛ یعنی، بعد از گذشتن از تمامی شرایط و بررسی‌های انجام شده توسط تجزیه‌گر و انتقال فیلد‌های داخلی فایل به حافظه اصلی (فیلدهایی مثل فیلد \lr{size} در برنامه مذکور). هر داده آزمونی که یکی از شرایط قبل از خط 21 را نداشته باشد، رد شده و آن اجرا از برنامه به اجرای خط 21 منتهی نمی‌گردد. برای آن که درصد خوبی از داده‌های آزمون تولید شده به اجرای خط 21 منجر شوند، بایستی یک فایل تقریباً معتبر و پیروی کننده از قواعد قالب فایل مورد انتظار برنامه \ref{codesnip1} را به عنوان داده آزمون تولید کرد. 
	
	آنچه از شهود داده شده در این قسمت نتیجه می‌شود آن است که از یک برنامه قابل اطمینان و غیر قابل نفوذ، انتظار می‌رود که تحت هیچ عنوان بر اثر پردازش یک ورودی دچار خطا نشود. تنها زمانی می‌توان این ادعا را داشت که مطمئن شویم برنامه ورودی‌های به اندازه کافی متنوع را پردازش کرده و در هیچکدام از آنها دچار خطا نشده است. ورودی‌ها بایستی قادر به اجرای بخش‌های زیادی از کد برنامه باشند. حالت ایده‌آل اجرای تمام کد یک برنامه پیچیده است.  
	
	 %برای روشن شدن این مسئله، مثال انگیزشی کد 1-2 را مطرح می‌کنیم.
	     
%در این پایان‌نامه روشی مبتنی بر مدل‌های زبانی عصبی برای تولید خودکار داده آزمون با استفاده از یادگیری آماری ساختار فایل ارائه می‌شود.


\subsection{کارهای مرتبط}
تعدادی کار در ارتباط با استخراج خودکار گرامر فایل انجام شده‌اند.
\lr{Bastani} 
و همکاران 
\cite{Bastani:2017:SPI:3140587.3062349}
الگوریتمی برای تولید یک گرامر مستقل از متن روی یک مجموعه از ورودی‌های نمونه داده شده ارایه کرده‌اند، که در نهایت برای تولید داده‌های جدید مورد نیاز آزمون فازی استفاده می‌شود. این الگوریتم یک مجموعه از مراحل تعمیم‌پذیری را با معرفی ساختارهای تکراری و متناوب برای عبارت‌های منظم به‌کار می‌بندد و غیر پایانه‌ها را برای گرامر مستقل از متن در هم ادغام می‌نماید که به‌نوبه خود یک گرامر یکنواخت از زبان ورودی به‌دست می‌دهد؛ اما، این روش برای قالب‌هایی مثل 
\lr{PDF}
 که ساختار مسطح (غیر تو در تو) ولی در عین حال محتوای مختلفی از انواع و جفت‌های کلید-مقدار دارند، مناسب نیست
 \cite{Godefroid:2017:LML:3155562.3155573}.
 
 
 \lr{AUTOGRAM} \cite{Hoschele:2016:MIG:2970276.2970321}
  نیز به‌صورت غیر-احتمالاتی یک گرامر مستقل از متن را یاد می‌گیرد. یک مجموعه ورودی داده‌شده و به‌صورت پویا مشخص می‌شود که چگونه ورودی‌ها در برنامه پردازش می‌شوند. در واقع برنامه تحت آزمون با آلودگی پویا  مشاهده می‌شود که حافظه را با قطعات ورودی که از آنها می‌آیند، برچسب‌گذاری می‌کند. بخش‌هایی از ورودی‌ها که توسط برنامه پردازش می‌شود، نهادهای نحوی در گرامر می‌شوند.  

در پژوهش‌های اخیر تمایل زیادی به استفاده از شبکه‌های عصبی برای تحلیل و تولید برنامه‌ها به‌وجود آمده‎‌است. در سال 2017، 
\lr{Godefroid} \cite{Godefroid:2017:LML:3155562.3155573}
و همکاران روش جدیدی را برای تولید داده آزمون جهت استفاده در آزمون فازی بر مبنای مدل کدگذار-کدگشا\LTRfootnote{\lr{Encoder-Decoder Model}}
 \cite{NIPS2014_5346, DBLP:journals/corr/ChoMGBSB14}
 ارایه کردند. در مقاله آنها، ساختار فایل 
\lr{PDF} 
برای آزمون انتخاب شده است. ایده اصلی یادگیری یک مدل مولد روی مجموعه‌ای از ویژگی‌های اشیای داده‌ای
\lr{PDF} 
  با داشتن مجموعه‌ای از نمونه‌های اولیه است. مدل کدگذار-کدگشا اجازه یادگیری متن با طول دلخواه را برای پیش‌بینی توالی بعدی کاراکترها، می‌دهد. 
  
  مدل استفاده شده توسط 
\lr{Godefroid}
و همکاران، مدل مبنایی وظایفی مانند ترجمه ماشینی یا تبدیل گفتار به نوشتار است که یادگیری ساختار فایل را نمی‌توان در این وظایف گنجاند؛ زیرا، این مدل‌ برای نگاشت دو توالی با دامنه‌های مختلف به کار گرفته می‌شود و این در حالی است که یادگیری ساختار فایل چنین وظیفه‌ای نیست. یعنی می‌توان از مدل‌های ساده‌تری مانند مدل زبانی نیز برای یادگیری ساختار فایل استفاده کرد. روش پیشنهادی آنها، تنها ساختارهای متنی فایل را مورد یادگیری قرار می‌دهد. این در حالی است که ساختار فایل‌های پیچیده هم متنی و هم دودویی هستند. افزون بر این، الگوریتم پیشنهادی آنها برای تولید داده آزمون نیز مشکلاتی دارد. از جمله اینکه ممکن است هیچ‌گاه پایان نیابد. در فصل \ref{related_work}، ضمن تشریح کامل این روش، مشکلات آن را نیز به‌صورت کامل‌تری بیان می‌کنیم. همچنین در فصل \ref{related_work}، دو فازر قالب فایل دیگر تحت عنوان
 \lr{AFL} \cite{Zalewsky2013}
  و
\lr{AFL}افزوده \cite{DBLP:journals/corr/abs-1711-04596}
 که با روش‌هایی غیر از یادگیری گرامر سعی در بهبود پوشش کد 
\gls{SUT}
در فرایند آزمون فازی را دارند، نیز بررسی می‌کنیم و مشکلات آنها را بیان خواهیم کرد.

در هیچ‌کدام از کارهای قبلی، مسئله مطرح شده در ارتباط با تمایز میان داده و فراداده در هنگام آزمون فازی دیده نشده است. به عبارت دیگر، این دیدگاه به آزمون فازی قالب فایل، دیدگاهی نو است و ارزش آزمایش شدن دارد.    امکان استفاده از الگوریتم ارایه شده در روش \cite{Godefroid:2017:LML:3155562.3155573}، برای تمایز میان داده و فراداده وجود دارد اما برای حل مابقی مشکلات، یک روش جدید را در فصل
\ref{ch:4}
، پیشنهاد خواهیم داد. 


\subsection{فرضیه‌ها و اهداف}
هدف اصلی در پایان‌نامه پیش‌ِرو، ارائه روشی کـارا جهت یافتن خطاها و آسیب‌پذیری‌ها در نرم‌افزارهایی مثل \gls{PDF}خوان‌ها بوده که ورودی آنها فایل با ساختار مشخص و معمولاً پیچیده است. در این راستا تولید خودکار فایل‌های ورودی با هدف افزایش \gls{CodeCoverage} \gls{SUT} از اهمیت ویژه‌ای برخوردار است. برای نیل بدین اهداف از فنون \gls{DeepLearning} در یادگیری و درک خودکار ساختار فایل و سپس تولید فایل‌های جدید، استفاده خواهیم کرد. 

چون ایده استفاده از یادگیری ماشینی در آزمون فازی جدید است، این حوزه هنوز برای پژوهشگران ناشناخته  بوده و بنابراین یکی دیگر از اهداف این پایان‌نامه شناسایی، تعریف و تفکیک پارامترهای حاکم در حوزه مذکور است.  به‌نظر می‌رسد که فنون یادگیری ماشینی راه‌گشای حل مسائل شرح داده شده در بخش \ref{problem_statement} باشد. به‌همین جهت فراهم آوردن چارچوبی استاندارد برای شکل‌دهی به کارهای آتی، مفید و مثمر ثمر خواهد بود.
به‌طور خلاصه ما چندین فرضیه در این پایان‌نامه درنظر داریم، که تدوین سازوکارهایی برای رد یا تأیید صحت آنها، اهدافِ ما خواهند بود:

\begin{itemize}
	\item{
	استفاده از فنون یادگیری ژرف بالأخص شبکه‌های عصبی مکرر ژرف، در یادگیری خودکار ساختار فایل، امکان‌پذیر و نتیجه‌بخش است.
}

%	\item{
	%استفاده از تولید مبتنی بر گرامر منجر به رسیدن به مسیرهای اجرایی جدید و بهبود میزان پوشش کد آزمون فازی در برنامه‌هایی که فایل با ساختار پیچیده را به عنوان ورودی می‌پذیرند، می‌گردد
%}

\item{
	خودکارسازی کامل فرایند آزمون فازی مبتنی بر گرامر با ترکیب مدل یادگیری (مدل زبانی عصبی) و روش‌های فاز (بد-شکل‌سازی) ورودی، به خوبی میسر می‌شود.
}

\item{
روش‌های ترکیبی تولید داده آزمون، یعنی روش تولید مبتنی بر گرامر به همراه روش تولید مبتنی بر جابه‌جایی، منجر به افزایش پوشش کد
\gls{SUT}
می‌گردند.
}

\item{
	امکان کشف خطاها و آسیب‌پذیری‌های احتمالی موجود در \gls{SUT} از طریق آزمون فازی با داده‌های آزمون تولید شده از طریق مدل‌های یادگیری ژرف، وجود دارد.
}

\end{itemize}

   

\section{روش پیشنهادی و نوآوری‌ها}
در این پایان‌نامه یک روش برای یادگیری خودکار ساختار فایل و سپس تولید داده‌های آزمون بر اساس آن ارایه می‌شود. برای یادگیری از مدل زبانی (\gls{LM})
 که یک مفهوم ابتدایی در پردازش زبان طبیعی (\gls{NLP})،
است استفاده می‌کنیم. مدل زبانی را با استفاده از کلاس خاصی از 
\glspl{DeepNeuralNetwork}
موسوم به شبکه‌ عصبی مکرر (\gls{RNN})، ایجاد می‌کنیم که در نتیجه به آن مدل زبانی عصبی (\gls{NLM}) هم گفته می‌شود. روش ارایه  شده در اینجا، همچنین، بخش‌های غیرمتنی را نیز در آزمون فازی لحاظ می‌کند، به هیچ قالب فایل خاصی وابستگی نداشته و به سبب پیاده‌سازی با زبان پایتون قابلیت اجرا برروی هر ماشینی را دارد. 

روش پیشنهادی در فصل 
\ref{ch:4}،
در سه بخش کلی ارائه شده است. بخش اول به یادگیری ساختار فایل می‌پردازد
(بخش \ref{sec:model})
، بخش دوم روشی برای تولید و بدشکل‌سازی همزمان داده‌های آزمون ارایه می‌دهد 
(بخش \ref{sec:neural_fuzzing_algorithms})
و در نهایت بخش سوم یک فازر کاملاً پیمانه‌ای را معرفی می‌کند که از آن برای آزمون فازی قالب فایل استفاده خواهد شد
(بخش \ref{sec:implementation})
. در حالی که تمرکز اصلی بر روی نحوه تولید داده‌های آزمون است، اما برای انجام آزمون فازی به ابزارهای دیگری مانند تزریق کننده داده آزمون و نیز پایش 
\gls{SUT}  
جهت ثبت خطاهای رخ‌ داده شده نیاز است. نوآوری‌های اصلی روش پیشنهادی به طور خلاصه عبارتند از:
\begin{enumerate}
	\item{
	یادگیری گرامر یا ساختار یک قالب فایل با استفاده از مدل‌های زبانی عصبی،	
}

\item{
	تولید داده‌های آزمون متنی و دودویی همگام با بدشکل‌سازی آنها با استفاده از یک روش ترکیبی،
}

\item{
ایجاد یک فازر قالب فایل و یک مجموعه داده آزمون برای آزمون فازی نرم‌افزارهای 
\gls{PDF}خوان،
}

\item{
	و بررسی و شناسایی پارامترهای مؤثر در یادگیری ساختار فایل با استفاده از فنون یادگیری ژرف.
}

\end{enumerate}

توضیح مبسوط‌تری از نوآوری‌های و دستاوردهای این پایان‌نامه در فصل 
\ref{ch:6}، 
ارایه شده است. در آن فصل همچنین مزایا و معایب فنون یادگیری ژرف در یادگیری ساختار فایل و نیز مزایا و معایب روش پیشنهادی بررسی و بیان شده‌اند. 


\section{اهمیت موضوع}
مانند هر محصول دیگری، نرم‌افزار نیازمند \gls{Test} و راستی‌آزمایی است.  ماهیت غیرقابل لمس و پیچیدگی ذاتی نرم‌افزار سبب می‌شود تا فرایند آزمون آن نیز متفاوت، پیچیده و پرهزینه باشد. اما این دشواری‌ها از اهمیت موضوع آزمون نمی‌کاهد. \gls{Fault}های نرم‌افزاری در مواردی سبب خسارت‌های مالی و جانی جبران ناپذیری شده‌اند. راکت آریـان 5\LTRfootnote{\href{https://en.wikipedia.org/wiki/Ariane\textunderscore5}{https://en.wikipedia.org/wiki/Ariane\textunderscore5}}
 اروپا در سال 1996، تنها 37 ثانیه پس از پرتاب منفجر شد. علت آن وقوع خطا در تبدیل نوع یک عدد ممیز شناور به عدد صحیح بود \cite{ammann2016introduction}. وجود خطا در ماشین پرتودرمانی \lr{Therac-25}\LTRfootnote{\href{https://en.wikipedia.org/wiki/Therac-25}{https://en.wikipedia.org/wiki/Therac-25}}،
سبب کشته شدن دست‌کم سه انسان بر اثر تششع بیش‌از حد پرتو، در سال‌های 1985 تا 1987 شد. مثال‌های دیگری از این قبیل در 
\cite{ammann2016introduction,Dubrova:2013:FD:2462571}
آمده است.

در مواردی وجود خطا منجربه \gls{Vulnerability} می‌شود که امکان سوء استفاده و دسترسی‌های غیرمجاز را به \glspl{Attacker} می‌دهد. \gls{Ransomware} 
\lr{WannaCrypt}\LTRfootnote{\href{https://docs.microsoft.com/en-us/windows/security/threat-protection/wannacrypt-ransomware-worm-targets-out-of-date-systems-wdsi}{https://docs.microsoft.com/en-us/windows/security/threat-protection/wannacrypt-ransomware-worm-targets-out-of-date-systems-wdsi}}
که در نیمه اول سال 2017، بیش از ۲۳۰ هزار رایانه را در ١٥٠ کشور جهان آلوده ساخت، از یک آسیب‌پذیری در هسته نسخه‌های قدیمی، سیستم عامل ویندوز شرکت مایکروسافت بهره‌برداری کرده ‌بود. این باج‌افزار اطلاعات کاربر را رمزنگاری و برای رمزگشایی آن درخواست پرداخت هزینه می‌کرد. شرکت سیمنتک\LTRfootnote{\href{https://www.symantec.com/}{https://www.symantec.com/}}
در گزارش \gls{ISRT} خود در سال 2018\LTRfootnote{\href{https://www.symantec.com/security-center/threat-report}{https://www.symantec.com/security-center/threat-report}} \cite{Symantec2018}، افزایش 600 درصـدی حملات در \gls{InternetOfThings} (اینترنت اشیاء) و افزایش تهدیدات در دیگر حوزه‌های سایبری از جمله تلفن همراه، را اعلام کرده است. در هر حال، کشف خطا و آسیب‌پذیری احتمالی ناشی از آن، در نرم‌‌افزارهایی که به طور گسترده توسط همگان مورد استفاده قرار می‌گیرند، مثل سیستم‌عامل‌ها، مرورگرهای وب، \gls{PDF}خوان‌ها و غیره، بسیار حائز اهمیت است؛ زیرا، در صورت برطرف نشدن آن خطر وقوع حملاتی مشابه حملات بالا دور از انتظار نخواهد بود.


هنگامی که نرم‌افزارها بزرگ می‌شوند، آزمون دستی پاسخ‌گو نیست و خودکارسازی آزمون اهمیت می‌یابد. آزمون فازی همان‌طور که در ابتدای فصل بیان شد، به عنوان یک فن مؤثر آزمون نرم‌افزار در شناسایی خطاهای حافظه و آسیب‌پذیری‌ها شناخته شده است. برای مثال چرخه حیات امن نرم‌افزار (\gls{SDL}) شرکت مایکروسافت
\LTRfootnote{\href{https://www.microsoft.com/en-us/sdl}{https://www.microsoft.com/en-us/sdl}}، در مرحله \gls{Verification}، استفاده از آزمون فازی را به عنوان یک روش استاندارد، اجباری می‌کند \cite{Corporation2010}. آزمون فازی \gls{WhiteBox} (رجوع کنید به بخش \ref{box_view})، حدود یک سوم کل آسیب‌پذیری‌های شناخته شده در سیستم عامل ویندوز 7 این شرکت را کشف کرده است \cite{Godefroid:2012:SWF:2090147.2094081}. شرکت گوگل در سال 2012، اطلاعاتی راجع‌به ابزار \lr{ClusterFuzz} خود منتشر کرد که از آن برای آزمون فازی پروژه‌های \lr{ Chromium}\LTRfootnote{\href{https://www.chromium.org/}{https://www.chromium.org/}} (شامل  \lr{Chromium OS} و مرورگر وب \lr{Chromium}) استفاده می‌کند \cite{Kettunen2014}. این شرکت همچنین به افرادی که موفق به کشف آسیب‌پذیری در پروژه‌های‌ نام‌برده شوند، جوایزی اهدا می‌کند.


تولید داده آزمون را بایستی مهم‌ترین مرحله در آزمون فازی دانست؛ چراکه داده‌هایی که نرم‌افزار با آنها آزمون می‌شود عامل اصلی اجرا شدن کد‌های قسمت‌های مختلف \gls{SUT} است و در صورتی که خطایی در آنها وجود داشته باشد، تنها از این طریق است که خود را نشان می‌دهد. البته باید بدین مسئله توجه کرد که اجرای کد خطادار شرط لازم برای آشکارسازی خطا است ولی کافی نیست و روش تولید داده‌های آزمون، می‌بایست شرایط خاص بدشکل بودن را نیز محیا کند. تولید داده مبتنی بر گرامر، مؤثرترین روش آزمون برنامه‌هایی با ساختار ورودی پیچیده است \cite{Chen2018}. موارد بیان شده در این بخش، به‌خوبی اهمیت موضوع تولید خودکار داده آزمون در آزمون فازی و لزوم ارایه روش‌های جدید را توجیه کرده و انگیزه کافی را برای پژوهش در این زمینه ایجاد می‌کنند.   




\section{ساختار پایان‌نامه}

این پایان‌نامه در شش فصل و دو پیوست تنظیم شده است و ساختار ادامه آن به  قرار زیر است. در فصل
%\hyperref[chapter2]{فصل دوم}
\ref{chapter2}
ادبیات موضوع شامل آزمون نرم‌افزار، آزمون ‌فازی و یادگیری ژرف را مطرح می‌کنیم. در این فصل ابتدا معیارهای سنجش کیفیت آزمون و چگونگی محاسبه‌ آنها را توضیح داده، سپس به معرفی آزمون فازی، فرایند کلی  و روش‌های تولید داده آزمون در آن می‌پردازیم. در بخش پایانی مباحث یادگیری ژرف را با تمرکز بر مفاهیم مرتبط با یادگیری ساختار فایل، عنوان خواهیم کرد.  


در فصل \ref{related_work} به پیشینه پژوهش و بیان کارهای مرتبط در تولید خودکار داده آزمون و نقـد و بررسی آنها می‌پردازیم. به‌طور خلاصه برخی راه حل‌های دیگران برای مسائل مطرح شده در بخش \ref{problem_statement} را معرفی و  سپس مشکلات آنها را بیان می‌کنیم. روش پیشنهادی در راســتای حل این مسائل و ارزیابی ما در مقایسه با نتایج ارائه شده قبلی در این فصل خواهد بود.

در فصل \ref{ch:4} روش پیشنهادی خود را برای تولید داده آزمون مطرح می‌کنیم. روش پیشنهادی در این فصل، همان‌طور که بدان اشاره شد، یک روش تولید مبتنی بر مدل‌های زبانی است که ما جابه‌جایی‌های تصادفی را نیز به آن اضافه کرده‌ و روشی ترکیبی خلق نموده‌ایم. در  همین فصل، ما دو الگوریتم جدید را برای فاز داده‌های آزمون معرفی می‌کنیم.

در فصل \ref{ch:5} معیارهای ارزیابی روش پیشنهادی، چیدمان آزمایش‌ها و نتایج حاصل از اجرای آنها را ذکر خواهیم کرد. مورد مطالعاتی ما در آزمایش‌های این فصل نرم‌افزار 
\lr{MuPDF}\LTRfootnote{\href{https://mupdf.com/}{https://mupdf.com/}} \cite{MuPDF2018}
 و قالب فایل \lr{PDF} است که در ابتدای فصل آنها را مختصر معرفی خواهیم کرد. 

در نهایت فصل \ref{ch:6} را به بیان نتیجه‌گیری، یافته‌ها و نوآوری‌های پایان‌نامه، محدودیت‌های روش پیشنهادی و کارهای قابل انجام در آینده اختصاص داده‌ایم. همچنین در پیوست \ref{appendix:1} ساختار فایل \lr{PDF} و در پیوست \ref{appendix:2} جزئیات پیاده‌سازی محصول نهایی پایان‌نامه را درج کرده‌ایم.



 % فصل اول: مقدمه %
% !TeX root=maintext.tex
% !TeX TS-program = XeLaTeX
% !TEX spellcheck = fa
% chapter2

\chapter{ادبیات موضوع}\label{chapter:2}
\thispagestyle{empty}
\epigraph{
«هر نادانی می‌تواند کدی بنویسد که کامپیوتر بفهمد. برنامه‌نویسان خوب کدی می‌نویسند که انسان‌ها می‌توانند آن را بفهمند!»
}
{$ \maltese $ {\large مارتین فاولر}}
\noindent
در این فصل مفاهیم اولیه استفاده شده در پایان‌نامه یا رساله را به اختصار توضیح دهید.
 هدف از این فصل كه دارای عنوان «تعاریف و مفاهیم مبنایی» یا «مبانی نظری» یا عنوان مشابهی است، بیان اصول، تعاریف و مبانی نظری مورد نیاز به عنوان دانش پیش‌زمینه  برای شروع مطالعه و فهم فصول بعدی پایان‌نامه است.
 این فصل می‌تواند شامل کارهای مرتبط نیز باشد. البته کارهای مرتبط می‌تواند در یک فصل مجزا بررسی شود. هدف کارهای كه دارای عنوان «مروری بر کارهای مرتبط» یا «مروری بر پیشینه تحقیق» است، بررسی و طبقه‌بندی یافته‌های تحقیقات دیگر محققان در سطح دنیا، مقایسه راه‌حلهای موجود و تعیین و شناسایی خلأهای تحقیقاتی و مسایل باز  است. در آخر این فصل در صورت لزوم مقایسه‌ای بین روشها یا راه‌حلهای موجود ارائه گردد تا جایگاه روش، رهیافت، فن یا طرح پیشنهادی شما مشخص گردد.
 \par
 این فصل برخی از ساختارهای پرتکرار نوشتار مانند بخش‌بندی، مرجع، شکل، جدول، الگوریتم و فرمول‌های ریاضی را در قالب نمونه‌هایی آموزش می‌دهد.
متن زیر نمونه‌ای از یک بخش تحت عنوان «درج مرجع» دارای یک پاراگراف و چند مرجع را نشان می‌دهد.  همچنین در ادامه آن یک «زیربخش» آماده است. 
 
\section{درج مرجع}\label{software_testing}
مجموعه فنون کشف و آشکارسازی 
\glspl{failure}ی
 نرم‌افزار در مراحل مختلف توسعه آن را
\textit{آزمون نرم‌افزار}
گویند. منظور از 
\gls{failure}
 بروز رفتار(های) ناخواسته و خلاف
 \glspl{specification}
 در یک نرم‌افزار یا قسمتی از آن است، که نتیجه آن برای کاربر قابل مشاهده است.
 \gls{failure}
 حاصل یک
 \gls{fault}
  (نقص) ایستا در نرم‌افزار است که در هنگام وقوع از چشم کاربر پنهان است. حالت داخلی نادرست برنامه را که ناشی از یک خطا است،  
\gls{error} 
می‌گویند
\cite{ammann2016introduction}.
 مفاهیم خطا، اشکال و خرابی از حوزه
 \gls{dependability}
  وارد آزمون نرم‌افزار شده‌اند
\cite{Dubrova:2013:FD:2462571}.
\par
درنهایت، خطاها ممکن است ناشی از درک نادرست نیازمندی‌ها به دلیل وجود ابهام در آنها باشند
\cite{Ferrari2017}.

\subsection{زیربخش}
زیربخش‌ها با سه عدد از سمت راست به چپ عدد فصل، عدد بخش و عدد زیر بخش به‌صورت خودکار شماره‌گذاری و در فهرست مطالب درج می‌شوند.

\section{درج شکل}
   شکل 
     \ref{fig:ch2_box_veiw_test_triangle_crop}
     انواع روش‌های  آزمون نرم‌افزار را نشان می‌دهد.
  
  
\begin{figure}[!h]
    \centering
    \includegraphics[width=0.75\linewidth, clip=true,  trim= 0 0 0 0]{chapter2/ch2_box_veiw_test_triangle_crop.pdf}
    \caption[ انواع روش‌های آزمون نرم‌افزار]
    {
        انواع روش‌های آزمون نرم‌افزار
    }
    \label{fig:ch2_box_veiw_test_triangle_crop}
\end{figure}

\section{درج جدول}
در اینجا نمونه‌ای از یک جدول به همراه ارجاع به آن در متن آماده است.
جدول
\ref{tabel:metrics}
متریک‌های مورد استفاده در رساله پیشنهادی را به تفکیک موضوع و سطح، نشان می‌دهد.  

\begin{table}[!h]
    \centering
    \caption[متریک‌های استفاده شده در رساله پیشنهادی]
    {متریک‌های نرم‌افزار
}
    \label{tabel:metrics}
    \resizebox{0.85\linewidth}{!}{%
        \begin{latin}

        \begin{tabular}{lllllll}
            \hline
            Subject     & Metric name                                      & Abbrivation & Method & Class & File & Package \\ \hline
            Size/Count  & Line of code                                     & LOC         & *      & *     & *    & *       \\
            & Number of statements                             & NOSM        & *      & *     & *    & *       \\
            & Number of static   methods                       & NOSM        &        & *     & *    & *       \\
            & Number of static   attributes                    & NOSA        &        & *     & *    & *       \\
            & Number of instance   methods                     & NOIM        &        & *     & *    & *       \\
            & Number of instance   attributes                  & NOIA        &        & *     & *    & *       \\
            & Number of method                                 & NOMT        &        & *     & *    & *       \\
            & Number of not accessor   or mutator methods      & NOMTNAMM    &        & *     & *    & *       \\
            & Number of constructores                          & NOCON       &        & *     &      &         \\
            & Number of parameters                             & NOP         & *      & *     & *    & *       \\
            & Number of classes                                & NOCS        &        &       & *    & *       \\
            & Number of files                                  & NOFL        &        &       &      & *       \\
            Complexity  & Cyclomatic complexity                            & CC          & *      & *     & *    & *       \\
            & Number of unique paths though a body of   code   & PATH        & *      & *     & *    & *       \\
            & Nesting level                                    & NESTING     & *      & *     & *    & *       \\
            & Number of overlapping jumps                      & KNOTS       & *      & *     & *    & *       \\
            Dependency  & Lack of cohesion in methods                      & LOCM        &        & *     &      &         \\
            & Coupling between   objects                       & CBO         &        & *     &      &         \\
            & Response for a class                             & RFC         &        & *     &      &         \\
            & Number of incoming invocations                   & FANIN       & *      & *     & *    & *       \\
            & Number of outgoing invocations                   & FANOUT      & *      & *     & *    & *       \\
            & Called foreign not   accessor or mutator methods & CFNAMM      &        & *     &      &         \\
            & Access to foreign data                           & ATFD        &        & *     &      &         \\
            & Data abstraction   coupling                      & DAC         &        & *     &      &         \\
            Visibility  & Number of default   methods                      & NODM        &        & *     & *    & *       \\
            & Number of private   methods                      & NOPM        &        & *     & *    & *       \\
            & Number of protected   methods                    & NOPRM       &        & *     & *    & *       \\
            & Number of public   methods                       & NOPLM       &        & *     & *    & *       \\
            & Number of accessor   methods                     & NOAM        &        & *     & *    & *       \\
            Inheritance & Depth of inheritance   tree                      & DIT         &        & *     &      &         \\
            & Number of children                               & NOC         &        & *     &      &         \\
            & Number of parents                                & NOP         &        & *     &      &         \\
            & Number of inherited   methods                    & NIM         &        & *     &      &         \\
            & Number of methods   overridden                   & NMO         &        & *     &      &         \\
            & Number of implemented interfaces                 & NOII        &        & *     &      &         \\
            Total       & 35                                               &             & 9      & 33    & 21   & 22      \\ \hline
        \end{tabular}%
        \end{latin}   
 }
\end{table}
 
 
 \section{درج الگوریتم}
 یکی از نقاط قوت 
 \LaTeX
 امکان حروف‌چینی بسیار خوانای الگوریتم‌ها و شبه‌کدها است که معمولاً در نوشتارهای مهندسی کامپیوتر وجود دارند. در اینجا یک نمونه الگوریتم (الگوریتم 
  \ref{alg:data-neural-fuzz}
 ) برای نمونه قرار داده شده است:
 
 %%% My algorithms %%%
 %%
 %% 1 - DataNeuralFuzz
 %%
 \begin{algorithm}%[ht]
     \onehalfspacing
     \caption{\lr{DataNeuralFuzz}} \label{alg:data-neural-fuzz}
     \begin{latin}
         %\begin{algorithmic}[1]
         \DontPrintSemicolon
         \setcounter{AlgoLine}{0}
         \LinesNumbered
         
         \SetKwFunction{Random}{Random}
         \SetKwFunction{RandInt}{RandInt}
         \SetKwFunction{Predict}{Predict}
         \SetKwFunction{EndsWith}{EndsWith}
         \SetKwFunction{Sample}{Sample}
         \SetKwFunction{Chars}{Chars}
         \SetKwFunction{Len}{Len}
         \SetKwFunction{AddBinaryPart}{AddBinaryPart}
         \SetKwFunction{MutateBinaryPart}{MutateBinaryPart}
         \SetKwInput{KwData}{Input}
         \SetKwInput{KwResult}{Output}
         
         \KwData{Learnt model $M$, Sequence prefix $P$, Diversity $D$, Fuzzing rate $FR$, End token $ET$, Binary token $BT$}
         \KwResult{Test data $TD$}
         
         \BlankLine
         
         $TD$  $\gets$ $P$\;
         
         $MaxLen$  $\gets$ \RandInt($a$, $b$)\;
         
         \While{$not$ \EndsWith($TD$, $ET$)}
         {
             $predicts$  $\gets$ \Predict($M$($P$))\;
             
             $c$, $p(c)$  $\gets$ \Sample($predicts$, $D$) \tcc*{Sample c from the learnt model}\;
             
             $p\_fuzz$  $\gets$ \Random($0,1$) \tcc*{Decide whether to fuzz}\;
             
             \If{ $p\_fuzz<FR \wedge p(c)<\alpha \wedge c\not\in$ \Chars($BT$) $\wedge c\not\in$ \Chars($ET$)}
             {
                 $c$  $\gets$ $argmin_{c'}\{ p(c') \in predicts \}$ \tcc*{Fuzz c by c' where c' is the lowest likelihood}\;
             } 
             
             $TD$  $\gets$ $TD$ + $c$\;
             
             $P$  $\gets$ $P[1:]$ + $c$ \tcc*{Propagate fuzz to prefix and next generated data}\;
             
             \If{ \Len($TD$) > $MaxLen$ }
             {
                 $TD$  $\gets$ $TD$ + $ET$ \;
                 
                 \textbf{Break}\;
             }
             
         }
         
         \If {$BT \in TD$}
         {
             $TD$ $\gets$ \AddBinaryPart($TD$)\;
             
             $TD$ $\gets$ \MutateBinaryPart($TD$)\;
         }
         
         \textbf{Return} $TD$\;
         
         %\end{algorithmic}
     \end{latin}
 \end{algorithm}
 
 
 \section{درج فرمول‌ها و روابط ریاضی}
 برای نمونه به رابطه محاسبه سرگشتگی عنایت فرمایید. همان‌طور که مشاهده می‌شود فرمول‌ها و روابط ریاضی به‌صورت خودکار شماره‌گذاری می‌شوند.
 
  \begin{equation}\label{ppl}
     \begin{split}
         PP_{LM}(x) & = \sqrt[n]{\prod_{i=1}^n(\frac{1}{p(x^{(i)}|<x^{(1)}, ..., x^{(i-1)}>)}} \\
         & = 2^{-\frac{1}{n}\sum_{i=1}^n\log_{2}{p(x^{(i)}|<x^{(1)}, ..., x^{(i-1)>})}}
     \end{split}
 \end{equation}
 
 
 
 \section{خلاصه}
 ‌در بخش خلاصه یا نتیجه‌گیری انتهایی هر فصل، خلاصه و جمع‌بندی مطالب آن فصل ارائه می‌گردد. 
 
  % فصل دوم: ادبیات موضوع %
% !TeX root=_main_.tex
% chapter3

\chapter{کارهای مرتبط}\label{related_work}
\thispagestyle{empty}
\epigraph{
	«یک شب تاریک و طـوفانی بود. پاییز 1994، در آپارتمان خود در مادیـسون نشسته بودم. آن شب من از طریق خط تلفن به سیستم‌های یونیکس دانشگاه متصل شدم. با بارش سنگین باران اختلال زیادی روی خط اتصالی بود که در اجرای فرمان‌های ارسالی من دخالت می‌کرد. رقابتی بین سرعت نوشتن یک فرمان قبل از آنکه اختلال آن را خراب کند وجود داشت. چیزی که مرا شگفت زده می‌کرد این حقیقت بود که اختلال سبب ایجاد خرابی و سقوط برنامه‌ها می‌شد و شگفت‌آورتر برنامه‌هایی بود که سقوط می‌کرد: ابزارهای رایج یونـیکس!»
}
{$ \maltese $ {\large بارتون میلر، مبدع آزمون فازی}}

\noindent
کارهای بسیاری برای بهبود آزمون فازی اولیه که داده‌های آزمون را به‌صورت تصادفی تولید می‌کرد \cite{Miller:1990:ESR:96267.96279}، انجام شده است. فازرهای \gls{GenerationBased} در برنامه‌های با ساختار ورودی پیچیده پوشش کد بیشتری نسبت به فازرهای \gls{MutationBased} فراهم می‌کنند\cite{Miller2007}، اما کاملاً خودکار نیستند \cite{Godefroid:2017:LML:3155562.3155573}. در مقابل فازرهای \gls{MutationBased} سعی کرده‌اند تا با استفاده از الگوریتم ‌های تکاملی مثل ژنتیک داده‌ آزمون‌های بهتری را برای جابه‌جایی انتخاب کنند. \lr{AFL} \cite{Zalewsky2013} نمونه‌ای موفق از فازرهای \gls{MutationBased} است. در سال 2017، فنون یادگیری ماشینی به هر دودسته از فازرهای بالا اعمال شده‌اند. در فازرهای \gls{GenerationBased}، برای یادگیری خودکار گرامر فایل \cite{Godefroid:2017:LML:3155562.3155573} و در فازرهای \gls{MutationBased} برای پیش‌بینی بهترین مکان جابه‌جایی \cite{DBLP:journals/corr/abs-1711-04596}. هریک از این کارها نواقص و محدودیت‌هایی دارند. در این فصل به معرفی، نقد و بررسی این کارها می‌پردازیم.



\section{فازر AFL}
\lr{AFL} \cite{Zalewsky2013}
 یک فازر قالب فایلِ جعبه خاکستری، با تولید داده آزمون مبتنی بر جابه‌جایی، دارای بازخورد، متن‌باز و رایگان است که توسط 
 \lr{Michal Zalewski}\LTRfootnote{\href{http://lcamtuf.coredump.cx/}{http://lcamtuf.coredump.cx/}}
  توسعه داده شده است. این فازر روی سیستم عامل‌های خانواده یونیکس قابل اجرا است. راه‌اندازی آن ساده بوده و واسط کاربری خوبی برای دنبال کردن جزئیات آماری فرایند آزمون فازی و خطاهای کشف شده دارد. در ‏شکل \ref{ch3_afl_gui.png} یک نمونه از اجرای این فازر را در عمل نشان داده‌ایم.
  
  
 به‌طور پیش‌فرض \lr{AFL} برای سنجش پوشش کد، به کد منبع \gls{SUT}  نیاز دارد. برنامه‌های نوشته شده به زبان‌های 
 \lr{C}،
 \lr{C++} و
 \lr{Objective-C}
 در آزمون فازی جعبه سفید با \lr{AFL} قابل استفاده هستند. همچنین نسخه‌هایی از \lr{AFL} برای برنامه‌های نوشته شده به زبان‌های
 	 \lr{Go} و \lr{Python}
 	  توسط دیگران انتشار یافته است. \lr{AFL} در آزمون فازی جعبه سیاه، برای ابزارگذاری و اخذ اطلاعات زمان اجرا، از ابزار
 	   \lr{QEMU} \cite{QEMU2018}
 	   استفاده می‌کند. این فازر به‌حدی موفق بوده‌ که پژوهش‌های زیادی برای بهبود جنبه‌های مختلف آن انجام شده است. اما چنان‌چه خواهیم دید روی قالب فایل‌هایی با ساختار پیچیده نمی‌تواند به پوشش خوبی دست پیدا کند \cite{DBLP:journals/corr/abs-1711-04596}.
 	   
 	   
 \begin{figure}[H]%[tbh!]%[ht]%[t!]
 	   	\centering
 	   	\includegraphics[width=\textwidth, clip=true,  trim= 0 0 0 0]{chapter3/ch3_afl_gui.png}
 	   	\caption[محیط زمان اجرای AFL]
 	   	{
 	اجرای \lr{AFL} بر روی ابزار \lr{mutool}  از نرم‌افزار 
 	\lr{MuPDF} \cite{MuPDF2018}. این تصویر جزئیات اجرا بعد از گذشت 55 روز از آغاز فرایند آزمون فازی را نشان می‌دهد. آزمون تا زمانی که کاربر $ctrl+z$ را فشار ندهد، اجرا می‌شود.
 	   	}
 	   	\label{ch3_afl_gui.png}
 	   	%\ref{ch3_afl_gui.png}
\end{figure}
 	      
 	   
 	   

\subsection{معماری AFL}
‏شکل \ref{ch3_afl_fuzz.png} 
\glspl{Component}ی 
اصلی فازر \lr{AFL}  و ترتیب استفاده از آنها در هنگام آزمون فازی را نشان می‌دهد. در این معماری چهار \gls{Component} مشاهده می‌شود: 

 \begin{figure}%[tbh!]%[ht]%[t!]
	\centering
	\includegraphics[width=0.95\textwidth, clip=true,  trim= 0 0 0 0]{chapter3/ch3_afl_fuzz.pdf}
	\caption[مؤلفه‌های فازر \lr{AFL} و ارتباط آنها]
{
	مؤلفه‌های فازر \lr{AFL} و ارتباط آنها با یکدیگر
	\cite{Zalewsky2013}.
}
\label{ch3_afl_fuzz.png}
%\ref{ch3_afl_fuzz.png}
\end{figure}



\begin{itemize}
	\item {
	\textbf{\lr{afl-gcc}.}
	یک جایگزین برای \lr{gcc}  یا \lr{clang}  استاندارد است که برای کامپایل کد منبع \lr{SUT}  استفاده می‌شود و خروجی آن به \lr{afl-as}  ارسال می‌شود. در رویکرد جعبه سفید استفاده مرحله اول کامپایل \lr{SUT} با
	 \lr{afl-gcc}
	  است.

}
\item {
	\textbf{\lr{afl-as}.}
	کد کامپایل شده با \lr{afl-gcc} را با تزریق کدهای اسمبلی ابزارگذاری می‌کند. ابزارگذاری به نحوی است که پوشش انشعاب  یا پرش را ضبط می‌کند. خروجی \lr{afl-as}  فایل دودویی قابل اجرای \lr{SUT}  است که سپس توسط
	 \lr{afl-fuzz}
	  استفاده می‌شود.
}
\item {
\textbf{\lr{afl-fuzz}.}
همانطور که انتظار می‌رود، هسته اصلی فازر است که عملیات فاز ورودی را انجام می‌دهد. این ابزار فایل دودویی و داده آزمون ورودی را دریافت و با توجه به اطلاعات دریافتی از \lr{afl-analyze} فرایند آزمون را ادامه می‌دهد. این مؤلفه برای آزمون فازی جعبه سیاه، مستقیماً به‌کار می‌رود. \lr{afl-fuzz} همچنین مسئول چاپ واسط کاربری روی ترمینال است.
}

\item {
\textbf{\lr{afl-analyze}.}
 اثر ورودی اجرا شده بر پوشش کد را بررسی می‌کند و اطلاعات مربوط به پوشش کد را به عنوان بازخوردی به \lr{afl-fuzz} می‌دهد تا برای جابه‌جایی‌های بهتر در ادامه استفاده کند.
}

\end{itemize}
 
حلقه بازخوردی که در ‏شکل \ref{ch3_afl_fuzz.png} وجود دارد نشان دهنده تکاملی بودن فرایند آزمون فازی در \lr{AFL} است. \lr{AFL} در هسته خود از الگوریتم ژنتیک استفاده می‌کند. یک فایل ورودی جهش یافته، مفید و مورد توجه است اگر قسمت‌های جدیدی از کد دودویی را اجرا کرده باشد یا تعداد اجرای کدهای قبلی مشاهده شده را افزایش داده باشد. این ویژگی با عنوان \lr{Input Gain} شناخته می‌شود \cite{DBLP:journals/corr/abs-1711-04596}. \lr{AFL}، سپس این داده آزمون را به انتهای صف داده‌های آزمون، اضافه می‌کند. به این ترتیب حاصل فرایند آزمون فازی در \lr{AFL}، علاوه بر اجرای \gls{SUT} و اندازه‌گیری پوشش کد، یک مجموعه داده آزمون جهش یافته با ویژگی‌های متمایز است. 


برای مقایسه نحوه عملکرد \lr{AFL} و یک فازر تصادفی، جزئیات الگوریتم‌های این دو فازر را مرور می‌کنیم. الگوریتم \ref{alg:random_fuzz}، فرایند فاز ورودی در یک فازر تصادفی و الگوریتم \ref{alg:afl_fuzz} همین فرایند را در \lr{AFL} نشان می‌دهد. تابع $ Mutate $ یک بایت از ورودی را به‌صورت درجا، با استفاده از فنونی مثل وارون‌کردن بیت، وارون‌کردن بایت، چرخش بیت‌ یا عملیات ریاضی و منطقی، جابه‌جا می‌‌کند. تابع $ Execute $ برنامه را با ورودی جابه‌جا شده اجرا و خرابی‌های احتمالی را گزارش می‌دهد. خطوط سایه زده شده در دو الگوریتم (خطوط 10، 14 و 15) قسمت‌های متفاوت را نشان می‌دهد \cite{DBLP:journals/corr/abs-1711-04596}. 


  
 %\begin{figure*}[t]
 %\centering
 %\removelatexerror
 %\begin{minipage}[t]{9cm}
 %	\vspace{0pt}
 \begin{algorithm}[ht]%[H]%
 	\onehalfspacing
 	\caption[\lr{Basic-Random Fuzzing}]{\lr{Basic-Random Fuzzing} \cite{DBLP:journals/corr/abs-1711-04596}} \label{alg:random_fuzz}
 	
 	\begin{latin}
 		\DontPrintSemicolon
 		\setcounter{AlgoLine}{0}
 		\LinesNumbered
 		
 		%\TitleOfAlgo{Simple Random Fuzzing}
 		
 		\SetKwFunction{RandInt}{RandInt}
 		\SetKwFunction{len}{len}
 		\SetKwFunction{mutate}{mutate}
 		\SetKwFunction{Execute}{Execute}
 		
 		\SetKwInput{KwData}{Input}
 		\SetKwInput{KwResult}{Output}
 		
 		\KwData{$Seeds$, Target program $P$}
 		\KwResult{$MaliciousInputs$}
 		
 		\For{$Seed$ $\in$ $Seeds$}{
 			
 			\For{$iterations \gets 0$ \KwTo $limit$ }{
 				
 				$input \gets Seed$
 				
 				$length \gets$ \len{$Seed$}
 				
 				$mutations \gets$ \RandInt{$length$}
 				
 				\For{$mut \gets 0$ \KwTo $mutations$}{
 					
 					$byte \gets$ \RandInt{$length$}
 					
 					\mutate{$input$, $byte$}
 				}
 				
 				\HiLi $result \gets$ \Execute{$P$, $input$}
 				
 				\If {$result$ is crash}{
 					
 					Append $input$ to $MaliciousInputs$
 				}
 				
 				\HiLi \;
 				
 				\HiLi \;
 				
 				\;
 			}
 			
 		}
 		
 	\end{latin}
 	
 	
 \end{algorithm} 
 %\end{minipage}%
 
 
 %\begin{minipage}[t]{9cm}
 %	\vspace{0pt}
 \begin{algorithm}[ht]%[H]
 	\onehalfspacing
 	\caption[\lr{AFL Fuzzing}]{\lr{AFL Fuzzing} \cite{DBLP:journals/corr/abs-1711-04596}} \label{alg:afl_fuzz}
 	
 	\begin{latin}
 		\DontPrintSemicolon
 		\setcounter{AlgoLine}{0}
 		\LinesNumbered
 		
 		%\TitleOfAlgo{AFL Fuzzing}
 		
 		\SetKwFunction{RandInt}{RandInt}
 		\SetKwFunction{len}{len}
 		\SetKwFunction{Mutate}{Mutate}
 		\SetKwFunction{Execute}{Execute}
 		\SetKwFunction{HasInputGain}{HasInputGain}
 		
 		\SetKwInput{KwData}{Input}
 		\SetKwInput{KwResult}{Output}
 		
 		\KwData{$Seeds$, Target program $P$}
 		\KwResult{$MaliciousInputs$}
 		\For{$Seed$ $\in$ $Seeds$}
 		{
 			\For{$iterations \gets 0$ \KwTo $limit$ }{
 				$input \gets Seed$
 				
 				$length \gets$ \len{$Seed$}
 				
 				$mutations \gets$ \RandInt{$length$}
 				
 				\For{$mut \gets 0$ \KwTo $mutations$}
 				{
 					$byte \gets$ \RandInt{$length$}
 					
 					\Mutate{$input$, $byte$}
 					
 				}
 				
 				\HiLi $result, cov \gets$ \Execute{$P$, $input$}
 				
 				\If {$result$ is crash}
 				{
 					Append $input$ to $MaliciousInputs$
 				}
 				\HiLi \If {\HasInputGain{cov}}
 				{
 					\HiLi  Append $input$ to $Seeds$
 				}
 				
 			}
 		}
 		
 	\end{latin}
 	
 \end{algorithm} 
 %\end{minipage}%
 %%%%%%%%
 % \end{figure*}


\subsection{مشکلات AFL}
آزمون فازی از لحاظ محاسباتی سنگین است. یعنی حتی یک \lr{Input Gain} کوچک نیاز نیازمند هزارها تا میلیون‌ها جابه‌جایی است \cite{DBLP:journals/corr/abs-1711-04596}. از طرفی همه جابه‌جایی‌های یکسان نیستند. \lr{AFL} با بهره‌گیری از بازخورد داده‌های آزمون بهتری را انتخاب می‌کند، اما همچنان آنها را به صورت تصادفی جهش می‌دهد یا جابه‌جا می‌کند. در نتیجه تعداد زیادی داده آزمون تکراری تولید می‌شود که لزوما تأثیری بر بهبود آزمون ندارند. از طرفی در ساختارهای پیچیده، تغییر برخی قسمت‌ها سبب می‌شود تا داده آزمون ورودی نامعتبر شود و سریعاً توسط تجزیه‌گر مردود اعلام گردد. در نتیجه تعداد زیادی داده آزمون هدر رفته خواهیم داشت. \lr{AFL}\gls{Augmented}
\cite{DBLP:journals/corr/abs-1711-04596}
مسئله انتخاب تصادفی مکان‌های جابه‌جایی را مورد مطالعه و راه‌کارهایی برای هوشمندسازی آن ارائه داده است (بخش \ref{sec:augmented-afl}). 


محدودیت دیگر \lr{AFL} \gls{Portability} پایین آن است. \lr{AFL} به‌خاطر استفاده از فراخوانی‌های سیستمی سیستم‌ عامل \lr{Linux} تنها در توزیع‌های آن قابل استفاده است. نسخه‌‌ای از \lr{AFL}، تحت عنوان
 \lr{WinAFL}\LTRfootnote{\href{https://github.com/ivanfratric/winafl}{https://github.com/ivanfratric/winafl}}
برای سیستم عامل ویندوز توسعه داده شده است که البته سرعت پایین و سربار بالایی دارد. \lr{WinAFL} برای افزایش سرعت پیشنهاد می‌کند که تنها یک تابع از برنامه که هدف اصلی آزمون فازی است، ابزارگذاری شود.

به‌دلیل مشکلاتی که اشاره شد، اجرای \lr{AFL} و هرگونه فازر مشابه آن، روی نرم‌افزارهایی مثل \gls{PDF}خوان‌ها که ساختار ورودی آنها پیچیده است به پوشش کد به‌مراتب کمتری دست می‌یابد که حتی با افزایش زمان آزمون نیز نتایج بهبود چندانی نمی‌کند. در این پایان‌نامه یک روش ترکیبی برای تولید داده آزمون ارائه می‌دهیم که پوشش کد بهتری نسبت به فازرهای مبتنی بر جابه‌جایی محض برای ساختارهای پیچیده فراهم کرده و در عین حال ویژگی‌های مثبت جابه‌جایی در فایل‌های دودویی را نیز به‌کار می‌بنند.



\section{فازر \lr{AFL}افزوده}\label{sec:augmented-afl}
 \lr{AFL}افزوده \cite{DBLP:journals/corr/abs-1711-04596} 
تلاش می‌کند با استفاده از فنون یادگیری ماشینی مکان‌های مناسبی را برای جابه‌جایی بایت‌ها پیدا کند. چارچوب ارائه شده در این کار شامل فازر \lr{AFL} و یک  مدل است که مکان‌های مفید برای جابه‌جایی را مشخص می‌کند. در طول اجرا فازر ابتدا مدل را برای اخذ آدرس مکان‌های جهش پرس‌وجو می‌کند و جابه‌جایی‌های فازر را روی مکان‌های بازگردانیده شده متمرکز می‌کند.

برای آموزش مدل فایل ورودی از مجموعه دانه اولیه، فایل جهش یافته و میزان پوشش کد هر دو فایل پس از اجرا توسط \gls{SUT} مورد نیاز است. در فرایند آموزش اگر پوشش کد فایل جهش یافته و فایل والد یکسان باشد (یعنی میزان \lr{Input Gain} صفر باشد)، مکان‌های جابه‌جا شده مکان‌های امیدبخشی نخواهند بود و چنان‌چه پوشش کد فایل جابه‌جا شده افزایش داشته باشد (یعنی میزان \lr{Input Gain} بزرگ‌تر از صفر باشد)، آنگاه جابه‌جایی‌ها امیدبخش محسوب می‌شوند. با تعریف یک تابع خطا که بسته به مقدار \lr{Input Gain} امتـیاز‌هایی به هریک از دو حالت بالا نسبت می‌دهد، مدل در حالت بانظارت آموزش می‌بیند. ورودی مکان‌های جهش در فایل و خروجی میزان امتیاز کسب شده بر اثر هریک از این‌ مکان‌ها است. صورت‌بندی ریاضی مطالب عنوان شده، به‌طور مفصل در \cite{DBLP:journals/corr/abs-1711-04596} ذکر شده است.

 یک ویژگی مثبت این مقاله آموزش و استفاده از چندین مدل متنوع یادگیری ژرف و نیز آزمون چندین برنامه هدف در انجام آزمایش‌ها است که سبب می‌شود تأثیر مدل‌های مختلف را بتوان با یکدیگر مقایسه کرد. برای ایجاد مجموعه آموزش، فازر \lr{AFL} با تعداد 180 دانه اولیه برای هر \gls{SUT} به مدت 24 ساعت اجرا گردیده و اطلاعات لازم ضبط شده‌اند.
 
 الگوریتم \ref{alg:augmented-afl_fuzz}، فرایند فاز ورودی و انجام آزمون فازی در \lr{AFL}افزوده را نشان ‌می‌دهد. خطوط سایه زده شده (خطوط 2، 11، 12 و 13) قسمت‌های اضافه شده به الگوریتم اصلی \lr{AFL}  هستند. 
 چون جابه‌حایی‌های \lr{AFL} در حالت عادی به‌صورت تصادفی است (خطوط 7 و 8 در الگوریتم \ref{alg:afl_fuzz} و خطوط 8 و 9 در الگوریتم \ref{alg:augmented-afl_fuzz})،
  تغییر انجام شده در \lr{AFL}افزوده بدین صورت است که ابتدا مکان‌های مناسب جابه‌جایی برای یک ورودی را توسط مدل پیش‌بینی می‌کند، سپس اجازه می‌دهد تا \lr{AFL}  ورودی را جابه‌جا کند. 
  حال چنان‌چه مجموع شباهت جابه‌جایی‌های \lr{AFL} با جابه‌جایی‌های گزارش شده که مطلوب مدل است، کمتر از یک حد آستانه $ \alpha$  باشد (خط 12 الگوریتم \ref{alg:augmented-afl_fuzz})؛
  این داده آزمون تولیدی برای اجرا ارسال نمی‌گردد. محاسبه شباهت دو رشته بیتی نیز با استفاده از ترکیب عطفی آنها (خط 12 الگوریتم \ref{alg:augmented-afl_fuzz}) به‌‌راحتی امکان‌پذیر است. چنان‌چه دو رشته مشابه نباشند مجموع بیت‌های ترکیب عطفی آنها کمتر از مجموع بیت‌های هریک از  آنها خواهد بود و بالعکس.
  
 
 \begin{algorithm}%[ht]%[H]
	\onehalfspacing
	\caption[\lr{Augmented-AFL Fuzzing}]{\lr{Augmented-AFL Fuzzing}  \cite{DBLP:journals/corr/abs-1711-04596}} \label{alg:augmented-afl_fuzz}
	
	\begin{latin}
		\DontPrintSemicolon
		\setcounter{AlgoLine}{0}
		\LinesNumbered
		
		%\TitleOfAlgo{Augmented-AFL Fuzzing}
		
		\SetKwFunction{RandInt}{RandInt}
		\SetKwFunction{len}{len}
		\SetKwFunction{Mutate}{Mutate}
		\SetKwFunction{Execute}{Execute}
		\SetKwFunction{HasInputGain}{HasInputGain}
		\SetKwFunction{QueryModel}{QueryModel}
		\SetKwFunction{Sum}{Sum}
		
		\SetKwInput{KwData}{Input}
		\SetKwInput{KwResult}{Output}
		
		\KwData{$Seeds$, Target program $P$}
		\KwResult{$MaliciousInputs$}
		
		\For{$Seed$ $\in$ $Seeds$}{
			
			\HiLi $bytemask \gets$ \QueryModel{$Seed$}
			
			\For{$iterations \gets 0$ \KwTo $limit$ }{
				
				$input \gets Seed$
				
				$length \gets$ \len{$Seed$}
				
				$mutations \gets$ \RandInt{$length$}
				
				\For{$mut \gets 0$ \KwTo $mutations$}
				{
					$byte \gets$ \RandInt{$length$}
					
					\Mutate{$input$, $byte$}
					
				}
				
				\HiLi $\mathit{diff}$ $\gets input \oplus Seed$
				
				\HiLi \If{$\sum \mathit{diff}$ $\land$ $bytemask$ < $\alpha$}
				{
					\HiLi   \textbf{Continue}
				}
				
				$result, codecov \gets$ \Execute{$P$, $input$}
				
				\If {$result$ is crash}
				{
					Append $input$ to $MaliciousInputs$
				}
				
				\If {\HasInputGain{codecov}}
				{
					Append $input$ to $Seeds$
				}
			}
		}
	
	\end{latin}%
	
\end{algorithm}%

%%%

\subsection{مشکلات \lr{AFL}افزوده}\label{sec:augmented_afl_problems}

الگوریتم \lr{AFL}افزوده روی چندین قالب فایل و چندین برنامه محتلف، مورد آزمایش قرار گرفته است؛ از جمله قالب‌های فایل
	 \lr{PNG}، \lr{XML} و \lr{PDF}. کمترین بهبود گزارش شده بازهم مربوط به قالب فایل \lr{PDF} و نرم‌افزار \lr{MuPDF} است. علت آن هم ساختار پیچیده و هم حجم بالای فایل‌های \lr{PDF} است. 
	 میزان درصد پوشش کد گزارش شده برای نرم‌افزار \lr{MuPDF} توسط \lr{AFL} اصلی 
	 $11.63$
	 و توسط \lr{AFL}افزوده در بهترین مدل $11.80$ بوده که افزایش ناچیز $0.17$ درصدی را نشان می‌دهد. در بخش نتایج این مقاله همچنین هیچ ارزیابی روی دقت مدل‌های مختلف آموزش داده شده انجام نشده و به یک نتیجه‌گیری کلی بسنده شده است.
	 
	 یک تأکید مهم که در مستندات \lr{AFL} به آن اشاره شده است کوچک نگاه‌ داشتن فایل‌های مجموعه دانه اولیه است (توصیه شده است که از فایل‌هایی با حجم زیر 1 کیلوبایت استفاده شود)؛ زیرا، در صورتی که از فایل‌های با حجم بالایی استفاده شود سرعت فازر به‌شدت کاهش می‌یابد. در واقع جابه‌جایی‌های بسیاری بایستی روی یک دانه اعمال گردد تا یک داده آزمون جدید ایجاد شود. این در حالی است که برخی ساختارها مثل \lr{PDF} ذاتاً حجم بالایی دارند. مثلاً یک فایل \lr{PDF} تنها حاوی عبارت 
	 \lr{Hello Wrold}
	 حجمی در حدود 1 کیلوبایت دارد.
	 برای این ساختارها، تولید یک فایل جدید از ابتدا، مثلاً از روی یک مدل، سرعت تولید داده آزمون را افزایش می‌دهد. استفاده از یک روش ترکیبی یعنی ثابت نگه داشتن بخش‌هایی از فایل و تولید بخش‌های دارای اهمیت در ساختار یک فایل سرعت تولید داده را حتی بیشتر از تولید یک فایل از ابتدا هم افزایش می‌دهد. روش ارائه شده در این پایان‌نامه یک روش ترکیبی است.
	 

	 
%%%%%%%%%%%%%%%%%%%%%%%
%%% Learn and Fuzz  %%%
%%%%%%%%%%%%%%%%%%%%%%%

\section{یادگیری و فاز}
    
مقاله \gls{LearnAndFuzz} \cite{Godefroid:2017:LML:3155562.3155573} روشی برای تولید داده آزمون جهت استفاده در آزمون فازی بر مبنای  \gls{RNN} و مدل \gls{EncoderDecoder} \cite{DBLP:journals/corr/ChoMGBSB14,NIPS2014_5346} ارائه کرده است. در این مقاله ساختار فایل  \gls{PDF} و مرورگر وب
 \lr{Edge}\footnote{مرورگر جدید شرکت مایکروسافت که همراه با سیستم عامل ویندوز 10 معرفی شد.}
  شرکت مایکروسافت، برای آزمون انتخاب شده‌اند. ایده اصلی یادگیری یک مدل مولد زبان روی مجموعه‌ای از ویژگی‌های اشیای \gls{PDF} با داشتن مجموعه‌ای از نمونه‌های اولیه است. ساختار قالب فایل \gls{PDF} و اشیای داده‌ای آن، در پیوست آ توضیح داده شده است.

مدل \gls{EncoderDecoder} \cite{DBLP:journals/corr/ChoMGBSB14,NIPS2014_5346} از دو \gls{RNN} تشکیل شده است. یک شبکه کدگذار که یک توالی ورودی با بُعد متغیر را به یک نمایش بعد ثابت تبدیل می‌کند (بردار زمینه) و یک شبکه کدگشا که یک توالی ورودی بُعد ثابت را می‌گیرد و به یک توالی ورودی با بعد متغیر تولید می‌کند. شبکه کدگشا توالی‌‌های خروجی را با استفاده از کاراکتر خروجی پیش‌بینی شده که در مرحله‌زمانی $ t $ به عنوان کاراکتر ورودی برای مرحله‌زمانی  $ t+1 $ تولید شده است، تولید می‌کند. یک بازنمایی از معماری این مدل در ‏شکل \ref{ch3_encoder_decoder_model_crop.png} نشان داده شده است. این مدل نخستین بار در وظیفه ترجمه ماشینی استفاده شده است \cite{DBLP:journals/corr/ChoMGBSB14}. با توصیف بیان شده، این معماری امکان یادگیری توزیع شرطی 
$ p(<y^{(1)}, y^{(2)}, y^{(3)}, ..., y^{(n)}>| <x^{(1)}, x^{(2)}, x^{(3)}, ..., x^{(n)}>) $ 
را فراهم می‌کند.

\begin{figure}%[tbh!]%[ht]%[t!]
	\centering
	\includegraphics[width=0.7\textwidth, clip=true,  trim= 0 0 0 0]{chapter3/ch3_encoder_decoder_model_crop.png}
	%\includegraphics[width=\textwidth]{figs/chapter1/ch1_fuzz_testing_flowchart2.png}
	\caption[مدل \gls*{EncoderDecoder}]
	{
		مثالی از معماری مدل کدگذار-کدگشا، متشکل از دو \gls{RNN} و بردار زمینه \lr{C}. این مدل برای یادگـیری تولید توالی خروجی 
		$ <y^{(1)}, y^{(2)}, y^{(3)}, ..., y^{(n_y)}> $ 
		 از روی بردار زمینه، که نماینده توالی ورودی 
		$ <x^{(1)}, x^{(2)}, x^{(3)}, ..., x^{(n_x)}> $ 
		  است، به‌کار می‌رود \cite{Goodfellow-et-al-2016}.
	}
	\label{ch3_encoder_decoder_model_crop.png}
	%\ref{ch3_encoder_decoder_model_crop.png}
\end{figure}

\subsection{آموزش مدل کدگذار-کدگشا}
مدل کدگذار-کدگشا در روش یادگیری و فاز، با استفاده از تکه‌ اشیای \gls{PDF} که هر یک به‌صورت توالی‌ای از کاراکترها در نظر گرفته شده است، آموزش می‌بیند. برای آموزش، ابتدا همه فایل‌های حاوی اشیای \gls{PDF} به‌شکل یک توالی از کاراکترها به یکدیگر الحاق  می‌شوند که منجربه ایجاد یک توالی بزرگ از کاراکترها به‌صورت 
$ \hat{s} = <s^{(1)}, s^{(2)}, s^{(3)}, ..., s^{(n)}> $
 خواهد شد. سپس توالی $ \hat{s} $ به چند زیر توالی آموزشی با طول ثابت $ d $ شکسته می‌شود به نحوی که $ i $امین نمونه آموزشی عبارت است از: 
 $ t_i = \hat{s}[i*d: (i+1)*d] $
  که 
  $ s[l:u] $
  یک زیرتوالی از $ s $ بین اندیس‌های $ l $ و $ u $ را نشان می‌دهد. در ادامه، توالی خروجی برای هر توالی آموزشی $ t_i $ عبارت است از توالی ورودی که یک واحد به سمت راست شیفت داده شده است؛ یعنی:
  $ o_{t_i} = \hat{s}[i*d+1: (i+1)*d+1] $.
  مدل سپس به‌صورت \gls{EndToEnd} آموزش می‌بیند تا یک مدل مولد روی مجموعه همه توالی‌های ایجاد شده، یادگیری شود. شکل \ref{ch3_learn_and_fuzz_model_crop.pdf} مدل ارائه شده در این مقاله را نشان می‌دهد.

\begin{figure}%[tbh!]%[ht]%[t!]
	\centering
	\includegraphics[width=\textwidth, clip=true,  trim= 0 0 0 0]{chapter3/ch3_learn_and_fuzz_model_crop.pdf}
	%\includegraphics[width=\textwidth]{figs/chapter1/ch1_fuzz_testing_flowchart2.png}
	\caption[مدل \gls*{RNN} \gls*{EncoderDecoder} برای تولید اشیای داده‌ای \gls*{PDF}]
	{
		مدل \gls{RNN} \gls{EncoderDecoder} برای تولید اشیای داده‌ای \gls{PDF} \cite{Godefroid:2017:LML:3155562.3155573}.
	}
	\label{ch3_learn_and_fuzz_model_crop.pdf}
	%\ref{ch3_learn_and_fuzz_model_crop.pdf}
\end{figure}


از آنجایی که مدل بحث شده در حالت بدون نظارت آموزش می‌بیند، برچسب‌های تولید شده به‌طور صریح برای مشخص کردن این که مدل یادگرفته شده چه‌قدر خوب عمل می‌کند، آزموده نشدند. به‌جای آن چندین مدل آموزش داده می‌شود که در تعداد دوره‌ها (بخش \ref{feedforwardtraining}) متفاوت هستند. مدل در 10، 20، 30، 40 و 50 دوره آموزش داده شده است. در تنظیمات شبکه ارائه شده در این مقاله، یادگیری روی هر دوره حدود 12 دقیقه طول می‌کشد و بنابراین مدل با 50 عدد حدودا 10 ساعت زمان نیاز دارد تا کامل شود. شبکه به‌صورت \gls{RNN} با 2 لایه مخفی که هر لایه 128 سلول \gls{LSTM} دارد، است.


\subsection{تولید داده‌های جدید}\label{sec:new_data_generation}
از مدل یادگیری شده برای تولید اشیای جدید \gls{PDF} استفاده شده است. راهبردهای مختلفی برای تولید اشیا، بسته به راهبرد نمونه‌برداری که برای نمونه‌برداری از توزیع یادگیری شده به‌کار می‌رود، وجود دارند. در این مقاله، با یک \gls{Prefix} از توالی \texttt{"obj"} (که آغاز یک شی را مشخص می‌کند) شروع و سپس مدل را پرس‌وجو کرده تا یک توالی از کاراکترهای خروجی تولید گردد، تا زمانی که مدل \texttt{"endobj"} را که مشخص کننده انتهای یک شی است، تولید نماید. سپس این مقاله، سه راهبرد مختلف را برای تولید اشیای جدید از روی مدل، استفاده کرده است \cite{Godefroid:2017:LML:3155562.3155573}:

\begin{itemize}
	\item{
		\textbf{\lr{NoSample}.}
	 در این راهبرد تولید، کاراکتر پیش‌بینی شده با بالاترین احتمال به‌عنوان کاراکتر بعدی انتخاب می‌شود که در واقع انتخابی \gls{Greedy} است. این راهبرد برای تولید اشیای \gls{PDF}ای که \gls{Wellformed} و \gls{Consistent} هستند، کاملاً نتیجه‌بخش است ولی تعداد اشیائی را که می‌توان تولید کرد، محدود می‌کند. با دادن یک پیشوند مثل  \texttt{"obj"}  بهترین توالی از کاراکترهای بعدی به‌صورت یکتا مشخص می‌شود و بنابراین این راهبرد همان شی را نتیجه می‌دهد. این محدودیت مغایر با اهداف آزمون فازی بوده و مانع مفید بودن استفاده از آن در آزمون فازی می‌شود. 
}
\item{
	\textbf{\lr{Sample}.}
در این راهبرد تولید، به‌جای انتخاب محتمل‌ترین کاراکتر پیش‌بینی شده، از توزیع یادگرفته شده، برای نمونه‌برداری کاراکتر بعدی در توالی‌ای که پیشوند آن داده شده است، استفاده می‌شود. راهبرد نمونه‌برداری قادر به تولید مجموعه گوناگونی از اشیا با ترکیب الگوهای مختلفی که از اشیای موجود یادگرفته شده است، خواهد بود. به دلیل نمونه‌برداری اشیای تولید شده، تضمینی نیست که آنها به میزان راهبرد قبلی،  \gls{Wellformed} باشند، که این ویژگی از منظر آزمون فازی مفید است.
}

\item{
	\textbf{\lr{SampleSpace}.}
این راهبرد ترکیب دو راهبرد قبلی است. \lr{SampleSpace} توزیع احتمالی برای تولید کاراکتر بعدی را تنها زمانی که پیشوند توالی کنونی، با فضای خالی خاتمه می‌یابد، نمونه‌برداری می‌کند. برای میان \gls{Token}‌ها (یعنی پیشوندهایی که با فضای خالی خاتمه نمی‌یابند)، بهترین کاراکتر را انتخاب می‌کند (راهبرد \lr{NoSample}) است که در مقایسه با راهبرد \lr{Sample}،  ورودی‌های خوش‌شکل‌تری تولید می‌شود. چون نمونه‌برداری در پایان کاراکترهای فضای خالی صورت می‌گیرد؛ نمونه‌های جدید نیز در این راهبرد، ساخته خواهند شد.
}
\end{itemize}

\subsection{اعمال آزمون فازی}

برای فاز داده‌های آزمون تولید شده، مقاله الگوریتم جدیدی تحت عنوان \lr{SampleFuzz} ( الگوریتم \ref{alg:sample_fuzz})، ارائه کرده است. \lr{SampleFuzz} توزیع یادگیری شده
 $ D(x,\theta) $،
احتمال فازینگ یک کاراکتر $ t\_fuzz $ و احتمال آستانه‌ $ p\_t $ که تصمیم می‌گیرد آیا کاراکتر پیش‌بینی شده اصلاح شود یا نه، را به عنوان ورودی می‌پذیرد. سپس برای توالی خروجی $  seq $، الگوریتم $ D(x,\theta) $ را نمونه‌برداری می‌کند تا کاراکتر بعدی یعنی $ c $ و احتمال آن $ p(c)$ را در یک مرحله زمانی مشخص $ t $ به‌دست آورد. اگر احتمال $ p(c)$ از آستانه فراهم شده توسط کاربر ($ p\_t $) بالاتر باشد؛ یعنی، اگر مدل مطمئن باشد که کاراکتر بعدی به احتمال زیاد  $ c $ است، الگوریتم تصمیم می‌گیرد تا یک کاراکتر دیگر $ c' $ که احتمال کمینه $ p(c') $ را دارد، جایگزین $ c $ کند. البته این اصلاح (فاز) تنها در صورتی که عدد تصادفی تولید شده $ p\_fuzz $ از ورودی $ t\_fuzz $  مقدار بیشتری داشته باشد، انجام می‌شود، که بدین ترتیب اجازه می‌دهد تا کاربر بتواند بر روی احتمال (درصد) کاراکترهای فاز شده کنترل داشته ‌باشد.

 
\begin{algorithm}[ht]
	\onehalfspacing
	\caption[\lr{SampleFuzz}]{\lr{SampleFuzz} \cite{Godefroid:2017:LML:3155562.3155573}} \label{alg:sample_fuzz}
	\begin{latin}
		%\begin{algorithmic}[1]
		\DontPrintSemicolon
		\setcounter{AlgoLine}{0}
		\LinesNumbered
		
		\SetKwFunction{Rnd}{Random}
		\SetKwFunction{EndsWith}{EndsWith}
		\SetKwFunction{Sample}{Sample}
		
		\SetKwInput{KwData}{Input}
		\SetKwInput{KwResult}{Output}
		
		\KwData{$D(x, \theta )$, $t\_fuzz$, $p\_t$}
		\KwResult{$ seq $}
		
		\BlankLine
		
		$ seq \gets$ "$obj$"\;
		
		%initialization\;
		
		\While{$ \sim seq.$\EndsWith{"$endobj$"}}
		{
			$ c, p(c) \gets sample(D(seq,\theta)) $ \tcc*{Sample c from the learnt distribution}\;
			
			$ p\_fuzz \gets $ \Rnd($0,1$) \tcc*{Random variable to decide whether to fuzz}\;
			
			\If{ $p\_fuzz > t\_fuzz \wedge p(c) > p\_t $ }
			{
				$ c \gets argmin_{c'}\{ p(c') \sim D(seq, \theta) \} $ \tcc*{Replace c by c’ (with lowest likelihood)}\;
			} 
			
			$ seq \gets seq + c$\;
			
			\If{$ len(seq) > MaxLen  $}
			{
				$ seq \gets$ "$obj$" \tcc*{Reset the sequence}\;
			}
				
		}
		\textbf{Return}   $ seq $\;

		%\end{algorithmic}
	\end{latin}
\end{algorithm}

 
 الگوریتم \lr{SampleFuzz} اطمینان می‌یابد که طول شی با عدد $ MaxLen $ محدود شود؛ اما، شرط حلقه $ while $ الگوریتم تضمین نمی‌کند که همواره خاتمه یابد. با این حال، نویسندگان گزارش کرده‌اند که در آزمایش‌های انجام شده در عمل، الگوریتم همواره پایان یافته‌است. چیدمان آزمایش‌ها و نیز تحلیل جامع نتایج در 
 \cite{Godefroid:2017:LML:3155562.3155573}
  آمده است که نشان از برتری الگوریتم \lr{SampleFuzz} در مقایسه با روش تصادفی و افزایش نسبی پوشش کد در سطح دستورات برنامه است.
  
  
  
  \subsection{مشکلات روش یادگیری و فاز}
  مقاله یادگیری و فاز برای نخستین بار از یادگیری ماشینی در تولید داده آزمون فازی استفاده کرده ‌است. در عین حال روش ارائه شده در این مقاله دارای مشکلات و کاستی‌هایی است که عبارتند از:
  
  \begin{enumerate}
  	\item %[\blacklozenge]
  	{
  	همواره با یک پیشوند ثابت شروع به تولید داده جدید می‌کند که این سبب می‌شود تنوع کمتری حاصل شود. در نتیجه ناچار به پیچیده کردن فرایند نمونه برداری از مدل می‌شود. همچنین این پیشوند کوتاه است و حاوی اطلاعات بعدی یک شی داده‌ای نیست. می‌توان مدل‌هایی با قابلیت پذیرفتن پیشوند‌های متفاوت ایجاد کرد.
  	}
  	\item  
  	{
  	مدل کدگذار-کدگشا معمولاً برای نگاشت توالی‌های با طول و دامنه‌های متفاوت به یکدیگر، استفاده می‌شود. بارزترین مثال وظیفه ترجمه ماشینی است که هدف آن نگاشت جمله‌های یک زبان به زبان‌ دیگر است \cite{NIPS2014_5346}. در این وظیفه، طول جمله زبان مقصد لزوماً برابر طول معادل همان جمله در زبان مبدأ نیست. به‌همین خاطر نیازمند معماری پیچیده کدگذار-کدگشا هستیم.
  	وظیفه یادگیری خودکار ساختار فایل بیشتر به ایجاد یک مدل زبانی روی الفبای زبان فایل ورودی شباهت دارد که در نتیجه استفاده از یک مدل زبانی عصبی در این وظیفه، منطقی‌تر و احتمالاً نتیجه‌بخش‌تر به نظر می‌رسد. چنان‌چه در روش یادگیری و فاز هم می‌بینیم که ساختار و وظایف شبکه کدگذار و کدگشا به خوبی تفکیک و تفهیم نگردیده است و صرفاً به پیچیده‌تر شدن مدل انجامیده است. 
    }

	\item  
{
فقط داده‌های متنی فایل تولید و فاز شده‌اند در حالی که یک فایل باینری در حالت کلی حاوی ترکیبی از داده‌های متنی و غیرمتنی است. می‌توان یک روش ترکیبی برای این منظور ارائه داد.
}
	\item  
{
	الگوریتم فاز ارائه شده، همان‌طور که دیدیم، ممکن است هیچ‌گاه پایان نیابد یا اجرای الگوریتم برای تولید یک نمونه خیلی طولانی شود که خوب به‌نظر نمی‌رسد. می‌توان با تعریف شرایطی از این اتفاق جلوگیری کرد. خاتمه‌یافتن در هرصورت، یک ویژگی است که در تعریف اصطلاح الگوریتم وجود دارد و از این بابت این مسئله بسیار مهم تلقی می‌شود.
}
	\item  
{
	معماری مدل پیشنهادی به نحوی است که توالی‌های تولید شده برای آموزش در برگیرنده همه اطلاعات و وابستگی‌های ساختار فایل نیستند. در هنگام تولید توالی‌های اموزشی هر بار به اندازه $ d $ واحد از روی مجموعه داده پرش می‌کنیم.  اگر $ d $ کوچک باشد، توالی‌های کوچکی خواهیم داشت که لزوماً شامل یک وابستگی طولانی نیستند و اگر $ d $ بزرگ انتخاب شود، تعداد وابستگی‌های کمتری تشخیص داده می‌شود. به این نکته بایستی توجه کرد که وابستگی‌های فایل‌های با قالب مختلف، یکسان نیستند. می‌توان راهکارهای بهتری در نحوه تولید ورودی و خروجی شبکه یافت.
}
	\item  
{
	در هر بار اجرا فقط یک نمونه را تولید می‌کند و بنابراین نمونه‌های تولید شده در چنیدن اجرای متوالی کاملاً مستقل از یکدیگر هستند. ممکن است چند نمونه متوالی به‌هم وابستگی داشته باشند در این صورت بایستی بتوان به‌نحوی این پارامتر را در تولید نمونه‌ها گنجاند.  
}

	\item  
{
	و بلأخره در مقاله یادگیری و فاز تنها از یک مدل یادگیری ژرف برای تولید داده آزمون استفاده شده است که می‌توان با انتخاب مدل‌های مختلف یا تغییر در ابرپارمترهای در مورد تأثیر نوع مدل نیز بحث کرد و آزمایش‌های جدید را تجربه نمود.  
}

  \end{enumerate} 


\section{دیگر کارهای مرتبط}
تعداد دیگری کار در زمینه تولید داده آزمون در فازرها و به‌ویژه در زمینه یادگیری ساختار فایل انجام شده است. در فصل اول، به‌طور خلاصه به‌ برخی از آنها اشاره کردیم
\cite{Bastani:2017:SPI:3140587.3062349, Hoschele:2016:MIG:2970276.2970321}.
بیشتر این روش‌ها بسیار پیچیده بوده، درک و پیاده‌سازی آنها مشکل است و سربار زیادی را به عملیات آزمون فازی یا عملیات پیش‌پردازش، عملیات پیش از انجام آزمون فازی، تحمیل می‌کنند. در ضمن اغلب‌ آنها در نهایت محدود به ساختارهای ورودی خاصی می‌شوند. تمرکز ما بر روی روش‌هایی است که کاملاً عملیاتی بوده و قابل استفاده بر روی نرم‌افزارهای دنیای واقعی هستند.


در 
\href{http://parsa.iust.ac.ir/reverse-engineering-lab/}{آزمایشگاه  تحقیقاتی مهندسی معکوس دانشگاه علم و صنعت ایران}،
 تعدادی زیادی پروژه، کار پژوهشی و پایان‌نامه در ارتباط با آزمون فازی و تولید داده آزمون، نگارش شده است
 \footnote{
فهرست کاملی از کارهای مرتبط انجام شده، از طریق وب‌سایت رسمی آزمایشگاه به نشانی 
\href{http://parsa.iust.ac.ir}{\lr{http://parsa.iust.ac.ir}} 
قابل دسترسی هستند. همچنین فهرستی سازمان‌دهی شده و کامل در ارتباط با منابع مختلف آزمون فازی در نشانی 
\href{https://github.com/secfigo/Awesome-Fuzzing}{\lr{https://github.com/secfigo/Awesome-Fuzzing}} 
وجود دارد. 
}.
یعقوبی 
 \cite{yaghoubi1392}
 یک روش خودکار تولید داده آزمون با استفاده از الگوریتم ژنتیک برای آزمون فازی جعبه سیاه مرورگرهای وب، ارائه داده است. هدف الگوریتم در این فرایند تکاملی، تشخيص مجموعه‌ای از برچسب‌های 
 \lr{HTML}
 است که حداکثر درهم ريختگی حافظه و کاهش سرعت اجرايی مرورگر را موجب ‌شوند. در واقع دسته برچسب‌های  
 \lr{HTML}
 به‌عنوان صفحات ورودی هر مرورگر، طوری ساخته مي‌شوند که در هر مرحله نسبت‌ به ‌دسته برچسب‌های قبل، ميزان مصرف حافظه اصلی را بالا  ببرند. 
 
 امکان بهره‌گیری از این روش برای آزمون فازی و شناسایی خطاها و آسيب‌پذيری‌های دیگر نرم‌افزارهایی که کد منبع آنها در دسترس نيست نیز وجود دارد. اما تمرکز این روش روی تولید برچسب‌های \lr{HTML} بوده و سرعت همگرایی آن کند است. علاوه بر این، تنها هدف الگوریتم ژنتیک در اینجا، افزایش حافظه تعریف شده است در حالی که می‌دانیم صرف افزایش حافظه منجربه کشف خطا نمی‌شود. در سیستمی با حافظه پایین ممکن است سیستم عامل برنامه را پیش از سقوط متوقف سازد یا برنامه فقط بر اثر کمبود حافظه نتواند یک ورودی به اندازه کافی بزرگ را پردازش کند. در ساختارهای پیچیده که تجزیه‌گر سخت‌گیری دارند، تولید داده به این روش بسیار زمان‌بر است؛ زیرا، در عمل خیلی از داده‌های آزمون تولیدی مورد پردازش قرار نمی‌گیرند و حافظه‌ای به آنها تخصیص داده ‌نمی‌شود.
 
 امینی
 \cite{amini1395}
 یک روش تکاملی با هدف حل مشكل بزرگی فضای ورودی، برای تولید داده آزمون ارائه داده است. در روش‌هایی مشابه این روش مسئله تولید داده آزمون به صورت یک مسئله جست‌وجو تعریف می‌گردد و سپس از الگوریتم‌های اکتشافی برای یافتن یک حل بهینه استفاده می‌شود. در راهكار امینی، زمانی که يک داده آزمون يک \gls{VulnerablePattern} را شناسایی می‌کند، داده آزمون بعدی به‌گونه‌ای توليد می‌شود که منجربه افشای آن آسيب‌پذيری گردد. در نتيجه احتمال کشف آسيب‌پذيری‌های شناسایی شده بهبود خواهد داشت. \gls{VulnerablePattern} در سطح کد منبع برنامه قابل شناسایی بوده و بنابراین این روش در فازرهای جعبه سفید قابل استفاده است. البته می‌توان آن را به فازرهای جعبه خاکستری هم تعمیم داد. روش مذکور روی برنامه‌های دنیای واقعی ارزیابی  نشده است. همچنین یافته‌های آزمایش‌های آن، با هیچ روش هوشمند تولید داده آزمونی، مقایسه نگردیده است.


\section{خلاصه}
جدول \ref{tabel:related-work}، یک جمع‌بندی از مزایا و معایب سه کار اصلی معرفی شده در این فصل 
\cite{Zalewsky2013,DBLP:journals/corr/abs-1711-04596,Godefroid:2017:LML:3155562.3155573}
را در کنار یکدیگر نشان می‌دهد. همچنین فازر 
\lr{FileFuzz} \cite{Sutton:2007:FBF:1324770}،
که یک فازر قالب فایل تصادفی و بسیار اولیه است به‌عنوان یک فازر مبنایی برای مقایسه، در جدول آورده شده است. البته تا به امروز، فازرهای بسیار زیادی در زمینه آزمون فازی قالب فایل توسعه داده شده‌اند که مجال بررسی همه آنها در این پایان‌نامه نبوده و در اینجا صرفاً آن دسته از کارهایی را بیان کردیم که قصد داریم در روش پیشنهادی خود مشکلات موجود در آنها را تا حد امکان برطرف کنیم. برخی از این فازرها مانند 
\cite{Godefroid:2012:SWF:2090147.2094081}
به صورت داخل سازمانی و یا تجاری ارایه شده‌اند و کد منبع آنها در دسترس نیست. عدم دسترسی به کد منبع یک فازر و در نتیجه عدم امکان تغییر در پیمانه‌های مختلف آن را می‌توان انگیزه‌ای برای توسعه فازرهای جدید دانست. 

\begin{table}%[ht]
	\caption{مقایسه کارهای مرتبط}
	\label{tabel:related-work}
	\centering
	\onehalfspacing
	\begin{tabularx}{\textwidth}{p{.15\textwidth}p{.215\textwidth}p{.265\textwidth}p{.265\textwidth}}
		
		\toprule[1.5pt] 
		فازر &
		مشخصه‌ها&
		مزایا&
		معایب\\
		\midrule[1.5pt]
		\lr{FileFuzz}\cite{Sutton:2007:FBF:1324770}&
		$\bullet$
		مبتنی بر جابه‌جایی
		
		$\bullet$
		جعبه ‌سیاه
		&
		
		\textcolor{green}{$\bullet$}
		سادگی پیاده‌سازی
		
	    \textcolor{green}{$\bullet$}
		عدم نیاز به اطلاع داشتن از ساختار ورودی و ساختار \lr{SUT}	
		&
		\textcolor{red}{$\bullet$}
		تعیین مکان جابه‌جایی‌ها به‌صورت کاملاً تصادفی
		
		\textcolor{red}{$\bullet$}
		پوشش کد بسیار پایین و غیر قابل قبول، برای بیشتر نرم‌افزارها پس از آزمون
		\\
		\hline  
		\lr{AFL}\cite{Zalewsky2013}&
			$\bullet$
		مبتنی بر جابه‌جایی
		
			$\bullet$
		مبتنی بر بازخورد
		
			$\bullet$
		جعبه ‌خاکستری
		&
		
		\textcolor{green}{$\bullet$}
		بیشینه کردن پوشش کد با بهره‌گیری از الگوریتم ژنتیک
		
		\textcolor{green}{$\bullet$}
		قابل استفاده به صورت جعبه‌سیاه، جعبه‌خاکستری و جعبه‌سفید	
		&
		\textcolor{red}{$\bullet$}
		تعیین مکان جابه‌جایی‌ها به‌صورت تصادفی
		
		\textcolor{red}{$\bullet$}
		تولید تعداد زیادی داده آزمون هدر رفته
		
		\textcolor{red}{$\bullet$}
		سرعت پایین فاز در فایل‌های با حجم بیشتر از 1 کیلوبایت
		\\
		\hline 
		\lr{َAFL}افزوده 
		\cite{DBLP:journals/corr/abs-1711-04596}&
			$\bullet$
		مبتنی بر جابه‌جایی
		
			$\bullet$
		مبتنی بر بازخورد
		
			$\bullet$
		جعبه ‌خاکستری&
			\textcolor{green}{$\bullet$}
		تعیین مکان (آدرس نسبی) جابه‌جایی‌ها از طریق یک مدل ارزش‌دهی به بایت‌ها&
			\textcolor{red}{$\bullet$}
		پوشش کد پایین برای نرم‌افزارهای با ساختار ورودی پیچیده
		\\
		\hline 
		\lr{Learn\&Fuzz} 
		\cite{Godefroid:2017:LML:3155562.3155573}&
		$\bullet$
		مبتنی بر تولید
		
		
		$\bullet$
		جعبه سفید&
		\textcolor{green}{$\bullet$}
		یادگیری گرامر فایل و تولید داده‌های آزمون جدید با استفاده از مدل‌ کدگذار-کدگشا
		
		\textcolor{green}{$\bullet$}
		تعیین مکان فاز داده آزمون توسط مدل یادگیری شده
			
			\textcolor{green}{$\bullet$}
		تعیین مقدار فاز شده داده آزمون&
		
		\textcolor{red}{$\bullet$}
		حذف کامل داده‌های دودویی
		
		\textcolor{red}{$\bullet$}
		شروع تولید داده آزمون با یک پیشوند ثابت
		
			\textcolor{red}{$\bullet$}
			عدم تضمین پایان یافتن الگوریتم فاز
		
			\textcolor{red}{$\bullet$}
			در دسترس نبودن فازر، مجموعه داده و دیگر پارامترهای مهم مربوط به آموزش مدل
		\\
		\bottomrule[1.5pt]
		
	\end{tabularx} 
\end{table}


%\section{خلاصه}
به‌طور خلاصه، در این فصل ابتدا فازر قالب فایل
 \lr{AFL} \cite{Zalewsky2013}
را به‌عنوان یک فازر با روش تولید داده آزمون مبتنی بر جابه‌جایی مورد مطالعه قرار دادیم و دیدیم که به علت جابه‌جایی تصادفی بایت‌های فایل، تعداد زیادی داده آزمون بیهوده برای ساختارهای پیچیده ایجاد می‌کند. سپس \lr{AFL}افزوده 
\cite{DBLP:journals/corr/abs-1711-04596}
را توضیح دادیم از که فنون یادگیری ماشینی برای تعیین محل مکان‌های جابه‌جایی استفاده می‌کند. ایده اصلی در این روش، آن است که در قالب‌های فایل، معمولاً فقط قسمت‌های کوچکی از فایل برای تجزیه‌گر مهم هستند و در نتیجه بیشتر جابه‌جایی آن‌ها منتهی‌ به اجرای مسیرهای جدید \gls{SUT} می‌گردد. این روش نیز روی قالب پیچیده‌ای مانند \gls{PDF} بهبود ناچیزی داشته است. با این حال روی قالب‌های کاملاً دودویی مثل قالب فایل \lr{PNG} در زمان برابر به پوشش کد بهتری دست یافته است.

در ادامه فصل، روش یادگیری و فاز
 \cite{Godefroid:2017:LML:3155562.3155573}
  را توضیح دادیم که از مدل کدگذار-کدگشا برای تولید و فاز داده آزمون استفاده می‌کند. ایده اصلی این روش یادگیری خودکار گرامر برای یک قالب فایل متنی و سپس تولید داده با روش مبتنی بر تولید است. این روش با یک روش تصادفی که در ابتدای فصل نیز مطرح شد، مقایسه شده‌ و نتایج بهبود داشته است. روش یادگیری و فاز روی ابزار متن بسته مرورگر \lr{Edge} شرکت مایکروسافت و توسط محققین همین شرکت مورد آزمایش قرار گرفته و هیچ‌گونه کدمنبع و مجموعه داده‌ای از آن منتشر نشده است. در مقابل اما فازر \lr{AFL} متن‌باز و رایگان است. \lr{AFL}افزوده نیز به‌تازگی توسط توسعه‌دهندگان آن منتشر شده است. 

در پایان این فصل مروری اجمالی بر دو کار پیشین انجام شده در زمینه آزمون فازی و تولید خودکار داده آزمون در آزمایشگاه مهندسی معکوس دانشگاه علم و صنعت ایران، که این پایان‌نامه در آنجا نگارش یافته است، داشتیم. هر دو کار 
\cite{amini1395,yaghoubi1392}
از الگوریتم ژنتیک برای تولید داده‌های آزمون استفاده کرده بودند که مزایا و معایب هریک نیز بیان شد. در این پایان‌نامه ما از الگوریتم‌های یادگیری ژرف برای تولید خودکار داده آزمون استفاده خواهیم کرد. با توجه به مزایا و معایبی که در کلیه روش‌های مطرح در این فصل شناسایی و معرفی کردیم، در فصل بعد یک روش ترکیبی که بتواند از ویژگی‌های خوب جابه‌جایی تصادفی فایل‌های دودویی در کنار تولید فایل‌های متنی براساس مدل، استفاده کند، ارائه خواهیم کرد. هدف بهبود برخی کاستی‌های شناسایی شده در کارهای مطرح در این فصل و حل مسائل اصلی اشاره شده در فصل اول است.






  % فصل سوم: سنجش و بهبود آزمون‌پذیری کد منبع و طراحی نرم‌افزار %
% !TeX root=maintext.tex
% !TeX TS-program = XeLaTeX
% !TEX spellcheck = fa
% chapter4

\chapter{ارزیابی}\label{chapter:4}
\thispagestyle{empty}
عنوان این فصل می‌تواند «ارزیابی روش/فن/طرح پیشنهادی» یا «مدل‌سازی و ارزیابی» یا «ارزیابی و مقایسه» یا موارد مشابه باشد كه شامل شرح روش اندازه‌گیری، مدل‌سازی و ارزیابی و نتایج حاصل از ارزیابی است.
   
   
   \section{معرفی روش ارزيابی و مدل‌سازی }
   
   
   \section{آزمایش‌ها }
   
   
      \section{نتایج و تفسیر آنها }
      
      
   
     \section{خلاصه} % فصل چهارم: سنجش و بهبود آزمون‌پذیری نیازمندی‌ها %
% !TeX root=_main_.tex
% chapter5

\chapter{ارزیابی روش پیشنهادی}\label{ch:5}
\thispagestyle{empty}

\epigraph{
«من شکست نخورده‌ام. من فقط $10,000$ راه پیدا کردم که کار نمی‌کند.»
}
{$ \maltese $ {\large توماس ادیسون}}


\noindent
ما روش پیشنهادی خود را روی قالب فایل \gls{PDF} به‌عنوان یک قالب فایل با ساختار پیچیده و ترکیبی و نرم‌افزار متن‌باز و رایگان
 \lr{MuPDF} \cite{MuPDF2018}
 مورد ارزیابی قرار دادیم. در این فصل به تشریح چیدمان آزمایش‌ها، معیارهای ارزیابی، یافته‌های حاصل از آزمایش‌ها و نتیجه‌گیری‌‌ از آنها می‌پردازیم. هدف علاوه بر اثبات کارایی و خوب بودن روش پیشنهادی، شناسایی و تفکیک پارامترهای مهم در هنگام استفاده از فنون یادگیری ماشینی در آزمون فازی و تولید خودکار داده آزمون است. استفاده از یادگیری ماشینی در آزمون فازی حوزه پژوهشی بدیع و ناشناخته‌ای است.  بنابراین در این فصل معیارهایی را که در هر روش پیشنهادی بایستی مورد ارزیابی قرار بگیرند، نیز مشخص خواهیم نمود. سپس روش پیشنهادی خود را با آنها مورد ارزیابی قرار می‌دهیم. 


\section{مورد مطالعاتی}
  نرم‌افزار
 \lr{MuPDF} \cite{MuPDF2018}
 مجموعه غنی از کتابخانه‌ها، \lr{API}ها و ابزارهای کاربردی برای کار با فایل‌های \gls{PDF} است. این نرم‌افزار مشتمل بر سه قسمت \lr{API}، ابزار \lr{mutool} و ابزار \gls{PDF}خوان است. ابزار \lr{mutool} برای ایجاد و تغییر فایل‌های \gls{PDF} استفاده می‌شود. ابزار \gls{PDF}خوان نیز برای نمایش فایل‌های \gls{PDF} به‌کار می‌رود. بسته نرم‌افزاری کامل، به زبان \lr{C} نوشته شده و در سیستم عامل‌های ویندوز، لینوکس و \lr{Mac} و همچنین اندروید قابل استفاده است. به‌دلیل قابلیت حمل بالا و کاربرد زیادی که دارد، پیدا کردن خطا در آن حائز اهمیت است. در آزمایش‌ها ما ابزار \gls{PDF}خوان نسخه ویندوز را مورد آزمون قرار دادیم. منظور از عبارت \lr{MuPDF} در ادامه این فصل همان ابزار \gls{PDF}خوان است.
 
  
  \lr{MuPDF}
  البته در نسخه‌های جدید قالب‌های فایل دیگری را نیز پشتیبانی می‌کند؛ از جمله قالب‌های  فایل \lr{XPS} و \lr{EPub}. اما قالب فایل \lr{PDF} بدون شک رایج‌ترین قالب اسناد و انتشارات الکترونیکی است که بیشتر مردم از آن استفاده می‌کنند. از این‌رو تضمین کیفیت و امنیت \lr{PDF}خوان‌ها به‌ویژه در تلفن‌های همراه اهمیت دو چندانی دارد. افزون بر آنچه گفته شد قالب فایل \gls{PDF} و نرم‌افزار \lr{MuPDF} تمامی مسائلی را که در بخش \ref{problem_statement} بدان اشاره کردیم، یعنی ساختار پیچیده ورودی، ساختار پیچیده کد و غیره، دارند و از طرفی در فصل \ref{related_work} دیدیم که فازرهای مبتنی بر جابه‌جایی مانند \lr{AFL}  و \lr{AFL}افزوده نتواسته بودند به پوشش کد خوبی حیـن آزمون فازی این نرم‌افزار دست‌یابند. بنابراین انتخاب آنها به‌عنوان مورد مطالعاتی به‌دلایل بالا و نیز در جهت فراهم شدن امکان مقایسه با کارهای پیشین، قابل توجیه است. آزمایش‌ها برای هر قالب فایل و هر نرم‌افزار دیگری به‌نحوی که در ادامه این فصل توضیح داده می‌شود قابل اجرا است. 
  
  
 ساختار قالب فایل \gls{PDF} در پیوست \ref{appendix:1} به‌طور کامل بررسی شده است. بخش عمده این ساختار را اشیای متنی تشکیل می‌دهند. از مدل‌های معرفی شده در بخش \ref{sec:model} برای یادگیری ساختار اشیای متنی استفاده می‌کنیم؛ زیرا یادگیری فایل  \gls{PDF} به‌صورت یکجا با توجه به اینکه قسمت‌هایی از آن برای آدرس‌دهی استفاده می‌شوند امکان‌پذیر نیست. اما \lr{MuPDF} یا هر \gls{PDF}خوان دیگری فایل‌های کامل \gls{PDF} را به عنوان ورودی می‌پذیرد به همین لازم است تا اشیای تولید شده را به یک فایل تبدیل کنیم. برای این کار اشیای جدید را با استفاده از سازوکار توضیح داده شده در پیوست آ در مورد بروزرسانی یک فایل \gls{PDF}، به یک فایل معتبر از پیش موجود، که آن را  \gls{Host} می‌نامیم، اضافه می‌کنیم. در نتیجه یک فایل \gls{PDF} جدید در اختیار خواهیم داشت که ساختار آن معتبر بوده و برخی اشیای آن تغییر داده شده و بازنویسی شده‌اند.
 
 
 
 \section{معیارهای ارزیابی}
 در تولید خودکار داده آزمون به روش یادگیری ژرف، تعداد زیادی پارامتر وجود دارد که می‌توان تأثیر آنها را در فرایند آزمون فازی سنجید. اما مهم‌ترین هدف در فرایند آزمون فازی همان‌گونه که پیش از این هم ذکر شد یافتن خطا در \gls{SUT} است. هدف مهم بعدی افزایش میزان پوشش کد برای نیل هر چه بیشتر به هدف اول است. هدف دیگر که مختص به این روش است یادگیری هر چه بهتر ساختار فایل ورودی است. براساس اهداف اشاره شده، معیارهای ارزیابی زیر را در این فصل لحاظ کرده‌ایم:
 \begin{itemize}
 	\item {
 		\textbf{خطا و دقت مدل‌های یادگیری ژرف.}
 	 شامل خطا و دقت هر مدل‌ روی مجموعه‌های آموزش و ارزیابی، میزان و نحوه تغییر آنها با گذشت زمان. این معیارها توسط کد نوشته شده به ابزار \lr{Tensorboard} گزارش شده و در حین فرایند آموزش یا در پایان آن قابل مشاهده هستند.
 
}
	\item {
	\textbf{سرگشتگی.}
	برای ارزیابی میزان خوب بودن مدل زبانی در پیش‌بینی نشانه بعدی توالی داده شده. معیار سرگشتگی از رابطه \ref{ppl} و با جایگذاری خطای مدل روی مجموعه ارزیابی، برای آن مدل محاسبه می‌گردد.
	
}

\item {
	\textbf{پوشش کد.}
	در هر اجرای \gls{SUT}، تعداد خطوط برنامه و نیز تعداد بلوک‌های اولیه‌ای که اجرا شده‌اند، شمارش می‌شود. با داشتن کل خطوط و بلوک‌های اولیه درصد پوشش کد نیز قابل محاسبه است. برای یک مجموعه آزمون، یعنی چندین بار اجرای متوالی \gls{SUT} پوشش کد عبارت خواهد بود از اجتماع پوشش کد تک‌تک هریک از اجراها. بنابراین منظور از اصطلاح کلی پوشش کد در ادامه پوشش بلوک‌های پایه است.  
	
}
\item {
\textbf{خطا و آسیب‌پذیری.}
\gls{SUT}
تحت ابزار پایش \lr{Application Verifier} اجرا می‌شود که خطاهای احتمالی را ثبت و شماره‌گذاری می‌کند. سپس می‌توان با تحلیل این خطاها در یک محیط اشکال زدا، ضمن تعیین مکان و علت خطا، امکان قابل بهره‌برداری بودن آنها را بررسی کرد.
}
 \end{itemize}

آزمایش‌ها بایستی به‌گونه‌ای طراحی شوند که هریک از آنها تأثیر یک پارامتر خاص را بر روی معیارهای بالا، به‌طور واضح نمایان سازد؛ اما، پارامـــترها کدامند؟ ما در این بخش، چندین پارامتر و مقادیر قابل آزمایش برای هریک را شناسایی و معرفی می‌کنیم. فهرستی از این پارامترها و مقادیری که می‌توانند بپذیرند در جدول \ref{tabel:all_parameters} ذکر شده است. همان‌طور که مشاهده می‌شود برخی از پارامترها مقادیری پیوسته و برخی دیگر مقادیری گسسته می‌پذیرند. بر این باور هستیم که هر روش تولید خودکار داده آزمون با استفاده از یادگیری ماشینی روی چنین پارامترهایی بنا می‌شود، لذا شناسایی و تفکیک این پارامترها پیش از هرگونه طرح آزمایشی لازم و ارزشمند است.

\begin{table}%[ht]
	\caption[پارامترهای مؤثر در تولید خودکار داده آزمون با روش‌های یادگیری ماشینی و مقادیر قابل آزمایش برای آنها]{
		پارامترهای مؤثر در تولید خودکار داده آزمون با روش‌های یادگیری ماشینی و مقادیر قابل آزمایش برای آنها.}
	\label{tabel:all_parameters}
	\centering
	\onehalfspacing
	%\setlength{\extrarowheight}{4mm}
	\begin{tabularx}{0.95\linewidth}{rrr}
		\toprule[1.5pt]
		 ردیف  &
		 پـارامتـر \hspace{2cm} &
		 مقادیر قابل آزمایش 
		\\
		\midrule[1.5pt] 
		1 &
		 نوع مدل & 
		$\bullet$
		 انواع مدل‌های یادگیری ژرف 
		\\
		\hline
		 2 &
		 راهبرد & 
		 $\bullet$
		 حریصانه 
		 \\
		  &
		  &  
		  $\bullet$
		  نمونه برداری
		 \\
		  &
		  &  
		  $\bullet$
		  ترکیبی (حریصانه + نمونه‌برداری) 
		 \\
		  &
		  &  
		  $\bullet$
		  چند فهرست بهتر 
		 \\
		\hline
		%\vspace{3mm}
		 3 &  تنوع 
		 ($D$) & 
		 $\bullet$ $(0, \infty)$
		\\
		\hline
		%\vspace{3mm}
		 4 &
		 روش فاز &
		 $\bullet$
		 بدون فاز 
		 \\
		  &
		  &
		 $\bullet$
		 تصـادفی
		 \\
		  &
		 &
		 $\bullet$
		 مبتنی بر مدل 
		 \\
		  &
		  &
		  $\bullet$
		 ترکـیبی
		 \\
		\hline
		%\vspace{3mm}
		5 & 
		الگوریتم فاز & 
		$\bullet$
		فاز داده
		\\
		&
		&
		$\bullet$
		فاز فراداده 
		\\
		\hline
		%\vspace{3mm}
		 6 &  تعداد دوره آموزش & 
		 $\bullet$ $\{1,2,3,...\}$
		\\
		\hline
		%\vspace{3mm}
		7 &
		نوع فایل &
		$\bullet$
		متنی
		\\
		&
		&
		$\bullet$
		متنی + دودویی
		\\
		\hline
		%\vspace{3mm}
		8 &
	
		استفاده از میزبان &
		$\bullet$
		استفاده
		\\
		&
		&
		$\bullet$
		عدم استفاده 
		\\
		\hline
		%\vspace{3mm}
		 9 &  تعداد میزبان & 
		 $\bullet$ $\{1,2,3,...\}$
		\\
		\hline
		10 &
		میزان تغییر میزبان &
		$\bullet$
		ثابت
		\\
		&
		&
		$\bullet$
		متغیر
		\\
		\bottomrule[1.5pt]	
	\end{tabularx} 
\end{table}
 
 نوع مدل در سطر اول جدول \ref{tabel:all_parameters} برای هر مدل، خود دارای اَبَرپارمترهایی مانند تعداد عصب، لایه، نرخ یادگیری و غیره است که در فصل قبل به آنها اشاره شد. بنابراین جدول \ref{tabel:all_parameters} یک دید جامع از همه پارامترهایی که تا اینجا معرفی شدند، در اختیار می‌گذارد. در بخش بعد چیدمان آزمایش‌ها خود را براساس این پارامترها شرح خواهیم داد.
 
 

 \section{چیدمان آزمایش‌ها}
 
  برای آموزش مدل‌های یادگیری معرفی شده در بخش \ref{sec:model} (جدول \ref{tabel:deep_model}) و تولید داده‌های آزمون از یک سیستم فیزیکی با پردازنده گرافیکی 
  \lr{NVidia GTX 1080}،
  پردازنده مرکزی 
  \lr{Intel Core i7 }
  و 20 گیگابایت حافظه اصلی به همراه سیستم عامل 
  \lr{Ubuntu 16.04 x64}
  استفاده کردیم. آزمون‌های فازی را نیز روی ماشین مجازی با پردازنده 
  \lr{Intel Core i7 } 
  و 8 گیگابایت حافظه اصلی به همراه سیستم عامل 
 \lr{Windows 10 x64}
انجام دادیم. ما همچنین از نسخه نهایی 
\lr{MuPDF}
در زمان انجام آزمایش‌ها یعنی نسخه 
\lr{MuPDF 1.11(2017-04-11) Final}
برای آزمون فازی استفاده کردیم. باتوجه به نهایی بودن نسخه انتظار می‌رود خطاهای نسخه تا حد زیادی، در مقایسه با نسخه‌های آلفا، بتا و 
\lr{RC}
برطرف شده باشند و لذا شناسایی خطا کار سخت‌تری خواهد بود.
مشخصه‌های کامل مدل‌ها در فرایند آموزش به‌همراه تعداد دوره و زمان آموزش هر دوره، در جدول \ref{tabel:model_parameters} آمده است.

\begin{table}%[ht]
	\caption{
		مشخصه‌ها و زمان آموزش مدل‌های جدول \ref{tabel:deep_model} در فرایند آموزش.}
	\label{tabel:model_parameters}
	\centering
	\onehalfspacing
	\begin{tabularx}{0.95\linewidth}{r p{15mm} p{15mm} p{15mm} p{15mm}}
		
		\toprule[1.5pt]  پارامتر / شماره مدل & 1 & 2 & 3 &  4
		\\
		\midrule[1.5pt] طول توالی‌های ورودی ($d$) & 50 & 50 &  50 & 50 
		\\
		%\hline
		 گام پرش ($j$) & 3 & 3 &  1 & 1 
		\\
		%\hline
		 تعداد دوره & 50 & 50 &  50 & 50 
		\\
		%\hline
		 زمان تقریبی یک دوره (ساعت: دقیقه) &   $  1:00'$  &   $1:45'$  &  $5:45'$&           $  9:30'$ 
		\\
		%\hline
		 حجم تقریبی مدل (مگابایت) & $ 1.24$ & $ 2.76$ &  $ 9.99$ & $ 5.41$ 
		\\
		\bottomrule[1.5pt]
		
	\end{tabularx} 
\end{table}
 
 پیچیدگی مدل‌های جدول \ref{tabel:model_parameters} با افزایش شماره، افزایش می‌یابد. یعنی مدل 2 از مدل 1 پیچیده‌تر درنظر گرفته شده است. همان‌طور که در جدول هم مشخص است مدل‌های پیچیده‌تر زمان آموزش طولانی‌تری دارند. برای مدل‌های پیچیده‌تر منطقی است که نمونه‌های آموزشی بیشتری داشته باشیم. در نتیجه برای مدل‌های 3 و 4 گام پرش را 1 و برای مدل‌های 1 و 2 گام پرش را 3 در نظر گرفتیم. همچنین برای مدل شماره 3 از روش منظم‌سازی
 \lr{Dropout} \cite{JMLR:v15:srivastava14a}
   $p=0.3$ 
  استفاده کردیم. در نهایت هر مدل را دست‌کم برای $50$ دوره روی داده‌های مجموعه آموزش، آموزش دادیم.
 
 \subsection{مجموعه داده}
 آموزش موفق شبکه‌های عصبی ژرف مستلزم داشتن مجموعه داده به‌اندازه کافی بزرگ و مناسب است. در یادگیری ماشینی هرچه‌قدر تعداد داده‌ها بیشتر باشد، یادگیری بهتر انجام خواهد شد. هنوز در بسیاری از زمینه‌ها مجموعه داده کافی برای آموزش وجود ندارد و این یکی از محدودیت‌های استفاده از شبکه‌های عصبی ژرف است. برای یادگیری آماری ساختار اشیای فایل \gls{PDF} پیکره بزرگی از فایل‌های \gls{PDF} جمع‌آوری کردیم؛ زیرا چنین پیکره‌ای از قبل وجود نداشت. بخشی از این پیکره شامل مجموعه داده‌های آزمون \gls{PDF} خوان
  \lr{Mozilla}\LTRfootnote{\href{https://github.com/mozilla/pdf.js/tree/master/test/pdfs}{https://github.com/mozilla/pdf.js/tree/master/test/pdfs}} 
   است که در مرورگر وب
    \lr{Firfox}
     و دیگر پروژه‌های این شرکت استفاده می‌شود و به‌صورت رایگان دردسترس است. بخش دیگری از آن \gls{PDF}های استفاده شده در فازرهای دیگر مثل
      \lr{AFL}
       است و بلأخره بخشی بزرگی نیز از طریق وب جمع‌آوری شد به‌نحوی که تنوع خوبی از لحاظ اندازه و محتوا داشته باشند. پیکره نهایی در حال حاضر شامل 6100 فایل \gls{PDF} مختلف و طبقه‌بندی شده است که ما آن را تحت نام 
       \lr{IUST PDF Corpus}\LTRfootnote{\href{https://github.com/m-zakeri/iust\_deep\_fuzz/tree/master/dataset}{https://github.com/m-zakeri/iust\_deep\_fuzz/tree/master/dataset}}
       منتشر کرده‌ایم و از آن می‌توان در موارد دیگر نیز استفاده کرد.
 
   
 از فایل‌های داخل این پیکره که اندازه آنها بین 1 تا 900 کیلوبایت و تا حداکثر 7935 کیلوبایت متغیر است، در حدود 504153 تعداد شیء داده‌ای \gls{PDF}  با طول‌های گوناگون استخراج شد که به عنوان مجموعه داده جهت یادگیری ساختار اشیای داده‌ای \gls{PDF} در نظر گرفته شده‌اند. این اشیاء سپس همگی در یک فایل الحاق شدند که حجم فایل نهایی در حدود 70 مگابایت است. از این تعداد در حدود 137157 شیء حاوی جریان‌های دودویی بودند که محتوای همه این جریان‌ها پس از شناسایی با توکن دودویی
   \texttt{\lr{stream}} 
 جایگزین شدند؛ زیرا، همان‌طور که قبل از این هم اشاره کردیم نیازی به شرکت دادن آنها در فرایند آموزش نیست و بعداً مجدداً آنها را اضافه ‌می‌کنیم.
 
 
 \subsection{پیش‌پردازش داد‌ه‌ها}

 تعداد اشیای استخراج شده بسیار زیاد هستند. قبل از انجام عملیات یادگیری مدل بایستی تعدادی از این اشیا را به نحوی از مجموعه داده حذف کرد. برای این منظور عملیات پیش‌پردازشی طی مراحل مختلف روی مجموعه داده انجام دادیم. مهمترین ویژگی در دسترس در تنظیم ابرپارامترهای مدل یادگیری ژرف در اینجا طول اشیای داده‌ای  \gls{PDF} است که هم در فرایند آموزش و هم در فرایند تولید نقش تأثیرگذاری دارد. لذا داشتن توزیعی همگن از اشیای  \gls{PDF} برحسب طول آنها فرضیه‌ای است که در اینجا در نظر گرفته شده است و در عملیات پیش‌پردازش نیز معیار طول اشیای داده‌ای بوده است. ویژگی طول اشیاء از این منظر مهم است که شبکه  \gls{LSTM} بر روی توالی‌های بسیار طولانی عملکرد ضعیفی دارد و بهتر است چنین توالی‌هایی را در نظر نگیریم. مراحل پیش‌پردازش داده‌ها به‌ترتیب در زیر ذکر شده است.
 
 \begin{enumerate}
 	%%1
 	\item {
 هر شیء داده‌ای یک موجودیت مستقل درنظر گرفته شده و ابتدا مجموعه داده برحسب طول اشیای داخل آن مرتب گردید. سپس کلیه اشیاء داده‌ای به‌شکل 
 \begin{LTR}
 	 \texttt{obj\\/\\endobj}
 \end{LTR}
و
 \begin{LTR}
	\texttt{obj\\null\\endobj}
\end{LTR}
از مجموعه داده اولیه حذف شدند؛ زیرا، حاوی بار اطلاعاتی مفیدی نیستند.
 
}

%%2
\item {
	پس از اعمال مرحله 1 تعداد اشیای مجموعه داده به 494979 کاهش یافت. اکنون به‌طور تقریبی هر صدک از مجموعه داده شامل حدود 4950 شیء داده‌ای  \gls{PDF} است. برای حذف هرچه بیشتر داده‌های پرت صدک‌های اول و آخر را نیز از مجموعه داده حذف کردیم. به‌این ترتیب در پایان این مرحله تعداد 485080  شیء داده‌ای  \gls{PDF} باقی‌ ماند.
	
}

%%3
\item {
	مرحله اصلی پیش‌پردازش خارج کردن داده‌های
	 \lr{outlier} و 
	 \lr{extreme value}
	  با استفاده از معیار $iqr$ توسط ابزار
	\lr{WEKA}\LTRfootnote{\href{https://www.cs.waikato.ac.nz/ml/weka/}{https://www.cs.waikato.ac.nz/ml/weka/}} \cite{Hall:2009:WDM:1656274.1656278}
	است. ورودی این مرحله مجموعه داده مرحله 2 است و خروجی آن حاوی تعداد 477104 شی‌ داده‌ای \gls{PDF} است. برای تعیین دو مقدار نامبرده یعنی
	 \lr{outlier} و 
	 \lr{extreme value}
	 به ترتیب از ضرایب 3 برای ضریب \lr{outlier} و 6 برای ضریب
	 \lr{extreme value}
	  استفاده شد؛ که مقادیر پیش‌فرض 
	\lr{WEKA} 
	 برای \gls{Filter}
	  $iqr$ 
	   هستند.	
}

%%4
\item{
	در این مرحله هدف تقسیم مجموعه داده به سه بخش مجموعه آموزش ، مجموعه ارزیابی  و مجموعه آزمون  است. از آنجایی که یادگیری مدل ما در حالت بدون ‌نظارت انجام می‌شود\footnote{برچسب‌ها از خود مجموعه آموزش برداشته می‌شوند. }، از مجموعه آزمون فقط برای انتخاب پیشوند جهت تولید اشیای جدید استفاده خواهد شد. به‌عبارت دیگر مدل روی مجموعه آموزش، آموزش داده می‌شود و در هنگام تولید اشیای جدید \gls{PDF} مقادیر اولیه از مجموعه آزمون برداشته می‌شود. برای این منظور $25$ درصد از کل تعداد اشیای \gls{PDF} را برای مجموعه آزمون و $75$ درصد را برای مجموعه آموزش/ ارزیابی در نظر گرفتیم. برای ایجاد این تفکیک نیز از فیلتر 
	\lr{StratifiedRemoveFolds} 
	موجود در ابزار
	\lr{WEKA} 
	با دانه اولیه 101  (کوچک‌ترین عدد اول سه رقمی) و مقدار $ fold=3 $ (یعنی بعد از \gls{Shuffle} و تقسیم کل مجموعه داده به 4 قسمت با نسبت‌هایی برابر با مجموعه داده اولیه، قسمت سوم آن برای مجموعه آزمون کنار گذاشته شود) استفاده شد. در نهایت تعداد 357828 شی در مجموعه‌های آموزش/ ارزیابی و تعداد 119276 شی داده‌ای هم در مجموعه آزمون قرار گرفتند. 

}

%%5
\item {
	در تکمیل مرحله 4، نیاز است تا مجموعه آموزش/ ارزیابی جدا شده، حاوی 357828 شیء خود به دو مجموعه مجزای آموزش و ارزیابی تقسیم شود. برای این منظور این مجموعه آموزش/ ارزیابی به 4 قسمت تقسیم گردید و قسمت چهارم به عنوان مجموعه ارزیابی در نظر گرفته شد. سه قسمت اول هم مجموعه آموزش را تشکیل می‌دهند. درصد‌ تعداد اشیاء هریک از مجموعه‌های آموزش، ارزیابی و آزمون در جدول \ref{tabel:dataset_divding_percentage} آمده است.

}

%%6
\item {
	آخرین مرحله پیش‌پردازش \gls{Normalization} متن است. در این مرحله برخی کاراکترهای بسیار کم تکرار (فراوانی نسبی نزدیک صفر) حذف یا با کاراکترهای مناسبی جایگزین شدند. در واقع طی این مرحله اندازه مجموعه واژگان کاراکترها از 96 به 64 تقلیل یافت. علت انجام این مرحله، آن است که کاراکترهای با تعداد کم داده پرت محسوب می‌شوند و به‌دلیل اینکه در مجموعه آموزش تعداد تکرار آنها پایین است عملاً احتمال بسیار ناچیزی به آنها تخصیص داده می‌شود که در واقع ضرورتی به این کار نیست. در عین حال با انجام این کار، حجم مکعب داده‌ها کمتر شده و فرایند آموزش سریع‌تر انجام می‌شود.\\
	
}

 \end{enumerate}
 
 \begin{table}%[ht]
 	\caption[نحوه و درصد‌های تقسیم مجموعه داده به مجموعه‌های آموزش، ارزیابی و آزمون]
 	{
 		نحوه و درصد‌های تقسیم مجموعه داده به مجموعه‌های آموزش، ارزیابی و آزمون.
 	}
 	\label{tabel:dataset_divding_percentage}
 	\centering
 	\onehalfspacing
 	\begin{tabularx}{0.5\linewidth}{p{20mm} p{20mm} p{20mm}}
 		
 		\toprule[1.5pt]  مجموعه & تعداد شیء & درصد
 		\\
 		\midrule[1.5pt] آموزش & 268371 & $56.25  $
 		\\
 		%\hline
 		 ارزیابی & 89457 & $ 18.75 $
 		\\
 		%\hline
 		 آزمون & 119276 & $ 25$ 
 		\\
 		\hline
 		 جمع & 477104 & $ 100$ 
 		\\
 		\bottomrule[1.5pt]
 		
 	\end{tabularx} 
 \end{table}
 
 
 
 \subsection{انتخاب فایل‌های میزبان}
 در 
 \cite{Godefroid:2017:LML:3155562.3155573}
 انتخاب فایل‌های میزبان به‌صورت انتخاب سه فایل با کمترین حجم از پیکره انجام گرفته است. در واقع این انتخاب بدون هدف خاص و کاملاً تصادفی صورت ‌گرفته است. پیرو روش آنها ما نیز برای آزمایش‌های خود سه فایل را به‌عنوان فایل‌های میزبان انتخاب کردیم، با این تفاوت که برای انتخاب فایل‌های میزبان ابتدا میزان پوشش کد را به‌صورت مجزا برای همه فایل‌های 
 \lr{IUST PDF Corpus}
 به‌دست آورده و سپس سه فایل با بیشترین میزان پوشش کد، کمترین میزان پوشش کد و نیز میزان پوشش کد میانگین، به‌ترتیب با نام‌های \lr{host1\_max}، \lr{host2\_min} و \lr{host3\_avg} به‌عنوان فایل‌های \gls{Host}انتخاب شدند. هدف از این کار بررسی تأثیر روش پیشنهادی بر بهبود میزان پوشش کد در فایل‌های با اختلاف پوشش کد زیاد است و سپس انجام آزمون فازی براساس بهترین میزبان است. ما همچنین برای هر آزمایش پوشش کد مجموع هر سه میزبان را با ادغام پوشش کد‌های هریک از آنها تحت عنوان 
\lr{host123}
گزارش کردیم. این کار کمک می‌کند تا بتوانیم نتایج دو آزمایش کاملاً جدا را در حالت کلی با یدکدیگر مقایسه کنیم.
 
 
 
 
 \subsection{پوشش کد مبنا}
 برای آنکه بتوانیم درک معنادار و مقایسه‌پذیری از پوشش کد‌های به‌دست آمده در هر آزمایش داشته باشیم. ابتدا پوشش کد مبنا را مطرح، اندازه‌گیری و گزارش می‌کنیم. پوشش کد مبنا، پوشش کد در غیاب استفاده از روش پیشنهادی برای تولید خودکار داده آزمون است. برای این منظور، ابتدا از مجموعه آزمون، تعداد 1000 شیء داده‌ای \gls{PDF} استخراج و در دو حالت به فایل‌های میزبان تزریق می‌نماییم:
  
  \begin{itemize}
  	\item {
  	
   \textbf{بـروزرسانی یک شیء }(\gls{SOU}):
برای هر فایل میزبان در هر مرحله یک شیء را انتخاب و آخرین شیء میزبان را با آن بازنویسی می‌کنیم، یعنی در واقع تغییر تنها یک شیء از فایل میزبان در هر مرحله بدون توجه به اندازه و تعداد اشیای فایل میزبان.
}
  	
  	\item {
  	\textbf{بـروزرسانی چندین شیء }(\gls{MOU}):
  تغییر درصد ثابتی از اشیای فایل میزبان در هر مرحله. در این روش ما برای \lr{host1\_max} تعداد 
  $\frac{1}{5}$، برای \lr{host2\_min} تعداد 
   $\frac{1}{3}$
  و برای \lr{host3\_avg} تعداد 
   $\frac{1}{4}$
  اشیای هر فایل را به صورت تصادفی انتخاب و با اشیای انتخابی از مجموعه آزمون بازنویسی کردیم. یعنی در این روش هر میزبان را به نسبت‌ بیشتری تغییر می‌دهیم، به‌امید اینکه این تغییرات سبب افزایش پوشش کد شود. می‌توان اشیاء را باروش‌هایی غیر از روش تصادفی نیز برای بازنویسی انتخاب کرد؛ به‌عنوان مثال انتخاب براساس ترتیب نزولی یا صعودی \lr{ID} هر شیء یا براساس بزرگی آدرس آنها. در این پایان‌نامه ما فقط روش تصادفی را آزمایش کردیم.
  	
  	}
  	
  \end{itemize}
  
   به این ترتیب سه مجموعه 1000 فایلی از حالت اول و سه مجموعه 1000 فایلی از حالت دوم حاصل  می‌گردد که آنها را به ترتیب 
   \lr{baseline\_sou}\LTRfootnote{\lr{Baseline for Single Object Update}}
   و
   \lr{baseline\_mou}\LTRfootnote{\lr{Baseline for Multiple Object Update}} 
   می‌نامیم. پوشش کد هر مجموعه و نیز پوشش کد اجتماع هر سه میزبان به عنوان یک مقدار پایه برای آزمایش‌های بعدی اندازه‌گیری و تعیین شدند. نمودار شکل \ref{ch5_baseline_code_coverage_crop.pdf} مقادیر پوشش کد مبنا را در مقایسه با پوشش کد هریک از میزبان‌ها به‌صورت تنها نشان می‌دهد. لازم به ذکر است که هر یک از میزبان‌ها پیش از بروزرسانی‌های افزایشی برای تولید داده جدید بررسی شدند و معتبر بودن ساختار آنها تأیید گردید. لذا، صرف انجام این کار به منزله آزمون فازی نخواهد بود؛ زیرا، هنوز داده‌ها را بدشکل (فاز) نکرده‌ایم. هرچند ممکن است در این فرایند نیز خطایی قابل شناسایی باشد.
   
 
 
 \begin{figure}%[ht]%[tbh!]%%[t!]
 	\centering
 	\includegraphics[width=\textwidth, clip=true,  trim= 0 0 0 0]{chapter5/ch5_baseline_code_coverage_crop.pdf}
 	\caption[پوشش کد برای هریک از میزبان‌ها و پوشش کدهای مبنا]
 	{
 		پوشش کد برای هریک از میزبان‌ها و پوشش کدهای مبنا.
 	}
 	\label{ch5_baseline_code_coverage_crop.pdf}
 	%\ref{ch5_baseline_code_coverage_crop.pdf}
 \end{figure}
 
 
 \subsubsection{مشاهدات}
 
 \begin{itemize}
 	\item{
 	میزان پوشش کد هریک از \lr{baseline}ها از پوشش کد فایل میزبان به تنهایی بیشتر است. یعنی تغییر میزبان‌ها منجر به اجرای بلوک‌های پایه جدید و احتمالاً مسیرهای اجرایی جدیدی، شده است.
 
}
\item {
میزان پوشش \lr{baseline}ها متناسب با میزبان‌ها است. یعنی به‌عنوان مثال \lr{host1\_max} بیشترین پوشش کد را در هر سه حالت تنها، \lr{baseline\_sou} و \lr{baseline\_mou} داشته است. این بدین معنی است که انتخاب فایل میزبان بسیار حائز اهمیت است و بخش قابل توجهی از پوشش کد در هر حالت مربوط به فایل میزبان است.

}

\item {
	میزان پوشش کد \lr{baseline\_mou} در تمامی میزبان‌ها از \lr{baseline\_sou} بیشتر است. این بدین معنی است که با افزایش میزان تغییر فایل شاهد افزایش میزان پوشش کد بوده‌ایم.
	
}

\item {
	 بیشترین پوشش کد مربوط به اجتماع پوشش کدهای \lr{baseline\_mou}  در \lr{host123} است که نشان می‌دهد هر کدام از میزبان‌ها مجموعه دستورات متفاوتی را اجرا کرده‌اند.
	
}

\item {
	درنهایت اعداد و ارقامِ پوشش کد در محدوده 25هزار بلوک پایه بوده که نشان می‌دهد \lr{MuPDF} نرم‌افزاری پیچیده است.
}

 \end{itemize}
 
 
 
 
 
 
\section{آزمایش‌ها، یافته‌ها و مقایسه نتایج} 
 در این بخش آزمایش‌های  صورت گرفته و نتایج حاصل از آن را بیان می‌کنیم. همچنین نتایج را با سایر کارهای مرتبط مقایسه خواهیم کرد. 
 چون \gls{SUT} مورد استفاده در یادگیری و فاز 
 \cite{Godefroid:2017:LML:3155562.3155573}
 برای آزمون دردسترس نبود، ما روش یادگیری و فاز را نیز پیاده‌سازی و آن را روی ابزار \lr{MuPDF} آزمایش کردیم. به‌این ترتیب توانستیم مقایسه معناداری بین روش پیشنهادی خود با این روش، به‌عنوان مرتبط‌ترین کار انجام شده داشته باشیم. نتایج نشان می‌دهد که روش پیشنهادی ما در معیار پوشش کد و همچنین در معیارهای خطا، دقت و سرگشتگی بهبود داشته است و توانسته روش یادگیری و فاز، که در آزمایش‌ها با نام \lr{laf} آن را نشان می‌دهیم، را شکست دهد. روش پیشنهادی ما همچنین نسبت به \lr{AFL} و \lr{AFL}افزوده بهبودهایی داشته است که خواهیم دید.
 
 چون مجموعه داده در مدل‌های یادگیری ژرف بسیار مهم است و روی نتایج تأثیر می‌گذارد در هنگام آموزش مدل \lr{laf} مجموعه داده را مطابق مقیاس ارائه شده در 
 \cite{Godefroid:2017:LML:3155562.3155573}
 لحاظ کردیم. بدین زمان آموزش کمتر برای هر دوره در حدود 15 دقیقه به طول انجامید. سایر پارامترها مانند طول توالی‌های آموزشی $d$ را که در مقاله به آن اشاره نشده بود نیز مشابه مقادیر آن در روش پیشنهادی قرار دادیم.
 
 
 \subsection{سرگشتگی، خطا و دقت مدل‌ها}
 
 جدول \ref{tabel:ppl_and_accuracy} سرگشتی، خطا و دقت مدل‌های چهارگانه پیشنهادی و مدل یادگیری و فاز را بعد از گذشت 50 دوره، روی مجموعه‌های آموزش و ارزیابی نشان می‌دهد. سرگشتی مطابق رابطه \ref{ppl} و خطا نیز از رابطه \ref{CrossEntropyLossFunction} برای مدل‌ها محاسبه شده است که در فصل \ref{chapter2} آنها را بیان کردیم. دقت نیز معیاری است که توسط \lr{Keras} برای هر مدل حساب می‌شود. همچنین  شکل \ref{ch5_loss_model2_vs_model_laf_crop.pdf} روند تغییرات خطا در فرایند آموزش را برای مدل 2 و مدل \lr{laf}  نشان می‌دهد. مدل 2 در این نمودار به‌این دلیل انتخاب گردیده که بیشترین میزان شباهت را از لحاظ معماری و مقادیر اَبرپارامترها با مدل \lr{laf} دارد. 
 
 
  \begin{table}%[ht]
 	\caption[سرگشتگی، دقت و خطای مدل‌های مختلف روی مجموعه‌های آموزش و ارزیابی]
 	{
 		سرگشتگی، دقت و خطای مدل‌های مختلف روی مجموعه‌های آموزش و ارزیابی.
 	}
 	\label{tabel:ppl_and_accuracy}
 	\centering
 	\onehalfspacing
 	\begin{tabularx}{0.95\linewidth}{r p{20mm} p{20mm} p{20mm} p{20mm} p{15mm}}
 		\toprule[1.5pt] 
 		معیار / مدل &
 		1 &
 		2 &
 		3 &
 		4 & 
 		\lr{laf}  
 		\\
 		\midrule[1.5pt] 
 		 سـرگــشتـگی &
 		$1.440$ &
 		$1.391$ &
 		$\boldsymbol{1.335}$ &
 		$1.350$ &
 		$1.860$
 		\\
 		%\hline 
 		بیشینه دقت آموزش &
 		$0.886$ &
 		$0.902$ &
 		$0.893$ &
 		$0.909$ &
 		$0.820$ 
 		\\
 		%\hline
 		بیشینه دقت ارزیابی &
 		$0.884$ &
 		$0.895$ &
 		$0.904$ &
 		$0.905$ &
 		$0.800$ 
 		\\
 		%\hline
 		کمینه خطای آموزش &
 		$0.353$ &
 		$0.298$ &
 		$0.324$ &
 		$0.276$ &
 		$0.623$ 
 		\\
 		%\hline
 		کمینه خطای ارزیابی &
 		$0.365$ &
 		$0.330$ &
 		$0.289$ &
 		$0.300$ &
 		$0.724$
 		\\
 		\bottomrule[1.5pt]
 		
 	\end{tabularx} 
 \end{table}
%\vspace{50mm}
 
  \begin{figure}%[ht]%[ht]%[tbh!]%%[t!]
 	\centering
 	\includegraphics[width=\textwidth, clip=true,  trim= 0 0 0 0]{chapter5/ch5_loss_model2_vs_model_laf_crop.pdf}
 	\caption[نمودار تغییرات خطای مدل‌های 2 و \lr{laf}]
 	{
 	نمودار تغییرات خطای مدل‌های 2 و \lr{laf} در دوره‌های 1 تا 50.
 	}
 	\label{ch5_loss_model2_vs_model_laf_crop.pdf}
 	%\ref{ch5_loss_model2_vs_model_laf_crop.pdf}
 \end{figure}

\subsubsection{مشاهدات}

 \begin{itemize}
 	\item{
 سرگشتگی و خطای همه مدل‌های پیشنهادی از مدل یادگیری و فاز کمتر و دقت آنها از یادگیری و فاز بیشتر است. یعنی مدل زبانی عصبی با معماری توضیح داده شده در بخش \ref{sec:model}، در یادگیری ساختار فایل از مدل کدگذار-کدگشا 
 \cite{Godefroid:2017:LML:3155562.3155573}
 بهتر عمل کرده است.  	
 }
\item
{
	بیشترین دقت در بین تمامی مدل‌ها مربوط به مدل 4 است. مدل 4 \gls{LSTM} دوسویه است. شبکه \gls{LSTM} دوسویه هنگام آموزش علاوه‌بر کاراکترهای قبلی، کاراکترهای بعدی را نیز در نظر می‌گیرد. بنابراین توانسته‌ است بـه ‌دقت بیشتر دست یابد. کمترین میزان سرگشتگی نیز متعلق به مدل 3 است. این مدل همان‌طور که در جدول نیز مشخص است دارای کمترین خطای ارزیابی است. البته اختلاف بین مقادیر مختلف در همه مدل‌های پیشنهادی کم است که نشان می‌دهد همه مدل‌ها قادر به یادگیری و درک ساختار فایل بوده‌اند. 
	سرگشتگی بیشینه، حالت بدون استفاده از مدل یادگیری،  برای تنظیمات ما برابر 64 (اندازه بردار واژگان) خواهد بود.
}

\item
{
	در شکل \ref{ch5_loss_model2_vs_model_laf_crop.pdf}، منحنی خطای مدل 2 در همه دوره‌ها، از منحنی خطای مدل \lr{laf} پایین‌تر است. البته دوره‌ها زمان متفاوتی دارند بنابراین مقایسه نظیر‌به‌نظیر آنها ممکن است جالب به‌نظر نیاید. با این‌حال در بازه زمانی مساوی از شروع فرایند آموزش نیز این وضعیت فوق برقرار است. برای مثال پایان دوره‌ 1 برای مدل 2 مصادف با پایان دوره 12 برای مدل \lr{laf} خواهد بود که باز هم خطای مدل \lr{laf}  بیشتر است. ما فرایند آموزش را برای هر دو مدل تا دوره 100 ادامه دادیم و تغییر خلافی مشاهده نشد. چنان‌چه در بخش \ref{sec:epochcompare} خواهیم دید پوشش کد مدل 2 نیز در دوره‌های مختلف از مدل \lr{laf} بالاتر است.
}

 \end{itemize}
 
 \subsection{پوشش کد مدل‌های مولد}\label{sec:gen_model_cove}
 
 در آزمایش این بخش، مدل‌های مولد پیشنهادی  خود را برای اولین مرحله تولید داده آزمون محک می‌زنیم. برای تولید داده‌های آزمون جدید از روش نمونه‌برداری و پارامتر تنوع $D$ در همه آزمایش‌ها استفاده کردیم. همچنین هربار با یک پیشوند تصادفی از مجموعه آزمون تولید داده از مدل را شروع می‌کنیم. می‌خواهیم ببینیم برای میزبان‌های مختلف کدام مدل مولد و نیز کدام تنوع تولید داده، به پوشش کد بالاتری دست می‌یابد. هدف مقایسه کارکرد مدل‌ها و انتخاب مناسب‌ترین مدل برای استفاده در آزمون فازی است.
 برای این منظور هریک از مدل‌های جدول \ref{tabel:deep_model} را به تعداد 50 دوره آموزش دادیم. در پایان هر دوره یک نمونه از مدل آموزش دیده را ذخیره کردیم. سپس مدل با کمترین خطا را از بین نمونه‌های ذخیره شده، برای تولید داده برگزیدیم.
 
 
 در مرحله بعد برای هر مدل در هنگام تولید داده با هریک از تنوع‌های تولید $0.5$، $1$ و $1.5$، تعداد 1000 فایل \gls{PDF} با هر میزبان تولید کرده‌ایم. یعنی در مجموع با هر مدل و هر میزبان 3000 فایل را تولید می‌کنیم. چون 4 مدل و 3 میزبان داریم، درکل 36000 فایل تولید می‌شود و چون دو روش بروزرسانی شیء
  \gls{SOU} 
  و 
  \gls{MOU} هم نیاز است در نهایت 72000 فایل تولید می‌کنیم. دقت شود که در این مرحله هنوز عملیات آزمون فازی یعنی فاز ورودی، مشاهده رفتار برنامه در اشکال‌زدا و ذخیره خطاهای احتمالی، اعمال نشده است و تنها تأثیر مدل‌ها، تنوع تولید داده از آنها و میزبان‌ها، بر روی میزان پوشش کد مورد نظر است.
  
  \hspace{0.5mm}
  برای مقایسه با 
  \lr{baseline\_sou}
  تعداد 1000 شیء داده‌ای از هر مدل تولید کرده و برای مقایسه با 
  \lr{baseline\_mou}
  تعداد 3000 شیء داده‌ای با هر مدل تولید می‌کنیم. 
  تولید 1000 فایل
  \gls{PDF}
  جدید با استفاده از مدل‌ها برای حالت \gls{SOU}  (با اندازه بافر
  100\footnote{یعنی هر بار یک لیست از 100 شیء توسط مدل مولد بازگردانیده می‌شود.}) 
  به‌طور میانگین 60 دقیقه و برای حالت \gls{MOU} به‌طور میانگین 190 دقیقه به‌طول انجامید. در پایان پوشش کد نرم‌افزار \lr{MuPDF} روی هریک از 72 مجموعه 1000 فایلی اندازه‌گیری شد. ما همچنین ادغام پوشش‌های کد را برای هر تنوع در قالب \lr{host123} اندازه ‌گرفتیم. اندازه‌گیری پوشش کد برای هر مجموعه نیز در حدود 60 دقیقه زمان برد. کلیه نتایج به تفکیک میزبان‌ها در شکل‌های \ref{ch5_host1_max_crop} تا \ref{ch5_host123_crop.pdf}  ذکر شده است.
 
 
 
 \begin{figure}%[hbt]
	\centering
 	%\begin{longtable}{c}
 %% \lr{host1\_max}
 %\subfigure[برچسب 1]{%[ht]%[tbh!]%%[t!]
 \begin{subfigure}{\linewidth}
 	\centering
 	\includegraphics[width=\textwidth, clip=true,  trim= 0 0 0 0]{chapter5/ch5_host1_max_crop.pdf}
 	\caption[نمودار تغییرات پوشش کد در تنوع‌های نمونه‌برداری $0.5$ تا $1.5$  برای \_max\lr{host1}]
 	{
 		نمودار تغییرات پوشش کد در تنوع‌های نمونه‌برداری $0.5$ تا $1.5$  برای  \lr{host1\_max}.
 	}
 	\label{ch5_host1_max_crop}
 	%\ref{ch5_host1_max_crop}
 \end{subfigure}
	\vspace{0.75cm}

%% \lr{host2\_min}
  %\subfigure[برچسب 1]{%[ht]%[tbh!]%%[t!]
  \begin{subfigure}{\linewidth}

 	\centering
 	\includegraphics[width=\textwidth, clip=true,  trim= 0 0 0 0]{chapter5/ch5_host2_min_crop.pdf}
 	\caption[ تغییرات پوشش کد در تنوع‌های نمونه‌برداری $0.5$ تا $1.5$  برای  \lr{host2\_min}]
 	{
 		تغییرات پوشش کد در تنوع‌های نمونه‌برداری $0.5$ تا $1.5$  برای \lr{host2\_min}.
 	}
 	\label{ch5_host2_min_crop.pdf}
 	%\ref{ch5_host2_min_crop.pdf}
\end{subfigure}
\vspace{1cm}

 \caption[نمودار تغییرات پوشش کد مدل‌های مختلف برحسب تنوع.]{
 	 نمودار تغییرات پوشش کد مدل‌های مختلف برحسب تنوع.   
  }
 
\end{figure}


 \begin{figure}%[hbt]
 	\ContinuedFloat
 
%% \lr{host3\_avg}
%\subfigure[برچسب 1]{%[ht]%[tbh!]%%[t!]
 \begin{subfigure}{\linewidth}
	\centering
	\includegraphics[width=\textwidth, clip=true,  trim= 0 0 0 0]{chapter5/ch5_host3_avg_crop.pdf}
	\caption[ تغییرات پوشش کد در تنوع‌های نمونه‌برداری $0.5$ تا $1.5$  برای  \lr{host3\_avg}]
	{
		تغییرات پوشش کد در تنوع‌های نمونه‌برداری $0.5$ تا $1.5$  برای \lr{host3\_avg}.
	}
	\label{ch5_host3_avg_crop.pdf}
	%\ref{ch5_host3_avg_crop.pdf}
	
\end{subfigure}
 \vspace{0.75cm}
 
 %% \lr{host123}
 %\subfigure[برچسب 4]{%[ht]%[tbh!]%%[t!]
 \begin{subfigure}{\linewidth}
 	\centering
 	\includegraphics[width=\textwidth, clip=true,  trim= 0 0 0 0]{chapter5/ch5_host123_crop.pdf}
 	\caption[تغییرات پوشش کد در تنوع‌های نمونه‌برداری $0.5$ تا $1.5$  برای  \lr{host123}]
 	{
 	تغییرات پوشش کد در تنوع‌های نمونه‌برداری $0.5$ تا $1.5$  برای \lr{host123}.
 	}
 	\label{ch5_host123_crop.pdf}
 	%\ref{ch5_host123_crop.pdf}

\end{subfigure}
\vspace{1cm}
%\end{longtable}

\caption[]{(ادامه.)
	نمودار تغییرات پوشش کد مدل‌های مختلف برحسب تنوع.    
}

 \end{figure}




 \subsubsection{مشاهدات}
 \begin{itemize}
 	\item {
 	پوشش کد داده‌های تولیدی از پوشش کد مبنا در اکثر موارد کمتر است؛ زیرا، اشیای تولید شده به خوش‌شکلی اشیای واقعی نیستند. البته دربرخی موارد شاهد افزایش پوشش کد هستیم. از جمله برای  \lr{host2\_min}
 در حالت \gls{MOU}.
}
\item{
تولید داده با تنوع 1 در بیشتر مدل‌ها و روی اکثریت میزبان‌ها پوشش کد بهتری داشته است. یعنی پارامتر تنوع واقعاً سبب بدشکل شدن و بالا رفتن گوناگونی داده‌های تولید شده و در نتیجه پارامــتری مؤثر بوده است.
	
}

\item{
	
	افزایش تنوع در مدل \gls{LSTM} دوسویه در حالت کلی سبب افزایش پوشش کد شده است؛ اما در دیگر مدل‌ها خیر.

}

\item{
تقریباً در همه نمودارها داده‌های تولید شده توسط مدل 2، پوشش کد بیشینه را نتیجه داده است. یعنی مدل‌های ژرف ساده بهتر از مدل‌های ژرف پیچیده همچون \lr{LSTM} دوسویه (مدل 4)، عمل کرده‌اند. به‌همین دلیل مدل 2 با تنوع تولید داده 1 پیروز نهایی این سری از آزمایش‌های ما هستند.

}

 \end{itemize}


\subsection{مقایسه با مدل کدگذار-کدگشا}

برای مقایسه با 
\cite{Godefroid:2017:LML:3155562.3155573}
ابتدا مدل کدگذار-گشای توضیح داده شده را پیاده‌سازی و آن 50 دوره آموزش دادیم. سپس 1000 فایل \lr{PDF} را با این مدل تولید و پوشش کد مجموع آنها را انداز‌ه‌گیری کردیم. چون راهبرد نمونه‌برداری برای مدل مذکور نیز بهترین روش گزارش شده است، ما نیز از نمونه‌برداری برای تولید داده‌ها با این مدل استفاده کرده‌ایم. برای مدل‌های چهارگانه خود نیز بالاترین پوشش کد کسب شده در میان همه تنوع‌های آزمایش شده در بخش \ref{sec:gen_model_cove} را به‌عنوان  نماینده پوشش کد برای هر مدل انتخاب کردیم. در نهایت  دو حالت \gls{SOU} و \gls{MOU} را به‌صورت مجزا مورد ارزیابی و مقایسه قرار دادیم. نتایج در شکل \ref{ch5_cmp_laf} نشان داده شده است. شکل \ref{ch5_cmp_laf_sou_crop.pdf} حالت \gls{SOU} و شکل  \ref{ch5_cmp_laf_mou_crop.pdf} حالت \gls{MOU} را نشان می‌دهد.

 
 \begin{figure}%[hbt]
 	\centering
 	%% SOU
	\begin{subfigure}{\linewidth}
		
		\includegraphics[width=\textwidth, clip=true,  trim= 0 0 0 0]{chapter5/ch5_cmp_laf_sou_crop.pdf}
		\caption
		{
			حالت \gls{SOU}.
		}
		\label{ch5_cmp_laf_sou_crop.pdf}
		%\ref{ch5_cmp_laf_sou_crop.pdf}
		
	\end{subfigure}
	\vspace{0.75cm}
	
	%% MOU
	\begin{subfigure}{\linewidth}
		\includegraphics[width=\textwidth, clip=true,  trim= 0 0 0 0]{chapter5/ch5_cmp_laf_mou_crop.pdf}
		\caption
		{
			حالت \gls{MOU}.
		}
		\label{ch5_cmp_laf_mou_crop.pdf}
		%\ref{ch5_cmp_laf_mou_crop.pdf}
		
	\end{subfigure}
	\vspace{1cm}
	%\end{longtable}
	
	\caption[]{
		    نمودار پوشش کد‌ مدل‌های مختلف در مقایسه با مدل \lr{laf} برحسب فایل‌های میزبان‌.
	    
    }
	\label{ch5_cmp_laf}
\end{figure}%


\subsubsection{مشاهدات}
\begin{itemize}
	\item{
در حالت \gls{SOU} و برای \lr{host1\_max} همه مدل‌های پیشنهادی پوشش کد بالاتری داشته‌اند. در همین حالت دو میزبان دیگر اختلاف پوشش کد ناچیز بوده است. اما در حالت اجتماع پوشش کدها یعنی \lr{host123} مدل‌های پیشنهادی بهتر ظاهر شده‌اند یعنی هر مدل برای هر میزبان مجموعه دستورات متفاوتی را اجرا کرده است.
}

	\item{
	در حالت \gls{MOU} مدل‌های پیشنهادی ما با اختلاف بهتر هستند. این نشان می‌دهد که تغییر بیشتر فایل‌های میزبان به افزایش پوشش کد، در تعداد داده آزمون مساوی، می‌انجامد. بنابراین حالت \gls{MOU} را برای انجام آزمون فازی توصیه می‌کنیم. 
}
\item{
	در هر دو حالت اختلاف بین پوشش کد مدل‌ها برای \lr{host2\_min} کمتر و برای \lr{host1\_max} بیشتر به‌چشم می‌خورد. این امر نشان می‌دهد که انتخاب فایل میزبان در مواردی که نیاز به آن هست، مانند فایل \gls{PDF} که ساختار پیچیده‌ای دارد و یادگیری کامل آن به‌آسانی محقق نمی‌شود، بسیار حائز اهمیت است و تأثیر چشم‌گیری روی ارتقاء پوشش کد دارد. این انتخاب نبایست تصادفی انجام شود، چون دیدیم افزایش پوشش کد رابطه مستقیمی با پوشش کد فایل میزبان دارد. بنابراین فایل میزبان با پوشش کد بیشتر را برای انجام آزمون فازی توضیه می‌کنیم.
	
}

\end{itemize}


\subsection{مقایسه در دوره‌های مختلف}\label{sec:epochcompare}

یک پارامتر قابل ارزیابی که در جدول \ref{tabel:all_parameters} مطرح کردیم و تأثیر آن روی پوشش کد در 
\cite{Godefroid:2017:LML:3155562.3155573}
نیز بررسی شده است تعداد دوره‌های آموزش است. به نظر می‌رسد که پوشش کد ارتباط مستقیمی با تعداد دوره‌های آموزش مدل ژرف داشته باشد. برای این منظور پوشش کد 1000 فایل تولید شده با استفاده از مدل 2 را در سه دوره $10$، $30$ و $50$ اندازه‌گیری کردیم. سپس همین کار را برای مدل \lr{laf} نیز انجام دادیم. برای آن که نتایج قابل اطمینان باشند و اثر پارامترهای تصادفی از بین برود، هر آزمایش را سه مرتبه با سه مجموعه داده مجزا تکرار و میانگین پوشش کدها اندازه گرفتیم. نتایج در شکل \ref{ch5_cmp_epochs_crop.pdf} نشان داده شده‌اند.



 \begin{figure}%[ht]%[ht]%[tbh!]%%[t!]
	\centering
	\includegraphics[width=\textwidth, clip=true,  trim= 0 0 0 0]{chapter5/ch5_cmp_epochs_crop.pdf}
	\caption
	{
		نمودار تغییرات پوشش کد برحسب دوره برای مدل‌های 2 و \lr{laf}.
	}
	\label{ch5_cmp_epochs_crop.pdf}
	%\ref{ch5_cmp_epochs_crop.pdf}
\end{figure}

\subsubsection{مشاهدات}
\begin{itemize}
	\item{
برای هر دو مدل پوشش کد از دوره 10 به 30 افزایش و سپس در دوره 50 کاهش یافته است. یعنی رابطه مستقیم معناداری بین پوشش کد و تعداد دوره‌های آموزش مدل وجود ندارد. 	

}
\item{
با بررسی مقدار خطای مدل در هر دوره متوجه شدیم تا زمانی که نرخ کاهش خطای مدل در فرایند آموزش زیاد است، پوشش کد نیز افزایش می‌یابد؛ چراکه ساختار فایل بهتر یادگیری شده و در نتیجه مدل قادر به تولید داده‌های خوش‌شکل‌تری خواهد بود. هنگامی که نرخ کاهش خطا از یک حد آستانه کمتر می‌شود، افزایش پوشش کد نیز ثابت شده یا اندکی کاهش می‌یابد. 
}

\item{
	در همه دوره‌های شکل \ref{ch5_cmp_epochs_crop.pdf} پوشش کد مدل پیشنهادی از مدل \lr{laf} بیش‌تر بوده است که نشان می‌دهد این مدل ساختار فایل را بهتر یادگرفته است. 
}
	
\end{itemize}

\subsection{آزمون فازی عصبی}
در پنجمین و آخرین آزمایش نرم‌افزار \lr{MuPDF} را با استفاده از فازر پیشنهادی در بخش \ref{ch4_iust_deep_fuzzer_crop.pdf}، مورد آزمون فازی قرار دادیم. برای این منظور الگوریتم‌های فاز عصبی داده و فاز عصبی فراداده (الگوریتم‌های \ref{alg:data_neural_fuzz} و \ref{alg:metadata_neural_fuzz}) را که در بخش \ref{sec:neural_fuzzing_algorithms} معرفی کرده بودیم، پیاده‌سازی و با استفاده از هریک از آنها تعداد $10000$ فایل \gls{PDF} را ایجاد کردیم. تنظیمات پارامترهای ورودی و ثوابت موجود در الگوریتم‌های پیشنهادی به قرار جدول 
\ref{table:alg-inputs}
 است. ستون آخر جدول
\ref{table:alg-inputs}،
همچنین مقادیر مجاز برای هریک از ورودی‌ها و ثوابت داده شده را نشان می‌دهد. حداکثر طول یک شیء داده‌ای 
\lr{PDF}
را در بازه متغیر 450 تا 550 انتخاب کرده‌ایم؛ زیرا، به طور میانگین طول اشیای استخراج شده مجموعه‌های آموزش و آزمون، در این بازه قرار دارد.
در آزمایش‌های آتی خود در نظر داریم تا آزمون فازی را با مقادیر متنوع انتخابی از مجموعه‌های داده شده انجام دهیم و بدین ترتیب قادر خواهیم بود تا اثر هریک از این پارامترها را به طور دقیق‌تری بررسی کنیم.


ما همچنین نسخه‌هایی از الگوریتم‌های \ref{alg:random_fuzz} 
\cite{Sutton:2007:FBF:1324770}
و \ref{alg:sample_fuzz} 
\cite{Godefroid:2017:LML:3155562.3155573}
را پیاده‌سازی و آزمون فازی را با تولید داده آزمون از طریق این الگوریتم‌ها نیز انجام دادیم. برای حالت تصادفی (الگوریتم \ref{alg:random_fuzz}) از ابزار \lr{FileFuzz} که یک فازر تصادفی تحت سیستم عامل ویندوز است و در بخش \ref{sec:fuzzer} معرفی شد، استفاده کردیم. متأسفانه در روش یادگیری و فاز 
\cite{Godefroid:2017:LML:3155562.3155573}،
تقریباً اکثر ابرپارامترهای مورد نیاز در هنگام آزمایش‌ها نامشخص هستند و نویسندگان اشاره‌ای به مقادیر آنها نکرده‌اند. برای این هریک از این ابرپارامترها نظیر 
$d$،
ما همان مقدار استفاده شده در الگوریتم‌های روش ‌پیشنهادی را استفاده کردیم، تا بدین ترتیب شرایط یکسانی بر آزمایش‌ها حاصل باشد. در همه آزمایش‌ها، \lr{host1\_max} به‌عنوان میزبان\footnote{در تولید داده آزمون به روش تصادفی و روش مبتنی برجابه‌جایی، به‌جای میزبان، به یک یا تعدادی دانه اولیه نیاز داریم که در اینجا از همان  \lr{host1\_max} استفاده گردید.}
 استفاده شد. نتایج حاصــل از پوشش کد روش‌های مختلف در جدول \ref{tabel:neural_fuzzing_result} آمده است. همچنین جدول \ref{tabel:neural_fuzzing_result_compare} میزان بهبود در پوشش کد روش پیشنهادی در مقایسه با کارهای مرتبط را نشان می‌دهد. 



\begin{table}
	\caption{مقادیر مجاز و مقادیر داده شده به ثوابت و ورودی‌های الگوریتم‌های فازی عصبی داده و فاز عصبی فرا داده در هنگام آزمون فازی قالب فایل \lr{PDF} }
	\centering
	\label{table:alg-inputs}
	\begin{tabular}{@{}rrr@{}}
		\toprule[1.5pt]
		\multicolumn{1}{r}{پارامتر ورودی / ثابت}                           & \multicolumn{1}{r}{مقادیر استفاده شده} & \multicolumn{1}{r}{مقادیر مجاز}                      \\ \midrule[1.5pt]
		\begin{tabular}[c]{@{}r@{}}\lr{Learnt model M}\end{tabular}  & 2    & \begin{tabular}[c]{@{}r@{}}$1,2,3,4, laf$\end{tabular}    \\
		\begin{tabular}[c]{@{}r@{}}\lr{Sequence prefix P}\end{tabular}      & انتخاب از مجموعه آزمون          & \begin{tabular}[c]{@{}l@{}} ثابت رشته‌ای\end{tabular} \\
		\begin{tabular}[c]{@{}r@{}}\lr{Diversity D}\end{tabular}            & 1                               & \begin{tabular}[c]{@{}r@{}}$(0, +\infty)$\end{tabular}           \\
		\begin{tabular}[c]{@{}r@{}}\lr{Fuzzing rate FR}\end{tabular}        & $0.10$                             & \begin{tabular}[c]{@{}r@{}}$(0,1]$\end{tabular}        \\
		\begin{tabular}[c]{@{}r@{}}\lr{End token ET}\end{tabular}           & \lr{endobj}                          & \begin{tabular}[c]{@{}r@{}}ثابت رشته‌ای\end{tabular} \\
		\begin{tabular}[c]{@{}r@{}}\lr{Binary token BT}\end{tabular}        & stream                          & \begin{tabular}[c]{@{}l@{}}ثابت رشته‌ای\end{tabular} \\
		$(a,b)$                                                              & $(450,550)$                       & \begin{tabular}[c]{@{}r@{}}$(len(P), +\infty)$\end{tabular}    \\
		\begin{tabular}[c]{@{}r@{}}$\alpha$   (\lr{DataNeuralFuzz})\end{tabular}     & $0.50$                            & \begin{tabular}[c]{@{}l@{}}$(0,1)$\end{tabular}          \\
		\begin{tabular}[c]{@{}l@{}}$\beta$   (\lr{MetadataNeuralFuzz})\end{tabular} & $0.90$ & \begin{tabular}[c]{@{}l@{}}$(0,1)$\end{tabular}          \\ \bottomrule[1.5pt]
	\end{tabular}
\end{table}



\begin{table}%[ht]
	\caption[نتایج پوشش کد حاصل از آزمون فازی نرم‌افزار \lr{MuPDF} با الگوریتم‌های تولید داده آزمون مختلف]
	{
		نتایج پوشش کد حاصل از آزمون فازی نرم‌افزار \lr{MuPDF} با الگوریتم‌های تولید داده آزمون مختلف.‌
	}
	\label{tabel:neural_fuzzing_result}
	\centering
	\onehalfspacing
	\begin{tabularx}{1.0\linewidth}{r r r r r}
		\toprule[1.5pt] 
		الگوریتم تولید داده آزمون / معیار &
		 پوشش بلوک پایه &
		 درصد \hspace{15mm} &
		 پوشش خط‌کد &
		 درصد
		\\
		\midrule[1.5pt] 
		\lr{DataNeuralFuzz} &
		$23719$ &
		$19.36$ &
		$18673$ &
		$20.81$
		\\
		%\hline 
		\lr{MetadataNeuralFuzz} &
		$22583$ &
		$18.43$ &
		$17894$ &
		$19.95$
		\\
		%\hline 
		\lr{SampleFuzz} \cite{Godefroid:2017:LML:3155562.3155573} &
		$20957$ &
		$17.10$ &
		$16793$ &
		$18.72$
		\\
		%\hline 
		\lr{RandomFuzz (FileFuzz)} \cite{Sutton:2007:FBF:1324770} &
		$7563$ &
		$6.17$ &
		$5002$ &
		$5.58$
		\\
		\bottomrule[1.5pt]
		
	\end{tabularx} 
\end{table}




\begin{comment}


\begin{table}%[ht]
	\caption[میزان بهبود پوشش کد ابزار  \lr{MuPDF}توسط الگوریتم‌های روش پیشنهادی در مقایسه با کارهای مرتبط.]
	{
		میزان بهبود پوشش کد ابزار \lr{MuPDF} توسط الگوریتم‌های روش پیشنهادی در مقایسه با کارهای مرتبط. اعداد داخل جدول به‌صورت درصد هستند. مقادیر هر خانه برابر اختلاف پوشش کد حاصله از الگوریتم‌های ستون و سطر مربوط به آن مقدار است.
	}
	\label{tabel:neural_fuzzing_result_compare2}
	\centering
	\onehalfspacing
	\begin{tabularx}{0.95\linewidth}{r p{35mm} p{35mm}}
		\toprule[1.5pt] 
		الگوریتم تولید داده آزمون &
		\lr{DataNeuralFuzz} &
		\lr{MetadataNeuralFuzz}
		\\
		\midrule[1.5pt] 
	    \lr{SampleFuzz} \cite{Godefroid:2017:LML:3155562.3155573} &
		$+2.26$ &
		$+1.83$
		\\
		%\hline 
		\lr{AFL} \cite{DBLP:journals/corr/abs-1711-04596} &
		$+7.73$ &
		$+6.80$
		\\
		%\hline 
		\lr{Augmented AFL } \cite{DBLP:journals/corr/abs-1711-04596} &
		$+7.56$ &
		$+6.63$	
		\\
		%\hline 
		\lr{RandomFuzz (FileFuzz)} \cite{Sutton:2007:FBF:1324770} &
		$+13.19$ &
		$+12.26$
		\\
		\bottomrule[1.5pt]
		
	\end{tabularx} 
\end{table}
\end{comment}

%\vspace{1cm}

\begin{table}
	\caption[میزان بهبود پوشش کد ابزار  \lr{MuPDF}توسط الگوریتم‌های روش پیشنهادی در مقایسه با کارهای مرتبط.]
	{
		میزان بهبود پوشش کد ابزار \lr{MuPDF} توسط الگوریتم‌های روش پیشنهادی در مقایسه با کارهای مرتبط. اعداد داخل جدول به‌صورت درصد هستند. مقادیر هر خانه برابر اختلاف پوشش کد حاصله از الگوریتم‌های ستون و سطر مربوط به آن مقدار است.
	}
	\label{tabel:neural_fuzzing_result_compare}
	\centering
	\onehalfspacing
	\begin{tabular}{@{}rrr@{}}
		\cmidrule[1.5pt](l){2-3}
		& \multicolumn{2}{c}{روش پیشنهادی}                                                                               \\ \midrule[1.5pt]
		\multicolumn{1}{r}{روش‌های موجود}           & \multicolumn{1}{r}{\lr{DataNeuralFuzz}} & \multicolumn{1}{r}{\lr{MetaDataNeuralFuzz}} \\ \midrule
		\lr{SampleFuzz} \cite{Godefroid:2017:LML:3155562.3155573}            & $+2.26$                                                  & $+1.83$                                                      \\
		\lr{AFL} \cite{DBLP:journals/corr/abs-1711-04596}                  & $+7.73$                                                  & $+6.80$                                                      \\
		\lr{AugmentedAFL} \cite{DBLP:journals/corr/abs-1711-04596}         & $+7.56$                                                  & $+6.63$                                                      \\
		\lr{RandomFuzz (FileFuzz)} \cite{Sutton:2007:FBF:1324770}& $+13.19$                                                 & $+12.26$                                                     \\ \bottomrule[1.5pt]
	\end{tabular}
\end{table}




\subsubsection{مشاهدات}

\begin{itemize}
	\item{
الگوریتم فاز عصبی فراداده به پوشش کد کمتری دست پیدا کرده است، زیرا همان‌طور که انتظار می‌رود دستکاری بخش کوچکی در از قالب فایل ممکن است آن را کاملاً نامعتبر کند و در مرحله اولیه تجزیه توسط تجزیه‌گر رد شود. بنابراین الگوریتم در رسیدن به هدف خود یعنی فاز فراداده موفق بوده است.  	
}
\item{
هر دو الگوریتم فاز عصبی داده و فاز عصبی فراداده پوشش کد بهتری نسبت به
\lr{SampleFuzz}
داشته‌اند که نشان از عملکرد بهتر مد‌ل‌های پیشنهادی در یادگیری ساختار فایل و عملکرد بهتر الگوریتم‌های پیشنهادی در فاز کردن این فایل‌ها دارد. پوشش کد حاصل شده در این آزمایش‌ها همچنین از پوشش کد گزارش شده در 
\cite{DBLP:journals/corr/abs-1711-04596}
که حاصل از آزمون فازی \lr{MuPDF} با  \lr{AFL} و \lr{AFL} افزوده بیشتر است. اعداد مربوط به آنها قبلاً در بخش \ref{sec:augmented_afl_problems} ذکر شدند. جدول \ref{tabel:neural_fuzzing_result_compare} در این بخش نیز اختلاف این مقادیر را با مقادیر حاصل از پوشش کد روش پیشنهادی نشان داده است.
	 
}
\item{
به‌وضوح می‌توان برتری روش‌های هوشمند تولید داده آزمون را نسبت به روش‌های تصادفی مشاهده کرد. همانطور که در ابتدای این پایان‌نامه گفتیم روش‌های تصادفی در ساختارهای پیچیده پوشش کد بسیار کمی را نتیجه می‌دهند. پوشش کد الگوریتم فاز عصبی داده بیش از 3 برابر الگوریتم فاز تصادفی است.
}

\item{
باوجود استفاده از هوش‌مصنوعی و الگوریتم‌های هوشمند در تولید داده آزمون همچنان شاهد پوشش کد پایینی (کمتر از 20 درصد) در پایان عملیات آزمون هستیم. به‌نظر می‌رسد رسیدن به پوشش کد بالا در ساختارهای پیچیده، با آزمون فازی جعبه سیاه کار دشواری باشد. از طرفی فنون آزمون جعبه سفید مانند روش‌های اجرای نمادین نیز روی این ساختارها به‌علت پیچیدگی بالا و وجود محدودیت‌های فراوان سخت، زمان‌بر و تا حد زیادی غیرممکن است و کماکان آزمون فازی مؤثرتر بوده است.   
}
\end{itemize}



\subsubsection{خطا‌ها و آسیب‌پذیری‌های شناسایی شده}
در بررسی گزارش‌های تولید شده توسط \lr{Application Verifier} پس از هر آزمون، هیچ‌گونه خطایی مشاهده نکردیم. باتوجه به‌اینکه ما نسخه نهایی نرم‌افزار \lr{MuPDF} را مورد آزمایش قرار دادیم، تصور می‌شود که بیشتر خطا‌های آن در نسخه‌های آزمایشی برطرف شده باشد و در نتیجه پیدا کردن خطای جدید سخت خواهد بود. از طرفی \lr{MuPDF} نرم‌افزاری تحت توسعه فعال و جامعه توسعه‌دهنده و کاربری بزرگی است که سبب می‌شود تا از کیفیت خوبی برخوردار باشد. با این حال الگوریتم فاز عصبی داده چندین مورد استفاده از توابع ناامن را شناسایی کرد که \lr{Application Verifier} آنها را درقالب هشدارهای امنیتی اعلام کرده است. 

پیش از آزمایش‌های اصلی ما فازر و \lr{Application Verifier} روی نرم‌افزارهای کوچکی که خطای آنها مشخص بود اجرا کردیم. به‌نظر می‌رسد \lr{Application Verifier} روی ویندوز 10 نسخه 64 بیتی قادر به تشخیص خطاهای حافظه برنامه‌های 32 بیتی نیست؛ زیرا موفق به شناسایی این خطاها نشدیم. برای برنامه‌های 64 بیتی اما این مشکل وجود نداشت. لذا ما هر دو نسخه 32 و 64 بیتی \lr{MuPDF} را مورد آزمون قرار دادیم. 

فازر، برنامه \lr{PDF}خوان را با داده آزمون ورودی باز و بعد از 10 ثانیه آن را بسته و برای تزریق داده آزمون بعدی اقدام می‌کند. همزمان پوشش کد و گزارش وضعیت حافظه نیز ثبت و ذخیره می‌شوند. بنابراین زمان آزمون برای هر مجموعه 10000 فایلی حدود 28 ساعت به‌طول انجامید\footnote{\lr{Application Verifier}
		 و ابزارگذاری کد سربارهایی را به زمان هربار اجرای برنامه اضافه می‌کنند. همچنین فاصله زمانی کوتاهی بین تزریق داده‌های آزمون متوالی درنظر گرفته شد؛ زیرا، در مواردی سقوط برنامه ناشی از خطای دیگر بخش‌های سیستم است. با لحاظ این زمان‌ها از سقوط‌های این چنینی با اطمینان زیادی جلوگیری می‌نماییم.
	  }. 
برای همه نرم‌افزارهای دیگر آزمون به‌این شیوه قابل انجام خواهد بود. گفتیم که آزمون فازی نوعی آزمون فشار است. در نظر داریم تا آزمون فازی با این روش را با استفاده از $100$ هزار فایل تولید شده و حتی بیشتر انجام دهیم. در این صورت امکان سقوط برنامه \lr{MuPDF} افزایش می‌یابد.



\subsubsection{تنوع داده‌های تولید شده}
مدل‌های ما در روش پیشنهادی قادر به تولید داده‌های متنوع و به‌صورت کنترل شده بدشکل هستند. الگوریتم فاز عصبی داده به‌خوبی در تغییر محتویات فایل‌های 
\lr{PDF}
عمل کرده و می‌تواند فایل‌هایی با داده‌های جدید بدون تغییر در ساختار ایجاد کند. شکل \ref{ch5_amazing_generated_test_data.png} نمونه‌ای از داده‌های تولید شده توسط الگوریتم فاز عصبی داده را نشان می‌دهد. همان‌طورکه در این شکل پیداست فایل‌های \lr{PDF} در عین معتبر بودن حاوی داده‌های جدیدی هستند که با مقادیر مرزی در فرایند تولید داده آزمون جایگذاری شده‌اند. به‌دلیل معتبر بودن تعداد بیشتری از فایل‌های تولید شده توسط الگوریتم فاز عصبی داده پوشش کد این الگوریتم از پوشش کد الگوریتم فاز عصبی فراداده بیشتر است. اما هر دو الگوریتم مورد نیاز هستند؛ زیرا، هرکدام مرحله مجزایی از مراحل پردازش فایل در کد برنامه را هدف آزمون قرار می‌دهند.



\begin{figure}[ht]%[ht]%[tbh!]%%[t!]
	\centering
	\includegraphics[width=0.95\textwidth, clip=true,  trim= 0 0 0 0]{chapter5/ch5_amazing_generated_test_data.png}
	\caption[نمونه‌ای از داده‌های آزمون متنوع تولید شده توسط الگوریتم فاز عصبی داده در فرایند آزمون فازی قالب فایل \lr{PDF}.]
	{
		نمونه‌ای از داده‌های آزمون متنوع تولید شده توسط الگوریتم فاز عصبی داده در فرایند آزمون فازی قالب فایل \lr{PDF}. در هر ردیف، سمت چپ‌ترین فایل، فایل میزبان را نشان می‌دهد و 3 فایل سمت راست فایل‌های حاصل از مدل مولد هستند. تغییر در محتوا به وضوح قابل مشاهده است.
	}
	\label{ch5_amazing_generated_test_data.png}
	%\ref{ch5_cmp_epochs_crop.pdf}
\end{figure}





\section{خلاصه}
در این فصل، ابتدا پارامترهای قابل بررسی و مؤثر در روش‌های تولید داده آزمون مبتنی بر یادگیری ژرف را معرفی و سپس روش پیشنهادی خود را که در فصل \ref{ch:4} مطرح کرده بودیم، روی نرم‌افزار \lr{MuPDF} و برای مهمترین این پارامترها مورد آزمایش و ارزیابی قرار دادیم. نتایج در حالت کلی حاکی از بهبود میزان پوشش کد در آزمون فازی \lr{MuPDF} هنگام استفاده از مدل‌های و الگوریتم‌های تولید داده آزمون پیشنهاد شده است. به‌خصوص نسبت به تعدادی از روش‌های قبلی از جمله 
\cite{Godefroid:2017:LML:3155562.3155573}
که مشابه‌ترین کار مرتبط بود، آمار و ارقام بهتری هم در دقت مدل‌ها و هم در پوشش کد مشاهده می‌شود. علاوه‌بر این نتیجه‌گیری کلی آزمایش‌های مختلف ما چندین واقعیت تجربی دیگر را مشخص کرد که مهمترین آنها عبارتند از:

\begin{itemize}
	\item {
\gls{LSTM} 
دوسویه به‌عنوان مدل زبانی می‌تواند به دقت بیشتر و خطای کمتری روی مجموعه داده دست پیدا کند.	با این حال مدل‌های ژرف ساده مانند \gls{LSTM} یک‌سویه بدون لایه‌های \lr{Dropout} توانستند مدل‌های پیچیده‌تری مثل \gls{LSTM} دوسویه را در آزمون فازی شکست دهند. نتیجه‌گیری مشابهی در 
\cite{DBLP:journals/corr/abs-1711-04596}
بیان شده است.

}
\item{
فایل‌های \gls{PDF}ای که پوشش کد ببیشتری دارند، هنگام تغییر نحوه بروزرسانی افزایشی نیز، اختلاف پوشش کد بیشتری نسبت به دیگر فایل‌ها فراهم می‌کنند.

}

	\item {
 تنوع تولید داده $1$ برای بیشتر مدل‌ها منجربه پوشش کد بهتری می‌شود.
}
\item{
افزایش دوره‌های آموزش لزوماً به افزایش پوشش کد نمی‌انجامد اما تا زمانی که خطای مدل‌ها را به‌نحو خوبی کاهش دهد، پوشش کد را افزایش می‌دهد.
}

\item{
	روش‌های ترکیبی تولید داده آزمون مانند الگوریتم‌های فاز عصبی داده و فاز عصبی فراداده، که بخش‌های دودویی را نیز در فرایند آزمون فازی شرکت می‌دهند منجربه پوشش کد بیشتری می‌شوند.
}

\item{
	آزمون فازی با روش‌های تولید داده هوشمند، آزمون فازی تصادفی را همواره شکست می‌دهد ولی هنوز هم روی ساختارهای خیلی پیچیده به پوشش کد آن‌چنان بالایی منتهی نمی‌گردد، این امر ارزشمند بودن کار بر روی روش‌های تولید داده آزمون در آینده را نشان‌ می‌دهد. البته دلیل پوشش کد پایین برای نرم‌افزار 
	\lr{MuPDF}
	بیشتر آن است که این نرم‌افزار قالب‌های فایل دیگری غیر از 
	\lr{PDF}
	را پشتیبانی می‌کند. لذا اجرای آن با فایل‌های 
	\lr{PDF}
	 تنها کدهای مربوط به تجزیه و پرداخت 
	 \lr{PDF}
	 را سبب می‌شود. بنابراین بایستی قالب‌های دیگر را نیز در فرایند آزمون شرکت داد که بدون شک منجر به افزایش پوشش کد می‌شود.
}


\end{itemize}

اثر تعدادی از پارامترهای دیگر مطرح شده در جدول \ref{tabel:all_parameters}مانند راهبردهای تولید داده از مدل، در این آزمایش‌ها بررسی نشدند که جای دارد آزمایش‌هایی برای آنها نیز طراحی کرد. با این حال باتوجه به محدودیت‌های که برای هریک از دیگر راهبردها مثل راهبرد حریصانه برشمرده شد، راهبرد نمونه‌برداری مناسب‌ترین راهبرد به‌نظر می‌رسد. 








 % فصل پنجم: زمان‌بندی پیشنهادی %
\clearpage
 % مراجع %
%\pagestyle{empty}
%\pagestyle{fancy}
%\fancyhead[LE,RO]{\slshape \rightmark}
\fancyhead[LO, RE]{\slshape }
\onehalfspacing
%\fancyhead[LE,RO]{\slshape ک}
%\fancyhead[LO,RE]{\slshape م}
	% تعیین قالب درج مراجع %
\bibliographystyle{ieeetr-fa}%{acm-fa}%{chicago-fa}%{plainnat-fa}%
\bibliography{bibitems/references-all.bib}

%\pagestyle{empty}

\clearpage
% ضمائم %
\appendix % فصل های پس از این قسمت به عنوان ضمیمه خواهند آمد. %
% -- اگر پیوست اول خود را در فایلی به جز appendix1 همراه با این کلاس نوشته‌اید، باید چندخط اول appendix1 را در فایل خود کپی کنید.
%\pagestyle{fancy}
%\fancyhead[LE,RO]{\slshape \rightmark}
\fancyhead[LO,RE]{\slshape \leftmark}
%% !TeX root=z-main.tex
% appendix1
% دستورات زیر باید در اولین فایل پیوست باشند. آنها را حذف نکنید!
\addtocontents{toc}
{
    \protect\renewcommand\protect\cftchappresnum{\appendixname~}%
    \protect\setlength{\cftchapnumwidth}{\mylenapp}
}%
    
\chapter{ساختار فایل }\label{appendix:1}
\thispagestyle{empty}


%% !TeX root=z-main.tex

\chapter{فازر }\label{appendix:2}
\thispagestyle{empty}



%\baselineskip=.75cm
\clearpage

\fancyhead[LO, RE]{\slshape}
\onehalfspacing
\printglossary %Glossary and index %
%\phantomsection
%\addcontentsline{toc}{chapter}{\indexname}
%\printindex
\clearpage
\bookmark[dest=latintitle]{Latin Abstract and Title}
% !TeX root=maintext.tex
% !TeX TS-program = XeLaTeX
% !TEX spellcheck = fa
% Title English
% Latin abtrsaction and other info
% By: Morteza ZAKERI

% در این فایل، عنوان پایان‌نامه، مشخصات خود و چکیده پایان‌نامه را به انگلیسی، وارد کنید.

%%%%%%%%%%%%%%%%%%%%%%%%%%%%%%%%%%%%
\baselineskip=1.25cm

\begin{latin}
	
	\latinuniversity{Iran University of Science and Technology}
	\latinfaculty{School of Computer Engineering}
	\latinsubject{Computer Engineering}
	\latinfield{Software}
	\hypertarget{latintitle}{}
    
	\latintitle{\begin{doublespace}
            Thesis Latin Title
        \end{doublespace}
    }  
    
	\firstlatinsupervisor{Dr.  }
	%\secondlatinsupervisor{Second Supervisor}
	\firstlatinadvisor{Dr.  }
	%\secondlatinadvisor{Second Advisor}
	\latinname{First Name}
	\latinsurname{Last Name}
	\latinthesisdate{February 2021}
	
	\latinkeywords{Keywords}
	\en-abstract{
	Put your latin abstract here.
     %%
}
\latinfirstPage

\end{latin}
 % Thesis english title %


\end{document}