
%%
% Last update: 2024-07-10
%%
% در این فایل، دستورها و تنظیمات مورد نیاز، آورده شده است.
%%%%%%%%%%%%%%%%%%%%%%%%%%%%%%%%%%%%%
% دستوراتی که پوشه پیش‌فرض زیرفایل‌های tex را مشخص می‌کند.
%\makeatletter
%\def\input@path{{./tex/}}
%\makeatother
\RequirePackage{flafter}
\RequirePackage[bottom]{footmisc}
%\RequirePackage{enumerate}
\RequirePackage{enumitem}
\setlist[itemize]{noitemsep,  nolistsep}
% در ورژن جدید زی‌پرشین برای تایپ متن‌های ریاضی، این سه بسته، حتماً باید فراخوانی شود
\RequirePackage{amsthm, amsmath, amssymb, amsfonts}
%\RequirePackage{amssymb, amsmath, amsfonts}
\RequirePackage{thmtools}
%\RequirePackage{thmbox}


% بسته‌ای برای تنطیم حاشیه‌های بالا، پایین، چپ و راست صفحه
\RequirePackage[top=28mm, bottom=22mm, left=16mm, right=24mm]{geometry}
\RequirePackage[T1]{fontenc}
%\RequirePackage{titlesec}
\RequirePackage{setspace}
\RequirePackage{pdflscape}
%\RequirePackage{lscape}
\RequirePackage{titling}


%\RequirePackage{color}
\RequirePackage[x11names, usenames, dvipsnames, svgnames, table]{xcolor}

%\RequirePackage{bm}
\RequirePackage{verbatim}
%\RequirePackage{rotating}
%\RequirePackage{lettrine}
%\renewcommand{\LettrineTextFont}{}

% بسته‌‌ای برای ظاهر شدن شکل‌ها و تعیین آدرس تصاویر
\RequirePackage[final]{graphicx}
\graphicspath{{./figs/}}
\RequirePackage{caption}
%\RequirePackage{subcaption}
%\RequirePackage{subfig}

\RequirePackage{float}
\RequirePackage{xparse}
\RequirePackage{listing}
\RequirePackage[final]{listings}
\lstset{inputpath=./codes/}



\RequirePackage{soulutf8}

% بسته‌ای برای رسم کادر
%\RequirePackage{framed}  %%%%13990905
% بسته‌‌ای برای چاپ شدن خودکار تعداد صفحات در صفحه «معرفی پایان‌نامه»
\RequirePackage{lastpage}

\RequirePackage{cite}	% Add by Morteza %

% بسته‌ٔ لازم برای: ۱. تغییر شماره‌گذاری صفحات پیوست. ۲. تصحیح باگ آدرس وب حاوی '%' در مراجع
\RequirePackage{etoolbox}

\RequirePackage{comment}

%%%%%%%%%%%%%%%%%%%%%%%%%%%%%%%%%%%%
%%% دستورات وابسته به استیل مراجع:
%% اگر از استیل‌های natbib (plainnat-fa، asa-fa، chicago-fa) استفاده می‌کنید، خط زیر را فعال و بعدی‌اش را غیرفعال کنید.
%\RequirePackage{natbib}
%\newcommand{\citelatin}[1]{\cite{#1}\LTRfootnote{\citeauthor*{#1}}}
%\newcommand{\citeplatin}[1]{\citep{#1}\LTRfootnote{\citeauthor*{#1}}}
%% اگر از سایر استیل‌ها استفاده می‌کنید، خط بالا را غیرفعال و خط‌های زیر را فعال کنید.
\let\citep\cite
\let\citelatin\cite
\let\citeplatin\cite

\RequirePackage{bibentry}

%%%%%%%%%%%%
%\RequirePackage[nostamp]{draftwatermark}
% بررسی حالت پیش نویس
\RequirePackage{ifdraft}
\ifdraft
{%
% بسته‌ٔ ایجاد لینک‌های رنگی با امکان جهش
	\RequirePackage[unicode=true,pagebackref=true, colorlinks, linkcolor=blue, citecolor=red, bookmarks=true, final]{hyperref}
	%\RequirePackage{todonotes}
	%\RequirePackage[firstpage]{draftwatermark}
	\RequirePackage{draftwatermark}
	\SetWatermarkText{\ \ \ پیش‌نویس}
	\SetWatermarkScale{1.2}
}
{
	% اضافه کردن امکان ایجاد لینک به صفحه به جای عنوان در فهرست‌ها
    %SpringGreen4
    %LimeGreen
	\RequirePackage[unicode=true, pagebackref=false, colorlinks, linkcolor=Blue, citecolor=SpringGreen4, urlcolor=RoyalBlue4, bookmarks=true, linktocpage, breaklinks]{hyperref} %RoyalBlue1 % FireBrick
	%\RequirePackage[disable]{todonotes} % final without TODOs
	%\RequirePackage[firstpage]{draftwatermark}
	%\RequirePackage{draftwatermark}
	%\SetWatermarkColor[RGB]{240, 240, 245}
	%\SetWatermarkLightness{0.9813}
	%\SetWatermarkText{مرتـضی ‌ذاکــری}
	%\SetWatermarkText{\ \ \ پیش‌نویس}
	%\SetWatermarkScale{0.65}
}


\RequirePackage[obeyDraft]{todonotes}
\setlength{\marginparwidth}{2cm}

% تعیین مشخصات فایل PDF
\hypersetup{
	pdftitle={Measuring and Improving Testability of Software Systems Artifacts},
	pdfauthor={Morteza ZAKERI},
	pdfsubject={Ph.D. dissertation},
	pdfkeywords={Software artifacts; testability; covergaeability; software smell; automated refactoring; intelligent software engineering; testabilty-driven development.},
    pdfdirection={R2L}
}


%%%%%%%%%%%%
%%% تصحیح باگ: اگر در مراجع، آدرس وب حاوی '%' بوده و pagebackref فعال باشد، دستورات زیر باید بیایند:
%% برای استیل‌های natbib مثل plainnat-fa، asa-fa، chicago-fa
\makeatletter
\let\ORIG@BR@@lbibitem\BR@@lbibitem
\apptocmd\ORIG@BR@@lbibitem{\endgroup}{}{}
\def\BR@@lbibitem{\begingroup\catcode`\%=12 \ORIG@BR@@lbibitem}
\makeatother
%% برای سایر استیل‌ها
\makeatletter
\let\ORIG@BR@@bibitem\BR@@bibitem
\apptocmd\ORIG@BR@@bibitem{\endgroup}{}{}
\def\BR@@bibitem{\begingroup\catcode`\%=12 \ORIG@BR@@bibitem}
\makeatother
%%%%%%%%%%%%%%%%%%%%%%%%%%%%%%%%%%%%



% بسته‌ و دستوراتی برای ایجاد لینک‌های رنگی با امکان جهش
%\RequirePackage[pagebackref=false,colorlinks,linkcolor=blue,citecolor=purple]{hyperref}
% چنانچه قصد پرینت گرفتن نوشته خود را دارید، خط بالا را غیرفعال و  از دستور زیر استفاده کنید چون در صورت استفاده از دستور زیر‌‌، 
% لینک‌ها به رنگ سیاه ظاهر خواهند شد که برای پرینت گرفتن، مناسب‌تر است
%\RequirePackage[pagebackref=false]{hyperref}
% بسته‌ لازم برای تنظیم سربرگ‌ها
\RequirePackage{fancyhdr}
%\RequirePackage{setspace}
% بسته‌های لازم برای نوشتن الگوریتم
\RequirePackage{algorithm}
\RequirePackage{algorithmic}
% Add by Morteza %
%\RequirePackage[linesnumbered,ruled,vlined]{algorithm2e}
% بسته‌های لازم برای رسم بهتر جداول
\RequirePackage{tabulary}
\RequirePackage{tabularx}
%% Add by Morteza
\RequirePackage{multirow}
\RequirePackage{threeparttable}
\RequirePackage{fourier-otf} 
\RequirePackage{array}
\RequirePackage{makecell}  % I use this %
\RequirePackage{booktabs}
\RequirePackage{hhline}
 \RequirePackage{vcell}
 \RequirePackage{colortbl}
 \RequirePackage{arydshln}
%\RequirePackage{graphicx}

%%
% بسته‌های لازم برای رسم تنظیم بهتر شکل‌ها و زیرشکل‌ها
\RequirePackage[export]{adjustbox}
%\RequirePackage{subfigure}

%\RequirePackage[subfigure]{tocloft}
\RequirePackage{tocloft}
%\RequirePackage{subfig}
%\RequirePackage{caption}
%\RequirePackage{subcaption}

%%%%%%%%%%%%%%%%%%%%%%%%%%%%%%%%%%%%
% بسته‌ای برای رسم نمودارها و نیز صفحه مالکیت اثر
\RequirePackage{tikz}
\RequirePackage{tikzsymbols}
\usetikzlibrary{calendar,fpu}
% بسته‌ای برای ظاهر شدن «مراجع» و «نمایه» در فهرست مطالب
\RequirePackage[nottoc]{tocbibind}
% دستورات مربوط به ایجاد نمایه
\RequirePackage{makeidx}
\makeindex
%%% بسته ایجاد واژه‌نامه با xindy
\RequirePackage[xindy, acronym, nonumberlist=true]{glossaries}


%%%%%%%%%%%%%%%%%%%%%%%%%%
% Use the package sectsty to do change the headings, and xcolor to get the colour definitions.
\RequirePackage{sectsty}
%\titleformat{\chapter}[display]{\titlefont\Huge}{\chaptertitlename\ \tartibi{chapter}}{1em}{}
%\titleformat{\chapter}[display]
%{}{{\Huge \filright\chaptertitlename\ \thechapter}}
%{%30pt}{\filright}

\sectionfont{\fontsize{11}{13}\selectfont}
%\sectionfont{\fontsize{11}{13}\yagut}
%Coloring chapter{} and sections{}
%\chapterfont{\raggedleft}
\chapterfont{\color{DodgerBlue4}  \Huge\Huge \raggedleft}  % sets colour of chapters
\sectionfont{\color{DeepSkyBlue4}}  % sets colour of sections
\subsectionfont{\color{DeepSkyBlue3}}  % sets colour of sections

%% Set the font-size of Figure and Table captions
\DeclareCaptionFont{mysize}{\fontsize{10.5}{9.45}\yagut}
\captionsetup{font=mysize}

\captionsetup[figure]{labelfont={color=Brown4, bf}, textfont={color=DarkSlateGrey}} %DarkSlateGrey %darkgray
\captionsetup[table]{labelfont={color=Brown4, bf},  textfont={color=DarkSlateGray}}  %%Brown4 % Indigo

\RequirePackage{titlesec}
%\RequirePackage{indentfirst}
%% Set spacing
%\setlength{\textfloatsep}{\baselineskip}
\setlength{\textfloatsep}{4pt}
%\setlength{\parindent}{0pt}
 
\titlespacing*\section{0pt}{0pt plus 1pt minus 1pt}{0pt plus 1pt minus 1pt}
\titlespacing*\subsection{0pt}{0pt plus 1pt minus 1pt}{0pt plus 1pt minus 1pt}
\titlespacing*\subsubsection{0pt}{0pt plus 1pt minus 1pt}{0pt plus 1pt minus 1pt}
%
\setlength{\abovedisplayskip}{-1pt}
\setlength{\belowdisplayskip}{-1pt}

%% End of set spacing

%\setkomafont{title}{\huge \linespread{1.5}}%
\pretitle{\begin{center}\linespread{1.5}\huge }
\posttitle{\par\end{center}\vspace{0.5em}}



% بسته زیر باگ ناشی از فراخوانی بسته‌های زیاد را برطرف می‌کند.
\RequirePackage{morewrites}

%%%%%%%%%%%%%%%%%%%%%%%%%%
% فراخوانی بسته زی‌پرشین و تعریف قلم فارسی و انگلیسی
% قلم فارسی بازنویسی و قابل حمل شده است. امکان استفاده از چندین فونت فارسی مختلف در این قسمت فراهم شده است. برای استفاده نیازی به نصب بودن فونت ها روی رایانه خود ندارید.

%\RequirePackage[extrafootnotefeatures]{xepersian}
\RequirePackage{xepersian}

\settextfont[Path={./font/Niloofar/}, BoldFont={XBNiloofarBd.ttf}, ItalicFont={XBNiloofarIt.ttf}, BoldItalicFont={XBNiloofarBdIt.ttf}, Scale=1.05]{XBNiloofar.ttf}

\setlatintextfont[Scale=0.90]{Times New Roman}



%-----------------
%% https://github.com/persiantex/xepersian/issues/17
% this is a temporary fix as stated in https://github.com/persiantex/xepersian/issues/17#issuecomment-709956867
\ExplSyntaxOn
\cs_set_eq:NN
\etex_iffontchar:D
\tex_iffontchar:D
\cs_undefine:N \c_one
\int_const:Nn \c_one { 1 } 
\ExplSyntaxOff
%-----------------
% چنانچه می‌خواهید اعداد در فرمول‌ها، انگلیسی باشد، خط زیر را غیرفعال کنید
%\setdigitfont[Path={./font/Zar/}, BoldFont={XBZarBd.ttf}, ItalicFont={XBZarIt.ttf}, BoldItalicFont={XBZarBdIt.ttf}, Scale=1.0]{XBZar.ttf}
\setmathdigitfont[Path={./font/Zar/}, BoldFont={XBZarBd.ttf}, ItalicFont={XBZarIt.ttf}, BoldItalicFont={XBZarBdIt.ttf}, Scale=1.0]{XBZar.ttf}

% اگر می‌خواهید که اعداد با فونت یکان نمایش داده شوند خط بالا را غیر فعال کرده و خط زیر را فعال کنید.

%\setdigitfont[Path={./font/Yekan/}, BoldFont={XMYekanBd.ttf}, ItalicFont={XMYekanIt.ttf}, BoldItalicFont={XMYekanBdIt.ttf}, Scale=1.0]{XMYekan.ttf}%{Persian Modern}

% تعریف قلم‌های فارسی و انگلیسی برای استفاده در بعضی از قسمت‌های متن
\defpersianfont\niloofar[Path={./font/Niloofar/}, BoldFont={XBNiloofarBd.ttf}, ItalicFont={XBNiloofarIt.ttf}, BoldItalicFont={XBNiloofarBdIt.ttf}, Scale=1.0]{XBNiloofar.ttf}

\defpersianfont\zar[Path={./font/Zar/}, BoldFont={XBZarBd.ttf}, ItalicFont={XBZarIt.ttf}, BoldItalicFont={XBZarBdIt.ttf}, Scale=1.0]{XBZar.ttf}

\defpersianfont\yagut[Path={./font/Yagut/}, BoldFont={XBYagutBd.ttf}, ItalicFont={XBYagutIt.ttf}, BoldItalicFont={XBYagutBdIt.ttf}, Scale=1.0]{XBYagut.ttf}

\defpersianfont\titlefont[Path={./font/Titre/}, BoldFont={XBTitreShadow.ttf}, ItalicFont={XBTitreIt.ttf}, BoldItalicFont={XBTitreShadowIt.ttf}, Scale=1.0]{XBTitre.ttf}

\defpersianfont\iranic[Path={./font/Zar/XBZarOblique/}, BoldFont={XBZarObliqueBd.ttf}, Scale=1.10]{XBZarOblique.ttf}%Italic}%

\defpersianfont\nastaliq[Path={./font/IranNastaliq/}, Scale=1.50]{IranNastaliq.ttf}







% بسته زیر و متعاقباً دستورات ادامه آن، برای برای ریست کردن شماره پانویس‌ها در هر صفحه قرار داده شده است (به درخواست دکتر محمد عبداللهی ازگمی)

%\RequirePackage[perpage]{footmisc}
\RequirePackage{zref-perpage}
\zmakeperpage{footnote}

%\RequirePackage[rldocument]{bidi}
\RequirePackage{bidi}
\makeatletter
\@bidi@removefromreset{footnote}{chapter}
\makeatother


% راستچین شدن todonotes
\presetkeys{todonotes}{align=right, textdirection=righttoleft}{}
\makeatletter
\providecommand\@dotsep{5}
\def\listtodoname{فهرست کارهای باقیمانده}
\def\listoftodos{\noindent{\Large\vspace{10mm}\textbf{\listtodoname}}\@starttoc{tdo}}
\renewcommand{\@todonotes@MissingFigureText}{شکل}
\renewcommand{\@todonotes@MissingFigureUp}{شکل}
\renewcommand{\@todonotes@MissingFigureDown}{جاافتاده}
\makeatother
% دستوری برای حذف کلمه «چکیده»
%\renewcommand{\abstractname}{}
% دستوری برای حذف کلمه «abstract»
%\renewcommand{\latinabstract}{}
% دستوری برای تغییر نام کلمه «اثبات» به «برهان»
\renewcommand\proofname{\textbf{برهان}}
% دستوری برای تغییر نام کلمه «کتاب‌نامه» به «مراجع»
\renewcommand{\bibname}{مراجع}
% دستوری برای تعریف واژه‌نامه انگلیسی به فارسی
\newcommand\persiangloss[2]{#1\dotfill\lr{#2}\\}
% دستوری برای تعریف واژه‌نامه فارسی به انگلیسی 
\newcommand\englishgloss[2]{#2\dotfill\lr{#1}\\}
% تعریف دستور جدید «\پ» برای خلاصه‌نویسی جهت نوشتن عبارت «پروژه/پایان‌نامه/رساله»
\newcommand{\پ}{پروژه/پایان‌نامه/رساله}

%\newcommand\BackSlash{\char`\\}

%%%%%%%%%%%%%%%%%%%%%%%%%%
\SepMark{-}


% تعریف و نحوه ظاهر شدن عنوان قضیه‌ها، تعریف‌ها، مثال‌ها و ...
\declaretheorem[style=remark]{xyz}

\declaretheorem[
style=plain,
thmbox={style=S, bodystyle=\normalfont\noindent},
name=مسئله,
within=chapter,
]{problem}

%\declaretheorem[
%style=plain,
%thmbox={style=S, bodystyle=\normalfont\noindent},
%name=راه‌حل,
%]{solution}

\declaretheorem[
style=remark,
thmbox={style=M, bodystyle=\normalfont, headstyle=\itshape Remark \upshape\theremark},
name=Remark,
within=section,
qed=$\blacksquare$
]{remark}

%\newtheorem[L]{problem}{مسئله}[section]
%\theoremstyle{plain}
\theoremstyle{definition}
%\newtheoremstyle{sltheorem}
%{}                % Space above
%{}                % Space below
%{\slshape}        % Theorem body font % (default is "\upshape")
%{}                % Indent amount
%{\bfseries}       % Theorem head font % (default is \mdseries)
%{.}               % Punctuation after theorem head % default: no punctuation
%{ }               % Space after theorem head
%{}                % Theorem head spec

%\theoremstyle{sltheorem}
%\newtheorem{theorem}{Theorem}
%\declaretheorem[style=definition,name=Definition,qed=$\blacksquare$]{sol}
\renewcommand{\qedsymbol}{$\heartsuit$}
%\newtheorem{isolution}{راه‌حل پیشنهادی}
\newtheorem*{solution*}{راه‌حل پیشنهادی}
%\newenvironment{\solution}{\begin{\isolution}}{\qed \end{isolution}}

\newtheorem{definition}{تعریف}[section]
%\theoremstyle{theorem}
%\newtheorem{theorem}[definition]{قضیه}
%\newtheorem{lemma}[definition]{لم}
%\newtheorem{proposition}[definition]{گزاره}
%\newtheorem{corollary}[definition]{نتیجه}
%\newtheorem{remark}[definition]{ملاحظه}
%\theoremstyle{definition}
%\newtheorem{example}[definition]{مثال}

%\renewcommand{\theequation}{\thechapter-\arabic{equation}}
%\def\bibname{مراجع}
\numberwithin{algorithm}{chapter}
\def\listalgorithmname{فهرست الگوریتم‌ها}
\def\listfigurename{فهرست شکل‌ها}
\def\listtablename{فهرست جدول‌ها}



%%%%%%%%%%%%%%%%%%%%%%%%%%%%
% دستورهایی برای سفارشی کردن سربرگ صفحات
% \newcommand{\SetHeader}{
% \csname@twosidetrue\endcsname
% \pagestyle{fancy}
% \fancyhf{} 
% \fancyhead[OL,EL]{\thepage}
% \fancyhead[OR]{\small\rightmark}
% \fancyhead[ER]{\small\leftmark}
% \renewcommand{\chaptermark}[1]{%
% \markboth{\thechapter-\ #1}{}}
% }
%%%%%%%%%%%%
%\def\MATtextbaseline{1.5}
%\renewcommand{\baselinestretch}{\MATtextbaseline}
%\fancyhead[LE,RO]{\slshape \rightmark}
%\fancyhead[LO,RE]{\slshape \leftmark}
%\fancyfoot[C]{\thepage}
\doublespacing
%%%%%%%%%%%%%%%%%%%%%%%%%%%%%
% دستوراتی برای اضافه کردن کلمه «فصل» در فهرست مطالب

\newlength\mylenprt
\newlength\mylenchp
\newlength\mylenapp

\renewcommand\cftpartpresnum{\partname~}
\renewcommand\cftchappresnum{\chaptername~}
\renewcommand\cftchapaftersnum{:}

\settowidth\mylenprt{\cftpartfont\cftpartpresnum\cftpartaftersnum}
\settowidth\mylenchp{\cftchapfont\cftchappresnum\cftchapaftersnum}
\settowidth\mylenapp{\cftchapfont\appendixname~\cftchapaftersnum}
\addtolength\mylenprt{\cftpartnumwidth}
\addtolength\mylenchp{\cftchapnumwidth}
\addtolength\mylenapp{\cftchapnumwidth}

\setlength\cftpartnumwidth{\mylenprt}
\setlength\cftchapnumwidth{\mylenchp}	

\makeatletter
{\def\thebibliography#1{\chapter*{\refname\@mkboth
   {\uppercase{\refname}}{\uppercase{\refname}}}\list
   {[\arabic{enumi}]}{\settowidth\labelwidth{[#1]}
   \rightmargin\labelwidth
   \advance\rightmargin\labelsep
   \advance\rightmargin\bibindent
   \itemindent -\bibindent
   \listparindent \itemindent
   \parsep \z@
   \usecounter{enumi}}
   \def\newblock{}
   \sloppy
   \sfcode`\.=1000\relax}}


%اگر مایلید در شماره گذاری حرفی و ابجد به جای آ از الف استفاده شود دستورات زیر را فعال کنید.   
%\def\@Abjad#1{%
%  \ifcase#1\or الف\or ب\or ج\or د%
%           \or هـ\or و\or ز\or ح\or ط%
%           \or ی\or ک\or ل\or م\or ن%
%           \or س\or ع\or ف\or ص%
%           \or ق\or ر\or ش\or ت\or ث%
%            \or خ\or ذ\or ض\or ظ\or غ%
%            \else\@ctrerr\fi}
%
% \def\abj@num@i#1{%
%   \ifcase#1\or الف\or ب\or ج\or د%
%            \or هـ‍\or و\or ز\or ح\or ط\fi

%   \ifnum#1=\z@\abjad@zero\fi}   
%  
%   \def\@harfi#1{\ifcase#1\or الف\or ب\or پ\or ت\or ث\or

% ج\or چ\or ح\or خ\or د\or ذ\or ر\or ز\or ژ\or س\or ش\or ص\or ض\or ط\or ظ\or ع\or غ\or

% ف\or ق\or ک\or گ\or ل\or م\or ن\or و\or ه\or ی\else\@ctrerr\fi}

\makeatother


%%%%%%%%%%
% Packages add by Morteza Zakeri to support code snips M.Sc. Thesis 13970415


% بسته‌های مورد نیاز برای نوشتن کدها، رنگ‌آمیزی آنها و تعیین پوشهٔ کدها

\definecolor{mygreen}{rgb}{0,0.6,0}
\definecolor{mygray}{rgb}{0.2,0.2,0.2}
\definecolor{mymauve}{rgb}{0.58,0,0.82} 

%\ProvidesFile{listings-xepersian.def}[2014/07/17 v0.3 bilingual captions for listings package]
%\def\lstlistingname{\if@RTL else\ Listing\fi}
%\def\lstlistlistingname{\if@RTL  else\  Listings\fi}

\lstdefinestyle{myStyle}{ 
	backgroundcolor=\color{white},   % choose the background color; you must add \RequirePackage{color} or \RequirePackage{xcolor}; should come as last argument
	basicstyle=\ttfamily, % whole listing /w verbatim font
	breakatwhitespace=false,         % sets if automatic breaks should only happen at whitespace
	breaklines=true,                 % sets automatic line breaking
	captionpos=b,                    % sets the caption-position to bottom
	commentstyle=\color{mygreen},    % comment style
	deletekeywords={...},            % if you want to delete keywords from the given language
	escapeinside={\%*}{*)},          % if you want to add LaTeX within your code
	extendedchars=true,              % lets you use non-ASCII characters; for 8-bits encodings only, does not work with UTF-8
	frame=tb,	                     % adds a frame around the code
	lineskip=.095cm,                   % space between code lines
	keepspaces=true,                 % keeps spaces in text, useful for keeping indentation of code (possibly needs columns=flexible)
	keywordstyle=\color{blue},       % keyword style
	language=Octave,                 % the language of the code
	morekeywords={*,...},            % if you want to add more keywords to the set
	numbers=left,                    % where to put the line-numbers; possible values are (none, left, right)
	numbersep=3pt,                   % how far the line-numbers are from the code
	numberstyle=\tiny\color{mygray}\lr, % the style that is used for the line-numbers
	rulecolor=\color{black},         % if not set, the frame-color may be changed on line-breaks within not-black text (e.g. comments (green here))
	showspaces=false,                % show spaces everywhere adding particular underscores; it overrides 'showstringspaces'
	showstringspaces=false,          % underline spaces within strings only
	showtabs=false,                  % show tabs within strings adding particular underscores
	stepnumber=1,                    % the step between two line-numbers. If it's 1, each line will be numbered
	stringstyle=\ttfamily\color{mymauve},     % string literal style
	tabsize=2,	                     % sets default tabsize to 2 spaces
	title=\lstname,                  % show the filename of files included with \lstinputlisting; also try caption instead of title
	captionpos=b,
	captiondirection=RTL
}

\lstdefinestyle{myStyle2}{
	basicstyle=\ttfamily, % whole listing /w verbatim font
	keywordstyle=\color{blue}\bfseries, % bold black keywords
	identifierstyle=, % nothing happens
	commentstyle=\color{LimeGreen}, % green comments
	stringstyle=\ttfamily\color{red}, % red typewriter font for strings
	showstringspaces=false % no special string spaces
	breaklines=true,
	breakatwhitespace=false,
	numbers=right, % line number formats
	numberstyle=\footnotesize\lr,
	numbersep=-10pt,
	frame=single,
	captionpos=b,
	captiondirection=RTL
}

\lstset{style=myStyle} % command to set default style
%\def\listingsfont{\ttfamily}
\def\lstlistingname{\rl{برنامه}}


%\def\lstlistingname{\rl{تکه کد}}


% for numbering subsubsections
%\setcounter{secnumdepth}{3}
% to include subsubsections in the table of contents
%\setcounter{tocdepth}{3}




%% Set LTRfootnote number to persian
\makeatletter
\def\LTRfootnote{\@ifnextchar[\@xLTRfootnote{\stepcounter\@mpfn
        \protected@xdef\@thefnmark{\persianfont\thempfn}%
        \@footnotemark\@LTRfootnotetext}}
\makeatother


%\RequirePackage{hyperref}
\RequirePackage{csquotes}
\RequirePackage{epigraph}
\RequirePackage{etoolbox}
\newcommand{\epigraphcolor}{DarkKhaki}
\makeatletter
\newlength\epitextskip
\pretocmd{\@epitext}{\color{NavajoWhite4} \yagut\itshape }{}{} %DimGray
\makeatother
%
\newcommand{\epigraphrulecolor}{DarkKhaki}
\makeatletter
\renewcommand{\@epirule}{{\color{\epigraphrulecolor}\rule[.75ex]{\epigraphwidth}{\epigraphrule}}}
\makeatother
%

\setlength\epigraphwidth{0.80\linewidth}
\renewcommand{\epigraphsize}{\small}
\renewcommand{\epigraphflush}{flushleft}
\renewcommand{\sourceflush}{flushleft}
\renewcommand{\textflush}{flushepinormal}

%\RequirePackage{polyglossia}
%\RequirePackage[algo2e,linesnumbered,ruled,vlined]{algorithm2e}
\RequirePackage{textgreek}
\RequirePackage{booktabs}  % professional-quality tables
\RequirePackage{nicefrac}  % compact symbols for 1/2, etc.
\RequirePackage{microtype}  % microtypography
\RequirePackage{textcomp}
\RequirePackage[algo2e, ruled, linesnumbered, resetcount]{algorithm2e}
\def\HiLi{\leavevmode\rlap{\hbox to \hsize{\color{yellow!50}\leaders\hrule height .8\baselineskip depth .5ex\hfill}}}

%\RequirePackage[printwatermark]{xwatermark}
\RequirePackage{watermark}
%\RequirePackage{bm}
\RequirePackage{subcaption}
%\RequirePackage{multicol}
%\renewcommand\thesubfigure{\harfi{subfigure}}
\renewcommand{\thesubfigure}{\harfi{subfigure}}

%%%PDF setting
\RequirePackage[open, openlevel=1, atend]{bookmark}[2022/08/30]
\bookmarksetup{color=blue}
\bookmark[dest=titlefa]{سنجش و بهبود خودکار آزمون‌پذیری مصنوعات سیستم‌های نرم‌افزاری}

%\bookmark[dest=acknowledgement]{درباره نسخه پیش‌نویس}
%\bookmark[dest=acknowledgment]{قدردانی}
\bookmark[dest=abstractfa]{چکیده}
\bookmark[dest=tableofcontent]{فهرست مطالب}

% end of packages add by Morteza
%%%%%%%%%


