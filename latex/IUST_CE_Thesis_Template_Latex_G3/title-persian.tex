% !TeX root=maintext.tex
% !TeX TS-program = XeLaTeX
% !TEX spellcheck = fa
% Farsi Title
% Persian absteract and other info
% در این فایل، عنوان پایان‌نامه، مشخصات خود، متن تقدیمی‌، ستایش، سپاس‌گزاری و چکیده پایان‌نامه را به فارسی، وارد کنید.
% توجه داشته باشید که جدول حاوی مشخصات پروژه/پایان‌نامه/رساله و همچنین، مشخصات داخل آن، به طور خودکار، درج می‌شود.
%%%%%%%%%%%%%%%%%%%%%%%%%%%%%%%%%%%%
% دانشگاه خود را وارد کنید
\university{علم و صنعت ایران}
% دانشکده، آموزشکده و یا پژوهشکده  خود را وارد کنید
\faculty{دانشکده مهندسی کامپیوتر}
% گروه آموزشی خود را وارد کنید
\department{گروه نرم‌افزار}
% گروه آموزشی خود را وارد کنید
\subject{مهندسی کامپیوتر}
% گرایش خود را وارد کنید

\field{نرم‌افزار}
% عنوان پایان‌نامه را وارد کنید
\title{
    عنوان پروژه / پایان‌نامه / رساله
 }
% نام استاد(ان) راهنما را وارد کنید
\firstsupervisor{دکتر ...}
%\secondsupervisor{استاد راهنمای دوم}
% نام استاد(دان) مشاور را وارد کنید. چنانچه استاد مشاور ندارید، دستور پایین را غیرفعال کنید.
\firstadvisor{دکتر ...}
%\secondadvisor{استاد مشاور دوم}
% نام دانشجو را وارد کنید
\name{نام}
% نام خانوادگی دانشجو را وارد کنید
\surname{نام خانوادگی}
% شماره دانشجویی دانشجو را وارد کنید
\studentID{97923000}
% تاریخ پایان‌نامه را وارد کنید
\thesisdate{اسفند 1399}
% به صورت پیش‌فرض برای پایان‌نامه‌های کارشناسی تا دکترا به ترتیب از عبارات «پروژه»، «پایان‌نامه» و »رساله» استفاده می‌شود؛ اگر  نمی‌پسندید هر عنوانی را که مایلید در دستور زیر قرار داده و آنرا از حالت توضیح خارج کنید.
%\projectLabel{پایان‌نامه}

% به صورت پیش‌فرض برای عناوین مقاطع تحصیلی کارشناسی تا دکترا به ترتیب از عبارات «کارشناسی»، «کارشناسی ارشد» و »دکتری» استفاده می‌شود؛ اگر  نمی‌پسندید هر عنوانی را که مایلید در دستور زیر قرار داده و آنرا از حالت توضیح خارج کنید.
\degree{دکتری}

\firstPage

\besmPage
\davaranPage

% در این قسمت اسامی اساتید راهنما، مشاور و داور باید به صورت دستی وارد شوند.
\begin{center}
\begin{tabularx}{0.95\linewidth}{|c|c|c|c|c|c|}
\hline
\makecell[c]{ردیف} & 
\makecell[c]{سمت} & 
\makecell[c]{نام و نام‌خانوادگی} & 
\makecell[c]{مرتبه دانشگاهی } & 
\makecell[c]{دانـشگاه / مؤسـسه } & 
\makecell[c]{~~~~~~~امضــاء~~~~~~~~~} \\ \hline
	1 & 
	\makecell[c]{استاد \\ راهنمای اول} & 
	\makecell[c]{دکتر \\ ...} &
 	\makecell[c]{دانشیار} &
	\makecell[c]{دانشگاه \\ علم و صنعت ایران} & 
    \makecell[c]{~ ~ ~ ~ ~ ~ ~ ~ ~} \\ \hline
	2 & 
	\makecell[c]{استاد \\ راهنمای دوم} &
	\makecell[c]{---} &
	\makecell[c]{---}&
	\makecell[c]{---} &	
     \makecell[c]{~ ~ ~ ~ ~ ~ ~ ~ ~} \\ \hline
	3 & 
	\makecell[c]{استاد \\ مشاور اول } &
	\makecell[c]{دکتر \\ ...} &
   \makecell[c]{استادیار} &
	\makecell[c]{دانشگاه \\ علم و صنعت ایران} &
	 \makecell[c]{~ ~ ~ ~ ~ ~ ~ ~ ~} \\ \hline
	4 & 
	\makecell[c]{استاد \\ مشاور دوم } &
	\makecell[c]{---} &
	\makecell[c]{---} &
	\makecell[c]{---} &	
     \makecell[c]{~ ~ ~ ~ ~ ~ ~ ~ ~} \\ \hline
	5 & 
	\makecell[c]{استاد \\ مدعو داخلی} & 
	\makecell[c]{دکتر \\ ...} & 
    \makecell[c]{استادیار} &
	\makecell[c]{دانشگاه \\ علم و صنعت ایران} & 
	 \makecell[c]{~ ~ ~ ~ ~ ~ ~ ~ ~} \\ \hline
	6 &	 
	\makecell[c]{استاد \\ مدعو داخلی} & 
	\makecell[c]{---} & 
	\makecell[c]{---} &
	\makecell[c]{دانشگاه \\ علم و صنعت ایران} & 
	 \makecell[c]{~ ~ ~ ~ ~ ~ ~ ~ ~} \\ \hline
	7 &	
	\makecell[c]{استاد \\ مدعو خارجی} &
	\makecell[c]{دکتر \\ ...} & 
   \makecell[c]{استادیار} &
	\makecell[c]{دانشگاه \\ ...} & 
     \makecell[c]{~ ~ ~ ~ ~ ~ ~ ~ ~} \\ \hline
	8 &	
	\makecell[c]{استاد \\ مدعو خارجی} &
	\makecell[c]{---} & 
   \makecell[c]{---} &
	\makecell[c]{---} & 
      \makecell[c]{~ ~ ~ ~ ~ ~ ~ ~ ~} \\ \hline
\end{tabularx}
\end{center}

\esalatPage
\mojavezPage

% -- چنانچه مایل به چاپ صفحات «تقدیم»، «نیایش» و «سپاس‌گزاری» در خروجی نیستید، خط‌های زیر را با گذاشتن ٪ در ابتدای آنها غیرفعال کنید.
% -- پایان‌نامه خود را حتماً تقدیم کنید! مرتضی :)

\newpage
\thispagestyle{empty}
%\watermark{مرتضی}
\centerline{\Large \titlefont  تقـــدیم }
\begin{center}
	محل قرار گرفتن متن قـدرانی و تقدیم در نــسخه نهایی پایان‌نامه. 
	
\end{center}


% -- متن سپاس‌گزاری
\begin{acknowledgementpage}
    از زحمات استاد گران‌قدر و فرزانه جناب آقای / سرکار خانم 
\myadvisor~
    که راهنمایی اینجانب را در دوره 
\mydegree~
   عهده‌دار بودند، بی‌نهایت سپاس‌گزار هستم. همچنین ...
	%نسخه پیشرو یک نسخه پیش‌نویس و غیرنهایی از پایان‌نامه، صرفاً جهت اهداف داوری بـــوده و غیـر قابل اســـتـنــاد است. برای تسهیل در امر داوری طرح پیشنهادی و کد پروژه در ادامه درج شده است. همچنین در قسمت مراجع، برای هر مرجع شماره صفحاتی که به آن ارجاع داده‌ شده نیز پیوند شده است. این بخش‌ها در نسخه نهایی حذف می‌گردد. 

% با استفاده از دستور زیر، امضای شما، به طور خودکار، درج می‌شود.
\signature 
\end{acknowledgementpage}

%%%%%%%%%%%%%%%%%%%%%%%%%%%%%%%%%%%%
% کلمات کلیدی پایان‌نامه را وارد کنید
\keywords{
    تعداد واژه‌ها يا عبارات كليدی حداكثر می‌تواند پنج كلمه يا عبارت باشد.
}
%چکیده پایان‌نامه را وارد کنید، برای ایجاد پاراگراف جدید از \\ استفاده کنید. اگر خط خالی دشته باشید، خطا خواهید گرفت.
%% 
%% 
\hypertarget{abstractfa}
{\fa-abstract{
        قالب مطابق ضوابط تعریف شده توسط دانشگاه علم و صنعت برای پروژه‌‌های کارشناسی، پایان‌نامه‌های کارشناسی ارشد و رساله‌های دکتری ایجاد شده است. 
        چکیده حداکثر یک صفحه باشد. در متن چکیده، از ارجاع به منابع و اشاره به جداول و نمودارها اجتناب شده و اصطلاح یا كلمه‌ای زیرنویس نشود. در صورت نیاز به معرفی حوزه تحقیق و مبانی نظری آن، حداکثر در پاراگراف اول چکیده ارائه شود. فقط به ارائه‌ روش تحقیق و نتایج نهایی و محوری آن بسنده شده و از ارائه‌ موضوعات و نتایج كلی اجتناب شود. 
        كلمات یا عباراتی كه در این بخش توضیح داده می‌شوند، باید كاملاً محوری و مرتبط با موضوع تحقیق و دست‌آوردهای نظری و عملی آن باشند. 
        در سه قسمت تنظیم شود. نخست شرح مسئله انجام شود. سپس، انگیزه‌ها و اهداف پژوهش بیان شود. در ادامه، روش یا طرح پیشنهادی، روش تحلیل یا ارزیابی نتایج و نتایج بدست آمده در مقایسه با روش‌های و کارهای موجود بیان شود.
%%        
\\}
}\label{abstractfa2} 

\abstractPage
\newpage\clearpage
