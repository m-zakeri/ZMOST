% !TeX document-id = {c214d142-2338-493b-958f-1e80cfa25002}
% !TEX TS-program = XeLaTeX
% Commands for running this example:
% 	 xelatex boxes-with-text-and-math
% End of Commands
\documentclass{article}
\usepackage[left=2.5cm,right=2.5cm]{geometry}
\usepackage{tikz}
\usetikzlibrary{shapes,snakes}
\usepackage{amsmath,amssymb}


\usepackage{xepersian}
\settextfont{Zar}

\begin{document}
\thispagestyle{empty}

% Define box and box title style
\tikzstyle{mybox} = [draw=red, fill=blue!20, very thick,
    rectangle, rounded corners, inner sep=10pt, inner ysep=20pt]
\tikzstyle{fancytitle} =[fill=red, text=white]

\begin{tikzpicture}
\setRTL
\node [mybox] (box){%
    \begin{minipage}{0.50\textwidth}
      این یک معادله هست که در اینجا می‌نویسم تا به شما نشان بدهیم
        \begin{align}
            \dot{n} &= u\cos\psi -v\sin\psi \\
            \dot{e} &= u\sin\psi + v\cos\psi
        \end{align}
        که چگونه می‌توان تعدادی معادله در این جعبه قرار داد تا به زیبایی نوشته کمک کند.
        \begin{align}
            \dot{n} &= u -v\delta_\psi \\
            \dot{e} &= u\delta_\psi + v
        \end{align}
    \end{minipage}
};
\node[fancytitle, left=10pt] at (box.north east) {\hboxR{یک عنوان}};
\node[fancytitle, rounded corners] at (box.west) {$\clubsuit$};
\end{tikzpicture}%

\bigskip
\tikzstyle{mybox} = [draw=blue, fill=green!20, very thick,
    rectangle, rounded corners, inner sep=10pt, inner ysep=20pt]
\tikzstyle{fancytitle} =[fill=blue, text=white, ellipse]
\begin{tikzpicture}[transform shape, rotate=10, baseline=-3.5cm]
\setRTL
\node [mybox] (box) {%
    \begin{minipage}[t!]{0.5\textwidth}
        طبق آخرین قضیه‌ی فِرْما داریم:
        \[
            x^n + y^n = z^n
        \]
        معادلهٔ فوق جواب صفر برای $x$, $y$ و $z$ ندارد وقتیکه $n > 2$.
    \end{minipage}
    };
\node[fancytitle] at (box.north) {\hboxR{آخرین قضیه‌ی فِرْما}};
\end{tikzpicture}
%

\end{document}